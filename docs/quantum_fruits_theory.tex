\documentclass[a4paper,12pt]{article}

% ——— Basic Setup (pdfLaTeX, stable) ———
\usepackage[a4paper,margin=2.0cm]{geometry}
\usepackage[T1]{fontenc}
\usepackage{lmodern}
\usepackage{microtype}
\usepackage{setspace}
\onehalfspacing

% ——— Math & Tools ———
\usepackage{amsmath,amssymb,mathtools,bm}
\usepackage{physics}
\usepackage{hyperref}
\hypersetup{colorlinks=true,linkcolor=blue,citecolor=blue,urlcolor=blue}
\usepackage[most]{tcolorbox}
\tcbset{colback=gray!3, colframe=black!50, arc=2mm, boxsep=1.2mm}

% ——— Short Macros ———
\newcommand{\hbarb}{\hbar}
\newcommand{\sigP}{\sigma_{\!P}}
\newcommand{\lP}{\ell_{\!P}}
\newcommand{\tP}{t_{\!P}}
\newcommand{\aS}{\alpha_{\sigma}}
\newcommand{\asig}{\alpha_{\!\sigma}}
\newcommand{\Ksig}{K_{\sigma_P}}
\newcommand{\Par}{\Pi} % Parallelpropagator
\newcommand{\Syn}{\mathcal{S}} % Synge-World-Function (verwechselungsfrei)
\newcommand{\Krets}{\mathcal{K}} % Kretschmann-Invariant

\title{\textbf{Die natürliche Struktur und Grenze der Raumzeit}\\
\large Eine vereinheitlichte Feldgleichung aus der Planck-Zweimaß \texorpdfstring{$\sigma_P=\ell_P t_P=\hbar G/c^4$}{σP}}
\author{Adrian Zander}
\date{\today}

\begin{document}
\maketitle

\begin{abstract}
Wir entwickeln eine \(\sigma_P\)-regulierte Quantengeometrie als minimalen, einheitlichen Rahmen von der Planckskala bis zur Kosmologie. Die Planck-Zweimaß
\[
\sigma_P=\ell_P\,t_P=\frac{\hbar G}{c^4},
\]
quantisiert das Produkt aus Raum und Zeit; einzeln können $L$ und $T$ variieren, gemeinsam besitzen sie eine kleinste Zelle. Mit $c=\ell_P/t_P$ folgt für die dimensionslose Feinstruktur
\[
\alpha_\sigma=\frac{\sigma_P}{R\,t},\qquad 
\Lambda_{\mathrm{eff}}=\frac{\alpha_\sigma}{\ell_P^{2}}=\frac{1}{c\,R\,t},
\]
wobei $R$ der Radius und $t$ die Dauer der kausal erreichbaren Domäne sind. Ohne neue Felder und ohne justierbare Parameter jenseits \{\hbar,c,G\} koppelt $\alpha_\sigma$ Mikrophysik und kosmische Skalen und reproduziert den beobachteten Zahlenwert $\alpha_\sigma\approx 4{,}60\times10^{-123}$.

Aus dem Wirkungsintegral
\[
S=\int d^4x\,\sqrt{|g|}\,\big[\mathcal{L}_{\mathrm{matter}}+\mathcal{L}_{\mathrm{int}}(\sigma_P)+\mathcal{L}_{\mathrm{geom}}(\alpha_\sigma)\big]
\]
folgt eine minimale Modifikation der Einsteinschen Feldgleichungen: die Materiequelle $T_{\mu\nu}$ wird mit einem lokal-kovarianten Planck-Kern geglättet. Dadurch bleiben die Bianchi-Identitäten erhalten, Krümmungsinvarianten bleiben endlich, und klassische Singularitäten werden durch geodätisch vollständige Planck-Kerne ersetzt. Auf FLRW-Hintergründen liefert $\Lambda_{\mathrm{eff}}\propto (R\,t)^{-1}$ die späte kosmische Beschleunigung als reines Maßstabsverhältnis.

Auf galaktischen Skalen erzeugt dieselbe Struktur eine universelle Interpolationskinematik, die flache Rotationskurven und die \emph{Radial Acceleration Relation} ohne dunkle Materie erklärt und den Skalensatz $a_0\simeq cH_0/(2\pi)$ fixiert.

\begin{tcolorbox}[title={Warum steckt in \(\ell_P\) ein $c^3$ und in $t_P$ ein $c^5$?}]
Weil nur die Kombination aus \hbar (Quant), $G$ (Gravitation) und $c$ (Kausalität) die \emph{korrekten Dimensionen} liefern kann: 
\[
\ell_P=\sqrt{\hbar G/c^3}\ \text{(Länge)},\qquad
t_P=\sqrt{\hbar G/c^5}\ \text{(Zeit)}.
\]
Das ist reine Dimensionslogik: Für eine Länge braucht es andere Potenzen von $c$ als für eine Zeit. Der Punkt ist \emph{nicht} Zahlenspielerei, sondern: Länge und Zeit sind nur gemeinsam sinnvoll definiert, was sich in
\(
\ell_P/t_P=c
\)
spiegelt. Separat absolut setzen funktioniert nicht, das Produkt $\sigma_P$ ist fundamental.
\end{tcolorbox}

\noindent
\textbf{Konsequenz.} Allgemeine Relativität und Quantenfeldtheorie sind keine Konkurrenzmodelle, sondern zwei Perspektiven desselben Naturgesetzes: der quantisierten Raumzeit.
\end{abstract}

\section{Leitthese}
\begin{tcolorbox}[colback=black!2,colframe=black!50,title={Leitthese}]
Die Natur ist nicht grenzenlos, sondern relational und kausal beschränkt.
Die elementare Grenze ist die Lichtkegelstruktur ($c$); die elementare Körnung ist das Planck-Zweimaß ($\sigP$).
Kosmologisch beobachtete Kleinzahlen sind keine Willkür, sondern Maßstabsverhältnisse zwischen Mikrozelle und endlicher kausaler Domäne.
\end{tcolorbox}

\section{Postulate (Planck–kovariante Fassung)}
\textbf{P1 Unteilbarkeit der Raumzeit.}
Raum und Zeit sind nur gemeinsam operational definiert. 
\[
\frac{\lP}{\tP}=c,\qquad \sigP:=\lP\,\tP=\frac{\hbar G}{c^{4}}.
\]

\noindent
\textbf{P2 $c$ als Existenzbedingung.}
$c$ ist nicht nur maximale Signalgeschwindigkeit, sondern die kausale Bedingung der Realität:
nur Prozesse innerhalb von Lichtkegeln sind physikalisch definiert.

\noindent
\textbf{P3 Minimale Zelle.}
Unterhalb der Breite $\sigP$ verlieren Punktidealisierungen operative Bedeutung.
Komposite Größen sind als kovariante Mittelwerte über $\sigP$ zu bilden; Bianchi bleibt erhalten:
$\nabla^\mu \overline T_{\mu\nu}^{(\sigP)}=0$.

\noindent
\textbf{P4 3+1 ist Projektion.}
Die 3+1-Aufspaltung ist eine makroskopische Näherung. Invarianten sind $\sigP$ und $\lP/\tP=c$,
nicht „Raum“ und „Zeit“ getrennt.

\noindent
\textbf{P5 Kosmische Kopplung.}
Die effektive Vakuumkrümmung skaliert als Maßstabsverhältnis:
\[
\alpha_\sigma:=\frac{\sigP}{R\,t},\qquad
\Lambda_{\rm eff}\sim \frac{\alpha_\sigma}{\lP^2},
\]
sodass $\Lambda_{\rm eff}\to 0$ für $R,t\to\infty$ (Minkowski-Grenze).

\section{Grundidentitäten und Dimensionsbeweis}
Mit $[\hbar]=M L^2 T^{-1}$, $[G]=M^{-1} L^3 T^{-2}$, $[c]=L T^{-1}$:
\[
\boxed{\ \lP=\sqrt{\frac{\hbar G}{c^{3}}}\ ,\qquad
\tP=\sqrt{\frac{\hbar G}{c^{5}}}\ ,\qquad
\sigP=\lP\tP=\frac{\hbar G}{c^{4}}\ }.
\]
Die scheinbar „naheliegende" Zeit $\sqrt{\hbar G/c}$ hat Dimension $L^{2}T^{-1}$ und ist daher \emph{keine} Zeit.
Nützliche Folgeformen:
\[
\lP^{\,2}=c\,\sigP,\qquad \tP^{\,2}=\frac{\sigP}{c}.
\]

\section{Warum so klein? \texorpdfstring{$\alpha_\sigma$}{alpha_sigma} als Maßstabszahl}
\subsection*{Formale Antwort}
\[
\boxed{\ \alpha_\sigma=\frac{\sigP}{R\,t}=\frac{\hbar G}{c^{4}}\;\frac{1}{R\,t}\ },
\]
ist das Verhältnis „Mikro-Zelle zu Makro-Domäne". Für heutiges $(R,t)$ ist $\alpha_\sigma\ll 1$.
Kleinheit ist hier keine Feineinstellung, sondern Geometrie: eine Zelle in einer riesigen, aber endlichen kausalen Fläche $R\,t$.
Mit der Horizont-Zweimaß $\Sigma_H:=\tfrac{c}{H^2}$ ergibt sich äquivalent
\[
\alpha_\sigma=\kappa\,\frac{\sigP}{\Sigma_H},\qquad
\Lambda_{\rm eff}=\frac{\alpha_\sigma}{\lP^2}
=\kappa\,\frac{H^2}{c^2}\quad(\kappa=3\ \text{für FLRW}).
\]

\subsection*{Didaktische Antwort (eine Zeile, klar und ehrlich)}
\begin{tcolorbox}[colback=yellow!5,colframe=yellow!40!black,title={Grenzen sind Teil der Physik, nicht der Frustration}]
Die Zahl ist so klein, weil die Natur uns zeigt, dass Wirklichkeit endlich strukturiert ist:
kausal beschränkt durch $c$ und granular durch $\sigP$.
Wir akzeptieren, dass nichts schneller als $c$ reist; genauso sollten wir akzeptieren,
dass unterhalb der Planck-Zelle keine sinnvolle Auflösung existiert.
\end{tcolorbox}

\section{Kosmische Projektion (eine Zeile Physik, viele Konsequenzen)}
\[
H^2=\frac{8\pi G}{3}\rho-\frac{k c^2}{a^2}+\frac{\Lambda_{\rm eff}(t)}{3},
\qquad
\Lambda_{\rm eff}(t)\approx \frac{\alpha_\sigma(t)}{\lP^2},\quad
\alpha_\sigma(t)=\frac{\sigP}{R(t)\,t}.
\]
Für $R(t)\propto c\,t$ folgt $\Lambda_{\rm eff}\propto t^{-2}$ und der de-Sitter-Fixpunkt $\Lambda\simeq 3H^2/c^2$.

\section{Konsequenzen (kompakt, testbar)}
\begin{enumerate}
  \item \textbf{Regularität:} Punktdivergenzen verschwinden unter kovarianter $\sigP$-Mittelung; Einstein bleibt formal unangetastet.
  \item \textbf{Kosmische Kleinzahlen:} $\Lambda$ ist Maßstabszahl, nicht freier Fit. Minkowski-Grenze ist automatisch $\Lambda\to0$.
  \item \textbf{Vorhersagen:} schwache zeitliche Drift $\Lambda(z)\sim H(z)^2$; galaktische Niedrig-$g$-Regime hängen über Randbedingungen an $H_0$.
\end{enumerate}

\end{document}

\section{Mass–Phase Coupling Constant $K$}
\begin{tcolorbox}[title={Definition}]
We define the mass–phase coupling constant as
\[
K := \frac{\hbar}{c\,G}\,,
\]
with numerical value
\[
K = 5.27\times10^{-33}\;\mathrm{kg^2\,s^2\,m^{-2}}\,.
\]
\end{tcolorbox}

\begin{tcolorbox}[title={Dimensionsanalyse}]
Using $[\hbar]=\mathrm{M\,L^2\,T^{-1}}$, $[c]=\mathrm{L\,T^{-1}}$, $[G]=\mathrm{L^3\,M^{-1}\,T^{-2}}$, we obtain
\[
\bigl[\tfrac{\hbar}{c\,G}\bigr] = \mathrm{M^2\,T^2\,L^{-2}}\,.
\]
Thus $K$ carries the dimensions of "mass squared times time squared per area" – a non‑geometric but physically meaningful combination in a quantum‑gravitational context.
\end{tcolorbox}

\begin{tcolorbox}[title={Physical Interpretation}]
$K$ can be viewed as a **mass–phase coupling scale**: it couples a quantum of action $\hbar$ with the gravitational coupling $G$ and the light‑speed $c$, yielding a quantity that measures how two quantum degrees of freedom (e.g. two spins or entangled trajectories) interact through spacetime curvature. In the language of the present framework it is the product of the interaction quantum $Z_{\text{int}}=\hbar^2/c$ and the inverse geometric stiffness $K_{\text{geom}}=1/(\hbar G)$, i.e.
\[
K = Z_{\text{int}}\,K_{\text{geom}}\,.
\]
\end{tcolorbox}

This constant will appear in the modified field equations and in the dimensionless Lagrangian density as a natural coupling between mass‑quadratic terms and geometric curvature.
