\documentclass{article}
\usepackage[utf8]{inputenc}
\usepackage{amsmath, amssymb}
\usepackage{listings}
\usepackage{xcolor}
\usepackage{hyperref}
\usepackage{geometry}
\geometry{margin=1in}

% Define colors for code listings
\definecolor{codebg}{rgb}{0.97,0.97,0.97}
\definecolor{codegreen}{rgb}{0,0.6,0}
\definecolor{codepurple}{rgb}{0.58,0,0.82}
\definecolor{codered}{rgb}{0.8,0,0}

% Configure listings for Python code
\lstset{
    language=Python,
    backgroundcolor=\color{codebg},
    commentstyle=\color{codegreen},
    keywordstyle=\color{codepurple}\bfseries,
    stringstyle=\color{codered},
    basicstyle=\ttfamily\footnotesize,
    breaklines=true,
    numbers=left,
    numberstyle=\tiny\color{gray},
    captionpos=b,
    frame=single,
    rulecolor=\color{gray},
    showspaces=false,
    showstringspaces=false
}

\title{Universale Principles - Core Formulas}
\author{Adrian Zander}
\date{May 26, 2025}

\begin{document}
\maketitle

\section*{Introduction}
The Universal Resonance Model (URM) provides a framework for understanding synchronization and resonance in complex networks across physical and biological systems. It interprets phenomena from classical oscillators to quantum fields as resonant modes in a vibratory medium. This document presents key equations paired with Python code snippets for simulation and analysis.

\section{Main Equations and Simulations}

\subsection*{0. The Master Field Equation (URME)}
The fundamental evolution of the resonant scalar field $\phi$ is governed by the Universal Resonance Master Equation:
\begin{equation}
\Box \phi + \gamma \partial^\mu \phi + \alpha \phi^3 + \beta \phi^5 = q E_{\text{Res}} + \lambda \langle \hat{Q} \rangle + \eta \epsilon^{\mu\nu\rho\sigma} \partial_\mu \phi \partial_\nu \phi \partial_\rho \phi \partial_\sigma \phi + \xi \zeta(x^\mu) + \kappa \partial_t \phi + \delta \phi E^2
\end{equation}

\subsubsection*{Physical Derivation of Terms}
\begin{itemize}
    \item \textbf{$\Box \phi$}: Relativistic core wave operator (Lorentz invariant background).
    \item \textbf{$\gamma \partial^\mu \phi$}: 4-vector damping, representing energy dissipation across spacetime dimensions.
    \item \textbf{$\alpha \phi^3 + \beta \phi^5$}: Non-linear self-interaction. $\phi^3$ handles localized symmetry breaking, while $\phi^5$ ensures large-scale field stability.
    \item \textbf{$q E_{\text{Res}}$}: Coupling to external resonance fields (environmental synchronization).
    \item \textbf{$\lambda \langle \hat{Q} \rangle$}: Expectation value of quantum observables, providing the bridge between field classicality and quantum operators.
    \item \textbf{$\eta \epsilon^{\mu\nu\rho\sigma} \partial_\mu \phi \partial_\nu \phi \partial_\rho \phi \partial_\sigma \phi$}: Topological invariant term. This protects the field against global deformations and defines the "winding number" of spacetime resonance.
    \item \textbf{$\xi \zeta(x^\mu)$}: Stochastic/perturbation component (background noise/vacuum fluctuations).
    \item \textbf{$\kappa \partial_t \phi$}: Specific temporal damping, defining the arrow of time as a dissipative process.
    \item \textbf{$\delta \phi E^2$}: Electrodynamic feedback, coupling the resonance field to the electromagnetic energy density.
\end{itemize}

\subsection*{0.1. D'Alembertian Identity}
The field density relates to spacetime curvature via the d'Alembertian operator:
\begin{equation}
\Box \phi = -\beta \phi - \beta \partial_t \phi + \beta \nabla^2 \phi + \beta \zeta R \phi
\end{equation}

\subsection*{0.2. Metric Coupling (Einstein-Zander Resonance)}
The geometry of spacetime responds to the field resonance and its entropic state:
\begin{equation}
G_{\mu\nu} + \left[ R_{ij} \cdot \frac{\sigma_P}{\hbar\sqrt{-g}} \right] = \oint_{\partial\Omega} \left[ \frac{\delta\phi}{\delta x^\alpha} \otimes J^\alpha + \kappa \int \zeta(t) dt \right] + \frac{\alpha\phi^3 + \beta\phi^5}{e^{\Delta S/k_B}}
\end{equation}
Where $\sigma_P = \ell_P t_P = \frac{\hbar G}{c^4}$ defines the fundamental spacetime quantum.

\subsection*{1. Coupled Oscillator Dynamics (Kuramoto Model)}
% Describing phase dynamics of coupled oscillators
The Kuramoto model describes the phase dynamics of \(N\) coupled oscillators:
\begin{equation}
\dot{\theta}_i = \omega_i + \sum_{j=1}^N K_{ij}(t) \sin(\theta_j - \theta_i) + \xi_i(t)
\end{equation}
where \(\theta_i\) is the phase, \(\omega_i\) the natural frequency, \(K_{ij}(t)\) the coupling strength, and \(\xi_i(t)\) noise.
\begin{lstlisting}[caption={Kuramoto Model Dynamics}]
import numpy as np
theta_dot = omega + np.sum(K * np.sin(theta - theta[:, None]), axis=1) + xi
\end{lstlisting}

\subsection*{2. Time-Dependent Coupling}
% Modeling dynamic coupling strength
The coupling strength evolves over time:
\begin{equation}
K_{ij}(t) = K_0 \cdot (1 + \sin(\Omega t + \phi_{ij}))
\end{equation}
\begin{lstlisting}[caption={Time-Dependent Coupling}]
K = K0 * (1 + np.sin(Omega * t + phi))
\end{lstlisting}

\subsection*{3. Resonance Frequency}
% Calculating natural resonance frequency
The resonance frequency of an oscillator:
\begin{equation}
\omega_i^{\text{res}} = \frac{1}{2\pi} \sqrt{\frac{k_i}{m_i}}
\end{equation}
\begin{lstlisting}[caption={Resonance Frequency}]
omega_res = (1 / (2 * np.pi)) * np.sqrt(k / m)
\end{lstlisting}

\subsection*{4. Network Coherence}
% Measuring phase synchronization
The order parameter quantifies coherence:
\begin{equation}
R(t) = \left| \frac{1}{N} \sum_{j=1}^N e^{i \theta_j(t)} \right|
\end{equation}
\begin{lstlisting}[caption={Network Coherence}]
R = np.abs(np.mean(np.exp(1j * theta)))
\end{lstlisting}

\subsection*{5. Network Energy}
% Quantifying energy in phase differences
The energy associated with phase differences:
\begin{equation}
E = \frac{1}{2} \sum_{i,j} K_{ij} (1 - \cos(\theta_i - \theta_j))
\end{equation}
\begin{lstlisting}[caption={Network Energy}]
E = 0.5 * np.sum(K * (1 - np.cos(theta[:, None] - theta)))
\end{lstlisting}

\subsection*{6. Nonlinear Coupling}
% Modeling nonlinear interactions
Nonlinear coupling strength:
\begin{equation}
K_{ij} = K_0 \cdot \tanh(\beta \cdot \cos(\theta_i - \theta_j))
\end{equation}
\begin{lstlisting}[caption={Nonlinear Coupling}]
K = K0 * np.tanh(beta * np.cos(theta[:, None] - theta))
\end{lstlisting}

\clearpage

\subsection*{7. 2D Nonlinear Oscillator Lattice}
% Simulating a 2D lattice of nonlinear oscillators
The equation of motion for a 2D lattice oscillator at position \((i,j)\):
\begin{equation}
m \ddot{x}_{i,j} = k (x_{i+1,j} + x_{i-1,j} + x_{i,j+1} + x_{i,j-1} - 4x_{i,j}) - c \dot{x}_{i,j} - \alpha x_{i,j}^3 + F_{i,j}(t)
\end{equation}
where \(F_{i,j}(t) = A \sin(\omega t) \delta_{i,i_0} \delta_{j,j_0}\) is localized forcing.
\begin{lstlisting}[caption={2D Nonlinear Oscillator Lattice},label={lst:2d_lattice}]
import numpy as np
import matplotlib.pyplot as plt

# Parameters
N = 32  # Lattice size
m, k, c, alpha = 1.0, 1.0, 0.1, 0.01  # Mass, spring, damping, nonlinearity
A, omega = 1.0, 2.0  # Forcing amplitude, frequency
dt, T = 0.01, 10  # Time step, total time
i0, j0 = N // 2, N // 2  # Forcing location

# Initialize lattice
x = np.zeros((N, N))
v = np.zeros((N, N))

# Time evolution
for t in np.arange(0, T, dt):
    for i in range(N):
        for j in range(N):
            laplacian = (x[(i + 1) % N, j] + x[(i - 1) % N, j] +
                         x[i, (j + 1) % N] + x[i, (j - 1) % N] - 4 * x[i, j])
            force = k * laplacian - c * v[i, j] - alpha * x[i, j]**3
            if i == i0 and j == j0:
                force += A * np.sin(omega * t)
            v[i, j] += force / m * dt
            x[i, j] += v[i, j] * dt

# Plot heatmap
plt.imshow(np.abs(x), cmap='hot', interpolation='nearest')
plt.colorbar(label='Amplitude')
plt.title('2D Nonlinear Oscillator Lattice')
plt.xlabel('j')
plt.ylabel('i')
plt.savefig('lattice_heatmap.png')
\end{lstlisting}

\subsection*{8. Energy Analysis of 2D Lattice}
% Computing kinetic and potential energy
Kinetic energy of the lattice:
\begin{equation}
E_{\text{kin}}(t) = \frac{1}{2} m \sum_{i,j} v_{i,j}^2(t)
\end{equation}
\begin{lstlisting}[caption={Kinetic Energy}]
def kinetic_energy(v, m):
    return 0.5 * m * np.sum(v**2)
\end{lstlisting}

\clearpage

Potential energy with linear and nonlinear terms:
\begin{equation}
E_{\text{pot}}(t) = \frac{k}{2} \sum_{\langle i,j \rangle} (x_{i,j}(t) - x_{\text{neighbor}}(t))^2 + \frac{\alpha}{4} \sum_{i,j} x_{i,j}^4(t)
\end{equation}
\begin{lstlisting}[caption={Potential Energy}]
def potential_energy(x, k, alpha):
    N = x.shape[0]
    E_pot = 0.0
    for i in range(N):
        for j in range(N):
            neighbors = [x[(i + 1) % N, j], x[(i - 1) % N, j],
                         x[i, (j + 1) % N], x[i, (j - 1) % N]]
            E_pot += np.sum((x[i, j] - np.array(neighbors))**2)
            E_pot += alpha / 4 * x[i, j]**4
    return 0.5 * k * E_pot
\end{lstlisting}

\subsection*{9. Quantum Resonance: Schrödinger Equation}
% Reframing quantum mechanics as resonance
The time-dependent Schrödinger equation as a resonance equation:
\begin{equation}
i \hbar \frac{\partial}{\partial t} \psi(x,t) = \hat{H} \psi(x,t)
\end{equation}
Stationary states from the time-independent equation:
\begin{equation}
\hat{H} \psi_n(x) = E_n \psi_n(x)
\end{equation}
\begin{lstlisting}[caption={Quantum Energy Levels}]
import numpy as np
hbar, omega = 1.0, 2.0  # Reduced Planck constant, frequency
n = np.arange(5)  # Quantum numbers
E_n = hbar * omega * (n + 0.5)  # Energy levels
print("Energy Levels:", E_n)
\end{lstlisting}

\section*{Conclusion}
The URM unifies classical and quantum systems by modeling them as resonant modes in a vibratory medium. From coupled oscillators to quantum fields, it provides a framework for understanding synchronization, energy transfer, and information flow. The provided Python code enables practical simulations, such as the 2D nonlinear oscillator lattice, which can be visualized to study wave patterns and energy dynamics.

\section*{References}
\begin{thebibliography}{9}
\bibitem{kuramoto} Y. Kuramoto, \emph{Self-entrainment of a population of coupled nonlinear oscillators}, Springer, 1975.
\bibitem{strogatz} S. H. Strogatz, \emph{Sync: How Order Emerges from Chaos}, Hyperion, 2003.
\bibitem{pikovsky} A. Pikovsky, M. Rosenblum, J. Kurths, \emph{Synchronization: A Universal Concept}, Cambridge University Press, 2001.
\bibitem{arenas} A. Arenas et al., \emph{Synchronization in complex networks}, Physics Reports, 469(3):93--153, 2008.
\bibitem{Zander} A. Zander, \emph{The Universal Resonance Model}, Zenodo, 2025, \href{https://doi.org/10.5281/zenodo.15476619}{10.5281/zenodo.15476619}.
\end{thebibliography}

\end{document}
