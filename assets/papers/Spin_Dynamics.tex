\documentclass[11pt,a4paper]{article}
\usepackage[utf8]{inputenc}
\usepackage[T1]{fontenc}
\usepackage{amsmath,amssymb}
\usepackage{microtype}
\usepackage{geometry}
\geometry{margin=25mm}

\title{Action vs. Gravity: The Spin-Mass Feedback Loop}
\author{Adrian Zander}
\date{\today}

\begin{document}
\maketitle

\section*{Status / Einordnung}
\noindent\textbf{Document type:} Analogy Note (concept sketch)\\
\textbf{Claim level:} Heuristic analogy and concept sketch, not a validated physical derivation or quantitative black-hole model.\\
\textbf{Use:} Use for intuition-building and future model formalization only.


\section*{Abstract}
This document outlines an alternative dynamic for Black Hole evolution based on the competition between pure interaction ($\hbar$, Action) and mass burden ($G$, Newtonian Gravity). In the Zander 2025 framework, the Black Hole core is viewed as a high-frequency kinematic engine rather than a static mass.

\section{The Pure Action Core}
When a stellar core collapses beyond the neutron-degeneracy limit, it sheds its classical mass-shell and enters a state of \textbf{Pure Action}. The core's energy is stored as pure angular momentum (Spin).
\begin{equation}
    E_{\text{core}} = \hbar \cdot \omega_{\text{max}}
\end{equation}
In this "no-load" state, the core spins at the Planck limit, generating a maximal magnetic field and a specific "Action Potential" that mimics attraction via energy density rather than mass.

\section{The Loading Effect (Braking)}
As the Black Hole environment accumulates matter (Accretion), this external mass $M_{\text{ext}}$ exerts a \textbf{Braking Force} on the core's rotation.
\begin{equation}
    F_{\text{brake}} = \frac{G \cdot M_{\text{ext}}}{r_s^2}
\end{equation}
The interaction between the rotating core and the infalling mass creates a vacuum-frictional resistance. This is the transition where classical Newtonian gravity begins to "win" over the pure kinematic spin.

\section{The Dynamics of the Curve}
\begin{enumerate}
    \item \textbf{Phase 1: Spin-up.} Low mass load. The core is free, spin is maximal. The magnetic field and local spacetime distortion (attraction) increase sharply.
    \item \textbf{Phase 2: Peak.} The point where the braking effect of accumulated mass equals the action potential of the core.
    \item \textbf{Phase 3: Equilibrium/Decline.} Mass burden dominates. The core's energy is transformed into heat, radiation, and eventually new star formation.
\end{enumerate}

\section{Analogy: The Earth's Core}
Consider the Earth's core. If the surrounding mass (mantle and crust) were removed, the core would still rotate, producing a magnetic field. Without the "drag" of the outer mass, its rotational dynamics would be purely determined by its intrinsic action potential. The Black Hole is the ultimate version of this "naked core" physics.

\section{Conclusion}
By treating the Black Hole as a \textbf{Kinematic Transformer}, we move away from the static "abyss" model toward a dynamic system that recycles information and energy into the cosmos.

\end{document}
