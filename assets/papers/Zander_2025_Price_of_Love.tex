\documentclass[11pt,a4paper]{article}
% --- Packages ---
\usepackage[a4paper,margin=2.4cm]{geometry}
\usepackage{newpxtext,newpxmath}
\usepackage[T1]{fontenc}
\usepackage[utf8]{inputenc}
\usepackage[english]{babel}
\usepackage{lmodern}
\usepackage{amsmath,amssymb,mathtools}
\usepackage{microtype}
\usepackage{csquotes}
\usepackage[hidelinks]{hyperref}
\usepackage{setspace}
\setstretch{1.06}
\setlength{\parskip}{0.4\baselineskip}
\setlength{\parindent}{0pt}
\usepackage{siunitx}
\usepackage{graphicx}
\usepackage{booktabs}
\usepackage{eurosym}
\usepackage[most]{tcolorbox}
\tcbset{colback=black!2!white,colframe=black!60!white}
\sisetup{group-separator = {\,}, group-minimum-digits = 4}
\DeclareSIUnit{\EUR}{\euro}

% --- Metadata ---
\title{\textbf{The Statistical Price of True Love:\\
An Abductive Approach to Romance, Entropy and Eurojackpot}}
\author{Adrian Zander \\ \small Department of Unsolicited Equations \\ \small \texttt{zander.adrian@proton.me}}
\date{December 2025}

\begin{document}
\maketitle

\section*{Status / Einordnung}
\noindent\textbf{Document type:} Conceptual Essay / Analogy Note\\
\textbf{Claim level:} Interpretive and analogical text. Scientific statements should be treated as heuristic unless independently derived and tested in a dedicated technical paper.\\
\textbf{Use:} Use for conceptual narrative and analogy exploration.


\begin{abstract}
We present a probabilistic estimate of the “price of true love” based on existential probabilities, demographic densities, astronomical coincidences, and repeated Eurojackpot wins.
A rare human event is translated into a statistically tangible scale and compared with known stochastic benchmarks.
This absurd, yet precise mapping frames an emotion — love — in the coordinate system of raw probability.
\newline
The result: true love is so rare, it makes lottery winners look like commuters.
\end{abstract}

\section{Introduction}
The statistical absurdity of human existence has always been a playground for poets, philosophers, and the occasional drunk physicist.
Every single human encounter is the endpoint of an improbable cascade of chance, causality, and cosmic indifference.
\newline
Yet in our world, such events are swiped away like they were mass-produced. This paper is not a love song. It is a calculation:
What is the actual probability that two specific people exist, meet, and are compatible — simultaneously?
\newline
To make this absurdity more digestible, we translate it into something everyone understands:
\newline
\textbf{money}. Specifically, the number of \emph{Eurojackpot} wins needed to match the same rarity — every single draw, twice a week.

\section{Step 1: Probability of Birth}
According to a (semi-mythical) Harvard estimate, the probability that a specific individual comes into existence is roughly
\[
P_{\rm birth} \approx 1\,\text{in}\,10^{2,685,000}.
\]
Approximating
\[
P_{\rm me} = P_{\rm you} \approx 10^{-2,685,000},
\]
we get
\[
P_{\rm existence} = P_{\rm me}\times P_{\rm you} = 10^{-5,370,000}.
\]

\section{Step 2: Encounter Probability}
Based on British demographic data, the chance of meeting someone compatible in one’s social orbit is
\[
P_{\rm orbit} \approx \frac{1}{285,000} \approx 3.51\times10^{-6}.
\]

\section{Step 3: The Eurojackpot as a Benchmark}
The chance to win the Eurojackpot jackpot (5+2 correct) is
\[
P_{\rm jackpot} = \frac{1}{139{,}838{,}160} \approx 7.15 \times 10^{-9}.
\]

\begin{tcolorbox}[title={\bf Numerical coincidence (intentional)}]
The jackpot probability is a small, clean, dimensionless number ($\sim 7 \times 10^{-9}$).
This is deliberately chosen as a bridge to other ``inverse scales'' that appear in physics when Planck quantities are made dimensionless by normalization.

In other words, a benchmark like $P_{\rm jackpot} \sim 10^{-9}$ mirrors typical inverse Planck-scale factors (e.g. $E/E_P$), turning a quantum-gravity-type suppression into a lottery-like odds statement.
\end{tcolorbox}

Draws occur twice per week (Tuesday and Friday).

We solve for \(N\), the number of consecutive jackpot wins needed to match the probability of true love:
\[
P_{\rm jackpot}^{N} = P_{\rm existence} \cdot P_{\rm orbit}.
\]
Taking \(\log_{10}\):
\begin{align*}
\log_{10}(P_{\rm jackpot}^{N}) &= -5,370,000 + \log_{10}(3.51) - 6 \\
&\approx -5,370,005.45, \\
N &\approx \frac{5,370,005.45}{8.145} \approx 659{,}500 \text{ consecutive wins.}
\end{align*}

At two draws per week, this corresponds to winning \textbf{every single draw for approximately 6,340 years}.

\section{Result}
\noindent
True love corresponds to winning the Eurojackpot \textbf{every Tuesday and Friday for about 6,340 years} — without a single miss.
Mathematically, it’s indistinguishable from divine intervention or a cosmological glitch.

\section{Monetary Translation}
Minimum payout (€10 million):
\[
\text{Total} \approx 6.595 \times 10^{12}\ \text{€}.
\]
Maximum payout (€120 million):
\[
\text{Total} \approx 7.914 \times 10^{13}\ \text{€}.
\]
That’s enough to gold-plate several asteroids — or remind us that no fortune can buy what chance rarely gives.

\begin{tcolorbox}[title={\bf Statement}]
That’s how much true love would cost if it were sold in Euros.
But gold cannot return a gaze.
And no planet can replace the one person who actually finds you.
\end{tcolorbox}



\section{Poetry for the Cold Equations}
\begin{quote}
The numbers are ice — sharp, unforgiving, eternal. \\
They count stars that never aligned, \\
gametes that never met, \\
moments that passed in silence. \\

Yet somewhere in that frozen infinity, \\
a hand found another. \\
Warm. Real. Defiant. \\

We wanted to watch the stars by the sea. \\
The equations said no. \\
But we did anyway.
\end{quote}

\section{Quote to Keep}
\begin{quote}
“The probability of meeting you is astronomical.
The respect for it should be, too.
We wanted to watch the stars by the sea.
Now I can explain every constant, every equation, every cold spark above —
but none of them replace the moment you should’ve been next to me.
Some encounters are so rare,
even the Universe doesn’t bother to calculate them twice.”
\end{quote}
\textit{For readers who don’t get goosebumps when they hear “operator product expansion.”}

\section*{Operators and Dynamical Einstein Equation}
In our framework, the Einstein tensor becomes an \emph{operator}:
\[
\hat{G}_{\mu\nu}[\sigma_P; W] \, |\Psi_W\rangle = \frac{8\pi G}{c^4} \, \overline{T}_{\mu\nu} \, |\Psi_W\rangle.
\]

\textbf{What this means:}
\begin{itemize}
\item The hat ($\hat{}$) indicates an operator, a flexible calculation tool.
\item It allows inputting different quantum states $|\Psi_W\rangle$.
\item The equation adapts dynamically to fluctuations in spacetime energy-momentum.
\item Gravity becomes responsive, not static.
\end{itemize}

\textbf{Layperson analogy:}\\ Classical Einstein’s equation is like a fixed road; operator Einstein’s equation is like a road with dynamic signals and lanes adapting in real time to the flow of cars (energy-momentum).


\begin{tcolorbox}[title={\bf Understanding $\sigma_P$ and Dimensions}]
\textbf{What is $\sigma_P$?}  

\[
\sigma_P = \frac{\hbar G}{c^4} \sim L \cdot T
\]

is the minimal \emph{action cell} of spacetime — the smallest “tick” the universe can count.  

\textbf{What are dimensions?}  

Dimensions (like $M$ for mass, $L$ for length, $T$ for time) tell us the kind of quantity we are dealing with:
\newline  
- $M$ means “something has mass.”
\newline
- $L$ means “it has length or spatial extent.”
\newline  
- $T$ means “it evolves over time.”  

\textbf{Dimensions analysis} is simply checking that equations make sense in terms of these units — like making sure meters are added to meters, not seconds. It also reveals hidden structure: how quantities are fundamentally related.  

\textbf{Why $\sigma_P$ matters dimensionally:}  

- $\sigma_P \sim L \cdot T$ tells us the minimal combination of space and time that a quantum of action occupies.
\newline  
- Multiply by $c^2$ to get $\sigma_P c^2 \sim L^3 \cdot T$, the effective volume per tick. 
\newline 
- Two masses in this cell give $M^2 \cdot \sigma_P c^2 \sim M^2 L^3 T$, which combined with relative scales reproduces $[\hbar^2] = M^2 L^4 T^{-2}$.  

\textbf{Kids Talk:}  

Imagine the universe has a “grid” of tiny spacetime cells. Each cell counts how two masses interact over space and time. Dimensions tell you what kind of quantities you’re stacking together (mass, length, time) so the universe can “count” consistently.  
\newline
In short: $\sigma_P$ is the building block, dimensions are the labels, and dimensional analysis is the consistency check that ensures all the universe’s calculations make sense — from Planck scale to the farthest galaxies.
\end{tcolorbox}

\vspace{0.5cm}

\begin{tcolorbox}[title={\bf Why $\hbar^2$ has this dimension}]
The dimension of $\hbar^2$ is
\[
[\hbar^2] = M^2 L^4 T^{-2}.
\]

\textbf{Why this makes sense:}  

1. Two masses ($M^2$) interact in a minimal spacetime cell.
\newline  
2. Length factors ($L^4$): One $L^3$ for the spatial volume, one $L$ for the relative distance between the masses.
\newline  
3. Time factor ($T^{-2}$): Time appears twice. First, for the \emph{simultaneity} of both masses (their clocks run in sync). Second, for the \emph{relative timing} of their interaction across space.  

\textbf{Scaling:}  
This structure is universal — from the Planck scale all the way to the largest observable galaxies. It encodes how two masses are coupled through space and time, in the smallest action unit $\sigma_P$.  

\textbf{Intuition:}  
Think of $\hbar^2$ as a “counting rule” for the universe: two masses in a 3D volume, interacting over time. Each tick measures their coupled effect. The $T^{-2}$ tells us that time flows for both, and also defines how their interaction unfolds across distance.  

\textbf{Bridge to Physics:}  
This is nothing less than Einstein’s general relativity in miniature — the interplay of mass, spacetime, and curvature — now expressed in a single quantum of action. It forms the bridge between quantum mechanics, geometry, and gravity.
\end{tcolorbox}



\section*{From Fundamental Action Cells to the Quantum-Covariant Einstein Equation}

\paragraph{Minimal spacetime cell:}  
The universe counts in discrete units of action, given by
\[
\sigma_P = \frac{\hbar G}{c^4} \sim L \cdot T,
\]
the smallest \emph{spacetime tick}. This defines the minimal volume in which physical interactions occur. Multiplying by \(c^2\) gives an effective 3D volume per tick:
\[
\sigma_P c^2 \sim L^3 \cdot T.
\]

\paragraph{Dimensions and dimensional analysis:}  
Every physical quantity carries dimensions: \(M\) for mass, \(L\) for length, \(T\) for time. Dimensional analysis ensures equations are consistent — for instance, you cannot add meters to seconds. More importantly, it reveals hidden structure: how masses, space, and time combine fundamentally.  

\paragraph{Why $\hbar^2$ has the right dimension:}  
\[
[\hbar^2] = M^2 L^4 T^{-2}.
\]
- \(M^2\) represents two masses interacting in the same minimal spacetime cell.  
- \(L^4\) comes from the 3D spatial volume plus a relative length factor.  
- \(T^{-2}\) arises twice: first for the simultaneity of both masses, second for the relative timing of their interaction.  
\newline
Intuitively, \(\hbar^2\) acts as a universal counting rule: two masses in a 3D volume interacting over time. This applies from Planck scale to the largest cosmic structures.

\paragraph{Planck-covariant averaging and operator Einstein equation:}  
To include quantum effects in general relativity, we average the energy-momentum tensor over the minimal cell \(\sigma_P\):
\[
\overline{T}_{\mu\nu} = \frac{1}{\sigma_P} \int_{\sigma_P} T_{\mu\nu} \, d\sigma.
\]

The quantum-covariant Einstein equation reads:
\[
\hat{G}_{\mu\nu}[\sigma_P; W] \, |\Psi_W\rangle = \frac{8\pi G}{c^4} \, \overline{T}_{\mu\nu} \, |\Psi_W\rangle,
\]
where \(|\Psi_W\rangle\) is the vacuum state of the entire universe.  

\paragraph{Effective cosmological constant:}  
The observed accelerated expansion is captured by a dynamical, scale-dependent \(\Lambda\):
\[
\Lambda_{\rm eff}(t) = \frac{1}{c R t}, \quad R = \text{Hubble radius}, \ t = \text{cosmological time}.
\]

\paragraph{Bridge to physics:}  
This framework unites quantum mechanics, geometry, and gravity in a single formalism. Minimal action cells \(\sigma_P\) encode the discrete structure, \(\hbar^2\) governs mass interactions, and the Planck-covariant operator Einstein equation ensures energy conservation and consistency across scales. In essence, it is Einstein’s general relativity expressed through the lens of quantum discretization.



\begin{enumerate}
\item \textbf{Cosmological Constant?} \\
    Why is the universe expanding faster instead of just sitting there? Because area $\times$ time ($cRt$) acts like a natural clock. No magic. No extra energy. Just entropy flow.
    \[ \Lambda_{\rm eff}(t) = \frac{1}{c R t} \]

    \item \textbf{Vacuum Energy?} \\
    The ``empty'' vacuum supposedly contains so much energy that we should have exploded long ago. If you match quantum physics properly with the universe, nature itself sets the pressure limit.
    \[ \rho_{\rm vac} = \frac{\hbar c}{L^4} \implies \Lambda_{\rm obs} \approx \frac{1}{\sigma_P \cdot N_{\sigma}} \]

    \item \textbf{Why Are Electrons Light and Top Quarks Heavy?} \\
    Because nature quietly uses a number $q \approx 0.22$ (Cabibbo says hi) when assembling particles. More $q$-exponents $\rightarrow$ lighter mass.
    \[ m_n = M_{\rm Planck} \cdot q^n \]

    \item \textbf{Neutrinos and Their Wild Mixing?} \\
    Instead of staying in their lanes, neutrinos dance between flavors. The choreography? Again $q$-exponents.
    \[ U_{\rm PMNS} \sim f(q, \theta_{13}) \]

    \item \textbf{What Happens When You Measure Something Quantum?} \\
    People once thought the universe collapses when you look at it. Now we know: everything stays unitary. Your measuring device just acts like a sieve, filtering reality through double uncertainty.
    \[ \Delta x \Delta p \ge \frac{\hbar}{2} \quad \xrightarrow{\sigma_P} \quad \text{Unitary Regularization} \]

    \item \textbf{Quantum Gravity?} \\
    No new gods needed. Gravity can be quantized if you admit all fields are slightly ``smeared'' across a fundamental spacetime area $\sigma_P$.
    \[ G_{\mu\nu} + \Lambda_{\rm eff} g_{\mu\nu} = \frac{8\pi G}{c^4} \langle T_{\mu\nu} \rangle_{\sigma} \]

    \item \textbf{Holography?} \\
    All the information in the universe is stored on the surface, not in the volume. Like a USB stick with $10^{123}$ bits.
    \[ S_{\rm max} = \frac{k_B A}{4 \ell_P^2} = \frac{k_B \pi R^2}{\sigma_P c} \]

    \item \textbf{Why Does Electric Charge Matter for Entropy?} \\
    Because a proton carries not only gravitational information but also charge. That boosts its contribution to cosmic order by factor 137.
    \[ S_{\rm total} = S_{\rm grav} + \alpha^{-1} S_{\rm em} \]

    \item \textbf{Black Hole Information?} \\
    Old story: ``It’s lost!'' New story: ``It was never lost — only holographically mirrored.'' Everything remains.
    \[ S_{\rm vN}(t) \le S_{\rm thermo}(M(t)) \]

    \item \textbf{Why Is the Universe Expanding Faster Than Expected?} \\
    The expansion isn’t wrong — our assumptions about large-scale effects were just naive. You just need to clean your lens.
    \[ H_0^{\rm local} = H_0^{\rm global} \cdot \sqrt{1 + \delta_{\sigma}} \]

    \item \textbf{Singularities?} \\
    Nope. It doesn’t collapse at $t = 0$ like a bad poem. It’s finite. Boring. And beautifully regular.
    \[ \rho_{\rm max} = \frac{c^2}{G \sigma_P} \]

    \item \textbf{The 3.7x Dipole (The Cosmic Speed Limit)?} \\
    The universe shows a directionality that shouldn't be there. Why? Because we move through a $\sigma_P$-grid, not through a smooth void. The observed excess is the "friction" of the discrete spacetime.
    \[ \vec{v}_{\rm dipole} \propto \frac{c^4}{G} \cdot \sigma_P \cdot \nabla \eta \]

    \item \textbf{Why Is Gravity Spin-2?} \
Because nature couples action quadratically — not linearly like light, but self-referencing. One mass curves the cell, the other feels the curve. \
$ \chi(M) = \frac{G M^2}{\hbar c^3} = \frac{M^2}{\sigma_P c^3} $
\item \textbf{What Is the Fundamental Spacetime Grain?} \
Not length. Not time. An action cell with dimension length × time. The minimal update the universe allows. \
$ \sigma_P = \frac{\hbar G}{c^4} $
\item \textbf{Entropy as Pure Counting?} \
Forget disorder. Entropy counts ticks — each quantum of action registered in the cell. Black holes don't lose information; they just finish counting. \
$ S = k_B \cdot N_{\rm ticks}, \quad N_{\rm ticks} = \sum \frac{\Delta A}{\sigma_P} $
\item \textbf{Why Black Holes Don't Evaporate to Nothing?} \
Because radiation isn't lost — quantum gravity pulls it back. Particles attract, merge, accumulate around the remnant. Enough dust, and a new star ignites. The end is just the seed for the next beginning. \
$ M_{\rm remnant} \sim \sqrt{\frac{\hbar c}{G}}, \quad \rho_{\rm dust} \to \rho_{\rm crit} = \frac{c^2}{G \sigma_P} $
\item \textbf{Why Does the Fine-Structure Constant Appear in Entropy?} \\
Charged particles carry extra bits beyond mass. Gravity counts gravitational charge alone; electromagnetism adds information through the vacuum's impedance, where $c = 1/\sqrt{\varepsilon_0 \mu_0}$ sets the propagation limit for all electromagnetic waves, including photons. Through $\alpha = e^2/(4\pi\varepsilon_0\hbar c)$ this extra structure enters as an inverse weight. \\
\[
S_{\rm total} \approx S_{\rm grav}\left(1 + \alpha^{-1}\right) 
= S_{\rm grav}\left(1 + \frac{1}{137}\right).
\]

\item \textbf{Why is the speed of light hidden in entropy?} \\
Because $c$ is not just a velocity, but the slope that couples space and time to energy and information. Through $c^2 = 1/(\varepsilon_0 \mu_0)$, the vacuum ties its electric inertia to the maximal rate of information propagation; through $[\hbar^2] = M^2 L^4 T^{-2}$, the Planck constant reads as a bilinear coupling of two masses inside a single action cell $\sigma_P$. \\
\[
c^2 = \frac{1}{\varepsilon_0 \mu_0}, \qquad
n^2 = \varepsilon_r \mu_r, \qquad
[\hbar^2] = M^2 L^4 T^{-2}.
\]
In $\sigma_P = \frac{\hbar G}{c^4}$, both worlds meet: geometry ($c$) and quanta ($\hbar$) jointly define the minimal entropy tick size with which the universe counts.

\item \textbf{Why Do We Have a Cosmic Horizon?} \
The observable universe is bounded by the total number of ticks since start. Beyond: not hidden, just uncounted yet. \
$ R(t) \sim c \cdot t \cdot \sqrt{N_{\sigma}} $

\item \textbf{Why Does $\hbar^2$ Look Like Two Masses in the Same Spacetime Cell?} \\
The dimension $[\hbar^2] = M^2 L^4 T^{-2}$ can be read as: two sources $M^2$, coupled over a $3+1$-dimensional action volume $\sigma_P c^2$ and a relative coupling scale $\sigma_{\rm rel}$. With $\sigma_P = \hbar G / c^4$ having dimension $L T$, multiplying by $c^2$ yields an effective space-per-tick volume. Two Planck masses in the same $\sigma_P$-block then realize quantum gravity: gravitational self-coupling of matter within a minimal spacetime grain. \\
\[
[\hbar^2] = (M^2)\,(\sigma_P c^2)\,(\sigma_{\rm rel}),
\qquad
[\sigma_{\rm rel}] = L T^{-1}.
\]

\end{enumerate}

\begin{thebibliography}{99}


\bibitem{Eurojackpot}
Eurojackpot – Official winning probabilities. 
\url{https://www.eurojackpot.de/}.

\bibitem{Harvard}
Al Binazir, Popular estimate of individual birth probability, 
\emph{Harvard Magazine}, 1997.

\bibitem{Modified-Drake}
Modified Drake-Equation Encounter Probability, see statistical models of human encounters.

\bibitem{Quantum-Gravity}
Zander, A., \emph{A Parameter-Free Unification of Quantum Mechanics and General Relativity}, 2025.  
Formal field equation:  
\[
\hat{G}_{\mu\nu}[\sigma_P; W] \, |\Psi_W\rangle = \frac{8\pi G}{c^4} \, \overline{T}_{\mu\nu} \, |\Psi_W\rangle
\]  
with Planck-covariant averaging \(\overline{T}_{\mu\nu}\) over minimal spacetime cells \(\sigma_P\).

\bibitem{Lambda}
Derivation of the effective cosmological constant:  
\[
\Lambda_{\rm eff}(t) = \frac{1}{c R t}, \quad 
\overline{T}_{\mu\nu} = \langle T_{\mu\nu} \rangle_{\sigma}, \quad
\hat{G}_{\mu\nu} + \Lambda_{\rm eff} g_{\mu\nu} = \frac{8\pi G}{c^4} \overline{T}_{\mu\nu}.
\]  
This form ensures energy conservation and Planck-covariant source averaging are consistent.


\end{thebibliography}

\begin{flushright}
\textit{
  Quantum Gravity on New Years Eve\\
We are tiny dots, of white light.\\
We are reflections, on a dark night.\\
Bright lives briefly, sparkle in time.\\
Watchers of the sky, a fire in our eyes.\\
We are jewels to blaze the Dark.\\
Treasure flying high, winking like an eye.\\
Fierce and burning, until we die.\\
A glimpse of beauty on the canvas of the Sky.\\
We are all Stars.}
\end{flushright}

\end{document}
