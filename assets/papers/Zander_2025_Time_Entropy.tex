\documentclass[12pt,a4paper,twoside]{article}

\usepackage[ngerman]{babel}
\usepackage[utf8]{inputenc}
\usepackage[T1]{fontenc}
\usepackage{amsmath,amssymb,amsfonts}
\usepackage{mathrsfs}
\usepackage{geometry}
\usepackage{fancyhdr}
\usepackage{titlesec}
\usepackage{abstract}

% Geometrie und Layout
\geometry{top=25mm, bottom=25mm, left=25mm, right=25mm}
\setlength{\parindent}{0em}
\setlength{\parskip}{0.5em}

% Header
\pagestyle{fancy}
\fancyhf{}
\fancyhead[LE,RO]{\thepage}
\fancyhead[RE]{Zander et al.}
\fancyhead[LO]{Zeit und Entropie als abgeleitete Größen}

% Titel-Formatierung
\title{\textbf{Zeit, Entropie und Irreversibilität als abgeleitete Größen quantisierter Raumzeit}\\
\large Eine wirkungsbasierte Formulierung ohne vorausgesetzten Zeitparameter}

\author{Adrian Zander}
\date{\today}

\begin{document}

\maketitle

\begin{abstract}
\noindent
In der etablierten Thermodynamik und der Allgemeinen Relativitätstheorie werden Zeit und Entropie häufig als a priori existente Strukturen oder statistische Emergenzen behandelt. Diese Arbeit postuliert stattdessen einen fundamentaleren Ansatz: Die Quantisierung der Raumzeit durch eine minimale Wirkungszelle $\sigma_P = \hbar G / c^4$. Wir zeigen, dass Zeitordnung und Entropie keine intrinsischen Eigenschaften des Universums sind, sondern aus der Zählung diskreter Wirkungsbelegungen (Ticks) hervorgehen. Die thermodynamische Temperatur wird hierbei als Irreversibilitätsrate $\mathcal{R}$ neu definiert, was eine skaleninvariante Beschreibung von makroskopischer Materie bis hin zu Schwarzen Löchern ermöglicht und das Informationsparadoxon durch eine unitäre Buchhaltung auflöst.
\end{abstract}

\hrule
\vspace{1em}

\section{Einführung: Das Problem des externen Zeitparameters}

Die konventionelle Physik beschreibt Dynamik als Funktion der Zeit $f(t)$. Dies setzt die Existenz einer kontinuierlichen, externen Zeitvariable voraus. In Regionen extremer Gravitation oder an der Planck-Skala verliert dieser Zeitbegriff jedoch seine operative Bedeutung.

Der hier vorgestellte Ansatz (Zander-Framework 2025) invertiert die Logik: Nicht die Zeit ermöglicht Veränderung, sondern diskrete Veränderung (Wirkung) erzeugt Zeit. Physikalisch real ist ausschließlich die Wirkung $\Delta A$. Zeit und Entropie sind Buchhaltungsgrößen dieser Wirkung.

\section{Axiomatik der Wirkungszellen}

\subsection{Die fundamentale Zelle $\sigma_P$}
Wir definieren die Raumzeit nicht als kontinuierliches Mannigfaltigkeit, sondern als ein Gitter minimaler Wirkungsaufnahmekapazität. Die Größe der Elementarzelle bestimmt sich aus den fundamentalen Naturkonstanten:
\begin{equation}
    \sigma_P = \frac{\hbar G}{c^4} \quad [\text{m} \cdot \text{s}].
\end{equation}
Diese Größe repräsentiert die minimale Kopplung von Quantenmechanik ($\hbar$) und Gravitation ($G$) in der Raumzeitstruktur.

\subsection{Der Tick-Index $i$}
Jeder physikalische Prozess mit Energie $E$ und Dauer $\Delta t$ belegt eine endliche Anzahl dieser Zellen. Wir definieren den dimensionslosen Tick-Index $i$:
\begin{equation}
    i = \frac{\Delta A}{\sigma_P} = \frac{E \cdot \Delta t}{\sigma_P}.
\end{equation}
Dabei ist $i$ streng ganzzahlig zu verstehen ($\Delta A \ge \sigma_P$), im makroskopischen Limes wird es als kontinuierlich genähert.

\section{Entropie und Zeit als Ableitungen}

\subsection{Entropie als Zählgröße}
Entropie $S$ wird in diesem Formalismus von ihrer statistischen Interpretation (Maß der Unordnung) befreit. Sie ist definiert als das kumulative Inventar der belegten Wirkungszellen:
\begin{equation}
    S \equiv k_B \cdot N_{\text{ticks}} = k_B \sum_{k} i_k.
\end{equation}
Ein System, das keine Wirkung umsetzt ($i=0$), besitzt keine Entropiedynamik. Informationsverlust ist ausgeschlossen, da jeder Tick $i$ eine physikalische Spur in der Matrix hinterlässt (Unitarität).

\subsection{Zeit als Ordnungsrelation}
Da $t$ nicht fundamental vorausgesetzt wird, ergibt es sich als Sekundärgröße aus der Wirkungsdichte. Die messbare Dauer $\Delta t$ ist das Verhältnis von akkumulierter Wirkung zu Systemenergie:
\begin{equation}
    \Delta t = \sigma_P \frac{i}{E}.
\end{equation}
Zeit ist somit nichts anderes als die sequentielle Ordnung irreversibler Ticks. Ein "Anhalten der Zeit" entspricht dem Zustand $\dot{S} = 0$.

\section{Irreversibilität statt Temperatur}

Die klassische Definition der Temperatur $T^{-1} = \partial S / \partial E$ suggeriert einen thermischen Gleichgewichtszustand. Wir ersetzen diesen Begriff durch die \textbf{Irreversibilitätsrate} $\mathcal{R}$, die den gerichteten Wirkungsfluss beschreibt.

Aus der Definition von $S$ und $i$ folgt:
\begin{equation}
    \frac{dS}{dE} = k_B \frac{\partial}{\partial E} \left( \frac{E \Delta t}{\sigma_P} \right) \approx \frac{k_B \Delta t}{\sigma_P}.
\end{equation}
Daraus definieren wir $\mathcal{R}$ als die Energieabgabe pro Entropieeinheit:
\begin{equation}
    \mathcal{R} \equiv \frac{dE}{dS} = \frac{\sigma_P}{k_B \Delta t}.
\end{equation}
\textbf{Interpretation:}
\begin{itemize}
    \item Für große $\Delta t$ (makroskopische, träge Systeme) geht $\mathcal{R} \to 0$. Das System verhält sich reversibel.
    \item Für $\Delta t \to t_P$ (Planck-Zeit) erreicht $\mathcal{R}$ das Maximum. Das System strahlt maximal irreversibel.
\end{itemize}

\section{Anwendung: Skaleninvarianz und Schwarze Löcher}

\subsection{Der makroskopische Grenzfall (Low-Load)}
Für gewöhnliche Materie der Masse $m$ ist die gravitative Kopplung $\chi(m) = G m^2 / (\hbar c^3)$ vernachlässigbar. Der Tick-Index bleibt weit unter dem Sättigungslimit:
\begin{equation}
    i \ll i_{\text{max}} = \frac{c^4}{G}.
\end{equation}
Die Raumzeit-Metrik bleibt klassisch, die Irreversibilitätsrate $\mathcal{R}$ ist durch die molekulare Zeitskala bestimmt und entspricht der thermischen Umgebungstemperatur.

\subsection{Das Schwarze Loch (Saturation)}
Ein Schwarzes Loch repräsentiert den Zustand, in dem die Wirkungsdichte die strukturelle Kapazität der Raumzeit erreicht ($i \to i_{\text{max}}$).
Die Bekenstein-Hawking-Entropie ergibt sich direkt aus der Zählung der Oberflächenzellen:
\begin{equation}
    N_{\text{ticks}} \approx 4\pi \frac{r_s^2}{\ell_P^2} = 4\pi \chi(M).
\end{equation}
Die Hawking-Strahlung ist in diesem Bild keine thermische Fluktuation, sondern die geometrisch erzwungene Irreversibilität $\mathcal{R}_{\text{BH}}$. Da $\Delta t \sim r_s/c$, folgt:
\begin{equation}
    \mathcal{R}_{\text{BH}} \propto \frac{\sigma_P}{k_B (GM/c^3)} \propto \frac{1}{M}.
\end{equation}
Dies reproduziert exakt das Verhalten $T_H \propto 1/M$, jedoch ohne Rückgriff auf statistische Ensembles. Das Schwarze Loch ist ein Kinematik-Transformator, der Masse durch hohe Tick-Dichte in Strahlung konvertiert.

\section{Fazit}

Wir haben gezeigt, dass sich Zeit, Entropie und Temperatur konsistent aus einer einzigen Annahme ableiten lassen: der Quantisierung der Wirkung in Einheiten von $\sigma_P$.

Das resultierende Weltbild verzichtet auf externe Parameter:
\begin{equation}
    \boxed{ \Delta A \xrightarrow{\sigma_P} i \xrightarrow{k_B} S \xrightarrow{\mathcal{R}} t }
\end{equation}
Es gibt keine Zeit ohne Wirkung. Es gibt keine Entropie ohne Prozess. Das Universum ist die Summe seiner Ticks.

\end{document}
