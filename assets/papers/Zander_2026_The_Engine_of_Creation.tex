\documentclass[11pt,a4paper]{article}
% --- Packages ---
\usepackage[a4paper,margin=2.4cm]{geometry}
\usepackage{newpxtext,newpxmath}
\usepackage[T1]{fontenc}
\usepackage[utf8]{inputenc}
\usepackage[english]{babel}
\usepackage{amsmath,amssymb,mathtools}
\usepackage{microtype}
\usepackage[hidelinks]{hyperref}
\usepackage{setspace}
\setstretch{1.08}
\setlength{\parskip}{0.5\baselineskip}
\setlength{\parindent}{0pt}
\usepackage{siunitx}
\usepackage[most]{tcolorbox}

% --- Metadata ---
\title{\textbf{The Engine of Creation:\\
Planck Remnants as Kinematic Transformers}}
\author{Adrian Zander \\ \small Department of Unified Resonance \\ \small \texttt{zander.adrian@proton.me}}
\date{January 2026}

\begin{document}
\maketitle

\begin{abstract}
In the standard semiclassical treatment, black hole evaporation concludes with either total evaporation (Hawking) or a static Planck-scale remnant. We propose an alternative within the $\sigma_P$-framework: the black hole core as a kinematic engine. We demonstrate that for systems with substantial accumulated external mass ($M_{ext}$), the remnant serves as a transformer rather than a sink. Through a coupled mass-action feedback mechanism, the core can be re-energized by orbital braking effects, effectively recycling quantum information into macroscopic energy. This model provides a thermodynamic basis for black-hole seeded star formation and resolves the information paradox via unitary feedback.
\end{abstract}

\section{Introduction}
The transition from a collapsing singularity to a stable quantum remnant is a hallmark of discrete spacetime theories. However, a static remnant poses a new teleological problem: what is the fate of the information and action stored within? 

In this paper, we extend the Zander 2025 Framework to include dynamic re-energization. We treat the Schwarzschild or Kerr core not as an isolated mathematical ideal, but as a \textit{Kinematic Motor} coupled to its environment. Similar to how a planetary core is influenced by tidal forces and mantle dynamics, a black hole remnant is driven by the interaction between its internal action potential ($\hbar$) and the surrounding inertial mass ($M$).

\section{The Mathematical Engine}
The energy state of a black hole system in the quantized frame is defined as the sum of pure quantum action and the potential of the surrounding mass reservoir:

\begin{equation}
E_{sys} = \underbrace{\hbar \cdot \omega_{\max}}_{\text{Core Action}} + \underbrace{\eta \cdot M_{ext} c^2}_{\text{Accretion Potential}}
\end{equation}

Where $\omega_{\max}$ is the fundamental Planck frequency, and $\eta$ is the efficiency factor of the kinematic coupling.

\subsection{Differential of Re-energization}
Unlike the purely monotonic decay of the Hawking model, the rotational frequency $\omega$ follows a balanced differential equation:

\begin{equation}
\frac{d\omega}{dt} = \Gamma_{acc}(M_{ext}) - \Gamma_{brake}(M_{core}, \omega)
\end{equation}

Here, $\Gamma_{acc}$ represents the energy transfer from accreted mass back to the core spin, while $\Gamma_{brake}$ is the \textit{Inertial Brake} caused by the spacetime curvature of the accumulated mass:

\begin{equation}
\Gamma_{brake} = \frac{G}{c^3 \hbar} \cdot (M_{core} + M_{ext})^2 \cdot \omega
\end{equation}

\section{Information Return and Feedback}
A crucial result of this model is the prediction of \textit{Spin Luminosity} ($L_{spin}$). As the core's spin is braked by incoming mass, the "uncounted" action units ($\sigma_P$) must be returned to the environment to satisfy the Unitary Restoration principle (Pillar 9).

\begin{equation}
L_{spin} = \hbar \cdot \left| \frac{d\omega}{dt} \right|_{brake}
\end{equation}

This luminosity is not random thermal noise, but structured information flow that carries the "echoes" of the interior. Over cosmic timescales, this energy acts as a catalyst, providing the pressure necessary to seed new stellar nurseries in the vicinity of the remnant.

\section{Conclusion}
The Zander-Index approach reveals that black holes are not graves for information, but the most efficient engines in the universe. They transform the collapse of matter into the rotation of spacetime itself, only to release that energy back when the system reaches its "Inertial Limit." The end of a black hole is, therefore, the primary seed for the next cycle of cosmic creation.

\begin{center}
\textit{"The universe does not lose information; it only re-energizes its memory."}
\end{center}

\end{document}
