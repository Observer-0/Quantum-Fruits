\documentclass[11pt,a4paper]{article}
\usepackage[utf8]{inputenc}
\usepackage[T1]{fontenc}
\usepackage{amsmath,amssymb,physics}
\usepackage{microtype}
\usepackage{geometry}
\usepackage{listings}
\usepackage{xcolor}
\geometry{margin=25mm}
\usepackage{tcolorbox}
\tcbuselibrary{skins,breakable}

\definecolor{codegreen}{rgb}{0,0.6,0}
\definecolor{codegray}{rgb}{0.5,0.5,0.5}
\definecolor{codepurple}{rgb}{0.58,0,0.82}
\definecolor{backcolour}{rgb}{0.95,0.95,0.92}

\lstdefinestyle{mystyle}{
    backgroundcolor=\color{backcolour},   
    commentstyle=\color{codegreen},
    keywordstyle=\color{magenta},
    numberstyle=\tiny\color{codegray},
    stringstyle=\color{codepurple},
    basicstyle=\ttfamily\footnotesize,
    breakatwhitespace=false,         
    breaklines=true,                 
    captionpos=b,                    
    keepspaces=true,                 
    numbers=left,                    
    numbersep=5pt,                  
    showspaces=false,                
    showstringspaces=false,
    showtabs=false,                  
    tabsize=2
}

\lstset{style=mystyle}

\title{SigmaP-Lab: Formelsammlung \& Struktur}
\author{Adrian Zander}
\date{\today}

\begin{document}
\maketitle

\section*{Status / Einordnung}
\noindent\textbf{Document type:} Formula Collection (working reference)\\
\textbf{Claim level:} Working notation and formula registry. Statements are concise summaries and may be revised as derivations are tightened.\\
\textbf{Use:} Use as a quick reference, not as a standalone demonstration.


\section*{Abstract}
Diese Sammlung definiert die Kerngleichungen des \textit{SigmaP-Lab}-Frameworks (Version 10.0). Sie umfasst die fundamentale Definition der Wirkungszelle $\sigma_P$, die Ableitung der Gravitation als Spin-2-Resonanzphänomen und die Neuinterpretation der Zeit als diskrete Tick-Sequenz.

\section{Fundamentale Identitäten}

\subsection{Die Wirkungszelle}
\begin{equation}
\boxed{\sigma_{\mathrm P} = \frac{\hbar G}{c^{4}}} \quad [L \cdot T]
\end{equation}
Die fundamentale Quantisierungseinheit der Raumzeit.

\subsection{Kopplungskonstanten}
\begin{align}
\alpha_{G}(M) &= \frac{G M}{\hbar c} \qquad \text{(Lineare Massenkoplung)} \\
\chi(M) &:= \frac{G M^2}{\hbar c^3} \qquad \text{(Quadratische Spin-2-Selbstkopplung)}
\end{align}

\subsection{Tick-Statistik}
\begin{align}
i &\equiv \frac{\Delta A}{\sigma_{\mathrm P}} \qquad \text{(Tick-Index)} \\
i_{\rm max} &= \frac{\hbar}{\sigma_P} = \frac{c^4}{G} \\
S &= k_B \cdot N_{\text{ticks}} = k_B \sum i
\end{align}

\subsection{Energie-Zeit & Hawking-Relation}
\begin{align}
E_{\text{quant}} \cdot t_{\text{quant}} &= \hbar \qquad \text{(Ein Quant = Ein Tick)} \\
E_{\rm kin} &= \sigma_P \cdot \frac{i_{\rm kin}}{t_P} \\
t_{\text{Kerr}} &:= t \cdot \sqrt{\chi(M)} \qquad \text{(Zeitdilatation als Tick-Dichte)}
\end{align}

\subsection{Spin-2 Signatur}
\begin{equation}
A_G = \frac{G^2}{c^4}, \qquad Z = \frac{\hbar^2}{c} \quad \Rightarrow \quad Z \cdot A_G = \sigma_P
\end{equation}
Gravitation ist die Resonanz zwischen Wirkungsquantisierung $Z$ und Geometrie $A_G$.

\section{Feldgleichungen und Pfadintegral}

\subsection{Die Einstein-Zander Wirkung}
\begin{equation}
S[g,\Psi] = \frac{c^3}{16\pi G}\int d^4x\sqrt{-g}\Big(R - 2\Lambda_{\rm eff}(t)\Big) + S_{\rm m}^{(\sigma)}[g,\Psi]
\end{equation}

\subsection{Vakuum-Skalierung}
\begin{equation}
\Lambda_{\rm eff}(t) = \frac{\alpha_\sigma(t)}{\ell_P^2} = \frac{1}{c R(t) t}
\end{equation}
Keine dunkle Energie, sondern geometrische Glättung über den kausalen Horizont.

\subsection{Feldgleichung}
\begin{equation}
G_{\mu\nu}[g] + \Lambda_{\rm eff}(t)g_{\mu\nu} = \frac{8\pi G}{c^4}\overline T_{\mu\nu}
\end{equation}
mit der Planck-kovarianten Quellmittelung:
\begin{equation}
\overline{T}_{\mu\nu}(x) = \int d^4y\sqrt{-g(y)} K_{\sigma_P}(x,y)\Pi_{\mu}{}^{\mu'}\Pi_{\nu}{}^{\nu'}T_{\mu'\nu'}(y)
\end{equation}

\section{Struktur-Übersicht: Von Newton bis Quantengravitation}

\begin{enumerate}
    \item \textbf{Klassisch (Newton):} Skalare Kraft
    \[ F_{ij} = \frac{G m_i m_j}{\lVert \vec r_i - \vec r_j\rVert^2} \]
    
    \item \textbf{Hamiltonian (Wechselwirkung):} Tensor-Kopplung
    \[ H_{\text{Wechsel}} \sim \kappa_{\mathrm Z}\sum_{i<j} T^{\mu\nu}{(i)}h_{\mu\nu}^{(j)} \]
    
    \item \textbf{Pfadintegral (QG):} Austausch von Ticks
    \[ Z[J^{\mu\nu}] = \int \mathcal Dh_{\alpha\beta} \exp\Big(i S_{\text{grav}}[h] + i\sum_k \int d^4x J^{\mu\nu}_k(x)h_{\mu\nu}(x)\Big) \]
    
    \item \textbf{Streuamplitude (Tree-Level):} 
    \[ \mathcal M \propto \kappa_{\mathrm Z}^2 \frac{T^{\mu\nu}{(1)}(q)P_{\mu\nu\rho\sigma}(q)T^{\rho\sigma}_{(2)}(-q)}{q^2} \]
    Hierbei ist der Propagator $1/q^2$ der geometrische Träger der Ticks.
\end{enumerate}

\section{Konzeptuelle Vertiefung}

\subsection{Warum Spin-2?}
Spin-2-Felder koppeln quadratisch. Während lineare Kopplung ($\alpha_G$) nur eine Wechselwirkung beschreibt, erlaubt die quadratische Struktur ($\chi \sim G M^2$), dass Energie-Impuls die Quelle seiner eigenen Geometrie ist. Gravitation ist keine Kraft, sondern eine \textbf{Spin-2-Resonanz zwischen Wirkung und Geometrie}.

\subsection{Ticks sind keine Zeit – sie sind Wirkung}
Der Tick-Index $i = \Delta A / \sigma_P$ zeigt:
\begin{itemize}
    \item Ein Tick ist keine Sekunde.
    \item Ein Tick ist ein diskretes Wirkungs-Update der Raumzeit.
    \item Entropie ist nicht statistisch („Unordnung“), sondern prozessual: Sie zählt die Anzahl der verarbeiteten Ticks.
\end{itemize}

\subsection{Hawking als Weltuhr}
Die Relation $E_H \cdot t_H = \hbar$ besagt: Ein emittiertes Hawking-Quant entspricht exakt einem Takt der fundamentalen Uhr. Schwarze Löcher verdampfen nicht zufällig, sie arbeiten ihre Tick-Schuld ab.

\section{Appendix: Der Rechenkern (Python-Skizze)}

Eine numerische Implementierung der $\sigma_P$-Kalkulation:

\begin{lstlisting}[language=Python]
import numpy as np

hbar = 1.054e-34
G = 6.674e-11
c = 2.998e8
kB = 1.381e-23

# Die fundamentale Zelle
sigma_P = hbar * G / c**4

def tick_count(E, t):
    """Berechnet die Anzahl der Ticks fuer einen Prozess (E, t)"""
    return (E * t) / sigma_P

def entropy(E, t):
    """Konvertiert Ticks in Entropie"""
    return kB * tick_count(E, t)

def alpha_G(M):
    """Lineare Gravitationskopplung"""
    return G * M / (hbar * c)

def chi(M):
    """Quadratische Spin-2 Kopplung"""
    return (G * M**2) / (hbar * c**3)

def lambda_eff(R, t):
    """Zeitabhaengige Kosmologische Konstante"""
    return 1.0 / (c * R * t)

def hawking_rate(M):
    """Tick-Rate der Evaporation"""
    chi_val = chi(M)
    return 1.0 / np.sqrt(chi_val)
\end{lstlisting}

\end{document}
