\documentclass[11pt,a4paper]{article}
\usepackage[utf8]{inputenc}
\usepackage[T1]{fontenc}
\usepackage{amsmath,amssymb}
\usepackage{microtype}
\usepackage{geometry}
\geometry{margin=25mm}
\usepackage{tcolorbox}
\title{Formelsammlung — Quantum Fruits}
\author{Adrian Zander}
\date{\today}

\begin{document}
\maketitle
\section*{Kurzbeschreibung}
Zusammenstellung der zentralen mathematischen Identitäten des \textit{Quantum Fruits}-Rahmenwerks. Diese Sammlung definiert die parameterfreie, dynamische Feldgleichung und die effektiven Skalierungsgesetze, die sich aus der Planck-Kovarianz ergeben.

\section{Parameterfreie, dynamische Feldgleichung}

Wir beginnen mit der Wirkungsfunktion:
\begin{equation}
S[g, \Psi] 
= \frac{c^3}{16\pi G} \int d^4x \, \sqrt{-g} \, \Big(R - 2\,\Lambda_{\rm eff}(t)\Big) 
+ S_{\rm m}^{(\sigma)}[g, \Psi] \,,
\end{equation}
wobei
\begin{itemize}
    \item $R$ der Ricci-Skalar der Raumzeitmetrik $g_{\mu\nu}$ ist,
    \item $\Lambda_{\rm eff}(t) \sim 1/(c R t)$ die stochastisch-entropische Glättung über die zugängliche Raumzeit beschreibt,
    \item $S_{\rm m}^{(\sigma)}[g,\Psi]$ die Planck-kovariante Mittelung der Materieaktion darstellt.
\end{itemize}

Die Variation der Wirkung liefert die Feldgleichung:
\begin{equation}
G_{\mu\nu} + \Lambda_{\rm eff}(t) \, g_{\mu\nu} = \frac{8 \pi G}{c^4} \, \langle T_{\mu\nu}^{(\sigma)} \rangle \,,
\end{equation}
mit
\begin{equation}
\langle T_{\mu\nu}^{(\sigma)} \rangle = -\frac{2}{\sqrt{-g}} \frac{\delta S_{\rm m}^{(\sigma)}[g,\Psi]}{\delta g^{\mu\nu}} \,,
\end{equation}
die **stochastisch gemittelte Materieantwort** definiert.  

\subsection{Planck-kovariante Glättung}

Die Mittelung erfolgt über den Kernel
\begin{equation}
K_{\sigma_P}(x,y) = \frac{1}{(2\pi)^2 \ell_\ast^3 \tau_\ast} 
\exp\Bigg[
- \frac{|\vec x - \vec y|^2}{2 \ell_\ast^2} 
- \frac{(t_x - t_y)^2}{2 \tau_\ast^2}
\Bigg],
\end{equation}
mit
\begin{equation}
\ell_\ast = c \, \sigma_P, \qquad \tau_\ast = \frac{\sigma_P}{c}, \qquad 
\sigma_P = \frac{\hbar G}{c^4}.
\end{equation}

\subsection{Herleitung von $\Lambda_{\rm eff}(t)$}
Die effektive Vakuumenergiedichte entsteht nicht durch ein statisches Feld, sondern durch die endliche Informationsdichte des kausalen Horizonts. Wenn wir die Planck-Zelle $\sigma_P$ über das gesamte kausale 4-Volumen $V_4 \approx R(t) \cdot ct$ mitteln, erhalten wir den Skalierungsfaktor $\alpha_\sigma$:

\begin{equation}
    \alpha_\sigma(t) := \frac{\sigma_P}{V_4^{eff}} \approx \frac{\sigma_P}{R(t) \cdot c t}
\end{equation}

Die effektive Krümmung $\Lambda_{\rm eff}$ ergibt sich aus der „Verdünnung“ der planckschen Krümmung $1/\ell_P^2$ um diesen Faktor:

\begin{equation}
    \Lambda_{\rm eff}(t) = \frac{\alpha_\sigma(t)}{\ell_P^2} = \frac{\sigma_P}{R \cdot c t} \cdot \frac{1}{\ell_P^2}
\end{equation}

Mit $\ell_P^2 = c \sigma_P$ folgt:
\begin{equation}
    \Lambda_{\rm eff}(t) = \frac{\sigma_P}{R \cdot c t \cdot c \sigma_P} = \frac{1}{c^2 R(t) t}
\end{equation}
Im heutigen Universum ($R(t) \approx c/H_0$ und $t \approx 1/H_0$) folgt daraus:
\begin{equation}
    \Lambda_{\rm eff}(t_0) \approx \frac{1}{c^2 (c/H_0) (1/H_0)} = \frac{H_0^2}{c^4} \quad (\text{Skalierung}) \implies \Lambda_{obs} \approx \frac{H_0^2}{c^2}
\end{equation}
Dies stimmt in der Größenordnung exakt mit den Beobachtungen überein ($\sim 10^{-52} \text{m}^{-2}$), ohne dass eine Feinabstimmung von $10^{120}$ Größenordnungen notwendig ist. Die „Dunkle Energie“ ist ein geometrischer Projektionseffekt.

\section{Kernformeln}
\begin{equation}
\sigma_{\mathrm P} = \frac{\hbar\, G}{c^{4}}
\end{equation}

\begin{equation}
\alpha_{G}(M) = \frac{G M}{\hbar c}
\end{equation}

\begin{equation}
\chi(M) := \frac{G M^{2}}{\hbar c^{3}}
\end{equation}

\begin{equation}
i = \frac{E\cdot t}{\sigma_P},\qquad i_{\max} = \frac{\hbar}{\sigma_P} = \frac{c^4}{G}
\end{equation}

\begin{equation}
n_{\text{Tick}} \sim \frac{c^4}{G},\qquad \Delta A = \hbar
\end{equation}

\section{Entropie und Bekenstein-Kompatibilität}
\subsection{Tick-Definition}
Die fundamentale Zählgröße ist der Tick-Index $i$, definiert als integrierte Wirkung in Einheiten von $\sigma_P$:
\begin{equation}
    S = k_B \cdot N_{\rm ticks}
\end{equation}

\subsection{Beispiel: Schwarzschild-Schwarzes Loch}
Für ein Schwarzes Loch der Masse $M$ ist der Ereignishorizont eine Fläche $A = 4\pi R_s^2$ mit $R_s = 2GM/c^2$. Die klassische Bekenstein-Hawking-Entropie lautet:
\begin{equation}
    S_{BH} = \frac{k_B c^3 A}{4 G \hbar} = \frac{k_B A}{4 \ell_P^2}
\end{equation}
In unserem Modell entspricht die Entropie der Anzahl der diskreten Raumzeit-Zellen auf dem Horizont. Wir nutzen die Identität $\ell_P^2 = c \sigma_P$:
\begin{equation}
    S_{BH} = \frac{k_B A}{4 c \sigma_P}
\end{equation}
Setzen wir $A = 16 \pi G^2 M^2 / c^4$ ein:
\begin{equation}
    N_{\rm ticks} = \frac{16 \pi G^2 M^2 / c^4}{4 c (\hbar G / c^4)} = \frac{4 \pi G M^2}{\hbar c}
\end{equation}
Dies ist dimensionslos und proportional zu $M^2$, konsistent mit der Informationskapazität. Der Faktor $4\pi$ ist geometrisch bedingt.

Unser Ansatz $S = k_B N_{\rm ticks}$ reproduziert also exakt das Bekenstein-Verhalten, liefert aber eine anschauliche Interpretation: **Entropie ist die Anzahl der $\sigma_P$-Zellen, die geometrisch auf den Horizont projiziert werden.** Die "Ticks" sind hierbei die elementaren Raumzeit-Quanten, die den Horizont aufspannen.

\section{Einheiten und Konstanten}
\[
[\hbar] = M L^2 T^{-1},\qquad [G]=M^{-1}L^3T^{-2},\qquad M_P^2=\frac{\hbar c}{G}
\]

\section{Symbolverzeichnis (kurz)}
\begin{description}
  \item[$\sigma_P$] Aktions‑Größe (Planck‑ähnlich), $\sigma_P=\hbar G/c^4$.
  \item[$\alpha_G(M)$] Gravitationskopplung (Massenskala).
  \item[$\chi(M)$] dimensionsloser Parameter $\propto GM^2/(\hbar c^3)$.
  \item[$i$] Zählgröße / Tick‑Index: $i=E\,t/\sigma_P$.
  \item[$\ell_P,t_P,m_P$] Planck‑Längen/Zeit/Masse.
  \item[$k_B$] Boltzmann‑Konstante.
\end{description}

\end{document}
