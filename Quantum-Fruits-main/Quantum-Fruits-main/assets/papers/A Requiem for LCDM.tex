% !TEX program = lualatex
% ============================================================
% UNIVERSAL MAINFRAME — σₚ Framework
% Author: Adrian Zander (2025)
% ============================================================

\documentclass[11pt,a4paper]{article}

% --- Seitenlayout & Sprache ---
\usepackage[a4paper,margin=2.3cm]{geometry}
\usepackage[ngerman]{babel}
\usepackage{fontspec}
\setmainfont{TeX Gyre Termes}
\usepackage{microtype}
\usepackage{csquotes}

% --- Mathematik ---
\usepackage{amsmath,amssymb,amsthm,mathtools}
\usepackage{physics} % für \dv, \pdv etc., optional
\usepackage{siunitx}
\sisetup{locale=DE,detect-all}

% --- Grafik & Plot ---
\usepackage{graphicx}
\usepackage{xcolor}
\usepackage{tikz}
\usetikzlibrary{arrows.meta,calc,decorations.pathmorphing,positioning}
\usepackage{pgfplots}
\pgfplotsset{compat=1.18}
\usepackage{bm}
\usepackage{booktabs}
\usepackage[most]{tcolorbox}

\usepackage{tocloft}
\usepackage{hyperref}
\usepackage{bookmark}

% Schöne Schrift & Layout für Inhaltsverzeichnis
\renewcommand{\cfttoctitlefont}{\Large\bfseries}
\renewcommand{\cftsecfont}{\bfseries}
\renewcommand{\cftsubsecfont}{\itshape}
\setlength{\cftbeforesecskip}{4pt}

% Hyperlinks (funktionierende Referenzen im PDF)
\hypersetup{
  colorlinks=true,
  linkcolor=blue!50!black,
  urlcolor=cyan!60!black,
  citecolor=black,
  pdfauthor={Adrian Zander},
  pdftitle={The Natural Structure of Spacetime – σₚ Framework},
  pdfsubject={Unified Quantum–Relativistic Model},
  pdfkeywords={sigma_P, quantum gravity, cosmology, Planck scale, Zander equation}
}

% Damit das Inhaltsverzeichnis im PDF anklickbar ist
\setcounter{tocdepth}{2}   % 1=Sections, 2=Subsections, 3=Subsubsections
\setcounter{secnumdepth}{2} % Nummerierungstiefe

% --- Eigene Makros: Naturkonstanten & σₚ-Struktur ---
\newcommand{\lp}{\ell_{\mathrm{P}}}
\newcommand{\tp}{t_{\mathrm{P}}}
\newcommand{\mpP}{M_{\mathrm{P}}}
\newcommand{\sigmaP}{\sigma_{\mathrm{P}}}
\newcommand{\alphaSigma}{\alpha_{\sigma}}
\newcommand{\Lambdaeff}{\Lambda_{\mathrm{eff}}}
\newcommand{\GNewton}{G}
\newcommand{\kb}{k_{\mathrm{B}}}
\newcommand{\TZ}{\Theta_{\mathrm{Z}}}

% Signatur: (-,+,+,+)
\newcommand{\signatur}{(-,+,+,+)}

% Planck-Zelle, zentrale Relationen
% σ_P = ℏ G / c^4
% α_σ = σ_P / (R t)
% Λ_eff = 3 / (c R t)

% --- Theorem-Umgebungen (falls gebraucht) ---
\theoremstyle{definition}
\newtheorem{definition}{Definition}
\theoremstyle{remark}
\newtheorem{bemerkung}{Bemerkung}
\theoremstyle{plain}
\newtheorem{satz}{Satz}



% --- Literatur ---
\usepackage[backend=biber,style=phys,biblabel=brackets]{biblatex}
\addbibresource{references.bib}

% ============================================================
% Titel
% ============================================================
\title{
  \vspace{-1em}
  \Large A Requiem for LCDM\\[0.5em]
}
\author{Adrian Zander}
\date{August 2025}

\begin{document}
\maketitle

% ============================================================
% Abstract (erzählerisch + technisch)
% ============================================================
\begin{abstract}
Seit fast einem Jahrhundert beschreiben wir das Universum in getrennten Sprachen:
Quantenmechanik für das Kleine, Allgemeine Relativitätstheorie für das Große,
kosmologische Parameter als Fudge-Faktoren dazwischen.
Das Standardmodell der Kosmologie, $\Lambda$CDM (Lambda Cold Dark Matter), war eine beispiellos erfolgreiche
Effektivbeschreibung. Doch präzise Messungen (Planck, SH0ES, schwache Linsenmessungen,
SPARC, LIGO) und die ersten tiefen Felder des James-Webb-Teleskops\cite{JWST2023Mission} legen nahe:
Wir beobachten ein Universum, das zu früh zu strukturiert, zu kohärent und zu stabil ist,
um bloß das Ergebnis eines feinjustierten Vakuumterms und unsichtbarer Komponenten zu sein.
\newline
In diesem Werk wird eine konsistente Alternative entwickelt, die ohne zusätzliche Felder
und ohne freie Parameter auskommt. Ausgangspunkt ist nicht ein hypothetisches neues Teilchen,
sondern die endliche Struktur der Raumzeit selbst. Die fundamentale Zelle
\[
  \sigmaP = \lp\,\tp = \frac{\hbar \GNewton}{c^4}
\]
wird als natürliches Wirkungsquantum der Raumzeit interpretiert.
Daraus folgt eine dimensionslose Feinstruktur der Raumzeit
\[
  \alphaSigma = \frac{\sigmaP}{R t} \approx 10^{-123},
\]
sowie eine geometrische kosmologische Konstante
\[
  \Lambdaeff = \frac{3}{c R t},
\]
die direkt aus der Endlichkeit des beobachtbaren Kosmos resultiert.
Die sogenannte ``Vakuum-Katastrophe'' verschwindet: lokale Quantenschwankungen und
makroskopische Krümmung sind zwei konsistente Skalen derselben Geometrie.
\newline
Auf dieser Basis wird eine voll quantisierte Feldgleichung formuliert,
in der Gravitation, Quantenfelder und thermodynamische Information über ein einziges
invariantes Maß --- $\sigmaP$ --- verknüpft sind.
Ein zentraler Baustein ist die Zander-Funktionaltemperatur
\[
  \TZ(M) = \frac{c^3}{8\pi \GNewton\, M\, \kb\, \hbar},
\]
die nicht den divergenten Energieausstoß, sondern den kontinuierlichen Austausch von
Raumzeit-Wirkung pro Masseneinheit beschreibt.
Sie steigt während der Verdampfung eines Schwarzen Loches bis in die Nähe der Planckmasse an,
fällt dann wieder ab und hinterlässt ein stabiles, informationsspeicherndes Remnant.
Die resultierende Dynamik liefert:
eine glatte Page-Kurve (unitäre Informationsentwicklung),
endliche Krümmung an allen Skalen (keine Singularitäten),
sowie eine natürliche Erklärung für $\Lambdaeff = 3/(c R t)$.
\newline
In diesem Rahmen erscheinen Schwarze Löcher nicht als Endpunkte der Physik,
sondern als organisierte Gedächtnisstrukturen der Raumzeit.
Das Universum vergisst nicht:
\newline
Es strahlt, speichert und entwickelt sich in geometrischer Harmonie.
\newline
Alle Resultate beruhen ausschließlich auf den Naturkonstanten 
$\{\hbar,\, c,\, G,\, k_{\mathrm B}\}$, den daraus gebildeten natürlichen Einheiten und 
empirisch überprüfbaren Messungen.

\end{abstract}

% ============================================================
% Legende: Sprache, Symbole, Signaturen
% ============================================================
\clearpage
\section*{Legende der Symbole und Signatur-Gleichungen}
\addcontentsline{toc}{section}{Legende der Symbole und Signatur-Gleichungen }

% --- Box 1 ---
\begin{tcolorbox}[
  colback=white,
  colframe=black!70,
  arc=2mm,
  boxrule=0.5pt,
  title={\bfseries Fundamentale Konstanten und Planck-Einheiten\cite{Planck1901}}]

\begin{align*}
c &\approx 2{,}998\times10^8~\mathrm{m/s}
&\text{Lichtgeschwindigkeit}\\[2pt]
G &\approx 6{,}674\times10^{-11}~\mathrm{m^3/(kg\,s^2)}
&\text{Gravitationskonstante}\\[2pt]
\hbar &\approx 1{,}055\times10^{-34}~\mathrm{J\,s}
&\text{reduzierte Planck-Konstante}\\[2pt]
k_{\mathrm B} &\approx 1{,}381\times10^{-23}~\mathrm{J/K}
&\text{Boltzmann-Konstante}
\end{align*}

\begin{align*}
\ell_{\mathrm P}
&= \sqrt{\frac{\hbar G}{c^3}}
&\text{Planck-Länge}\\[2pt]
t_{\mathrm P}
&= \sqrt{\frac{\hbar G}{c^5}}
&\text{Planck-Zeit}\\[2pt]
M_{\mathrm P}
&= \sqrt{\frac{\hbar c}{G}}
&\text{Planck-Masse}\\[2pt]
\sigma_{\mathrm P}
&= \ell_{\mathrm P}t_{\mathrm P}
= \frac{\hbar G}{c^4}
&\text{Raumzeit-Quant}
\end{align*}
\end{tcolorbox}

\vspace{0.8em}

% --- Box 2 ---
\begin{tcolorbox}[
  colback=white,
  colframe=black!70,
  arc=2mm,
  boxrule=0.5pt,
  title={\bfseries Geometrie: Minkowski–Zander und Feldgleichungen\cite{Zander2025_ProblemOfTime}}]

\paragraph{Minkowski-Raum.}
\begin{align*}
ds^2 &= c^2 dt^2 - dx^2 - dy^2 - dz^2,\\
\gamma(v) &= \frac{1}{\sqrt{1-v^2/c^2}}.
\end{align*}

\paragraph{Raumzeit-Unschärfe.}
\[
\Delta x\,\Delta t \ge \sigma_{\mathrm P}.
\]

\paragraph{Minkowski–Zander-Metrik.}
\begin{align*}
\tilde{g}_{\mu\nu}(x)
&= \frac{g_{\mu\nu}(x)}{1+\sigma_{\mathrm P}^{-1}f(\mathcal{R}(x))},\\
d\tilde{s}^2 &= \tilde{g}_{\mu\nu}dx^\mu dx^\nu.
\end{align*}

\paragraph{Einstein–Zander-Gleichung.}
\begin{align*}
\left\langle
\widehat{G}_{\mu\nu}
+ \Lambda_{\mathrm{eff}}(W)\,g_{\mu\nu}
\right\rangle_{\sigma_{\mathrm P}}
&=
\frac{8\pi G}{c^4}
\left\langle
\widehat{T}_{\mu\nu}
\right\rangle_{\sigma_{\mathrm P}}^{(W)},\\
\Lambda_{\mathrm{eff}}(W)
&= \frac{3}{c R t}.
\end{align*}

\paragraph{Klassischer Grenzfall.}
\[
G_{\mu\nu} + \Lambda_{\mathrm{eff}}g_{\mu\nu}
= \frac{8\pi G}{c^4}\bar T_{\mu\nu}.
\]
\end{tcolorbox}

\vspace{0.8em}

% --- Box 3 ---
\begin{tcolorbox}[
  colback=white,
  colframe=black!70,
  arc=2mm,
  boxrule=0.5pt,
  title={\bfseries Schwarze Löcher, Entropie und Zander-Funktional}]

\paragraph{Schwarzschild-Radius.}
\[
r_s = \frac{2GM}{c^2}.
\]

\paragraph{Bekenstein–Hawking-Entropie.}
\[
S_{\mathrm{BH}} = \frac{k_{\mathrm B}A}{4\ell_{\mathrm P}^2},
\quad A = 4\pi r_s^2.
\]

\paragraph{Hawking-Temperatur.}
\[
T_H(M) = \frac{\hbar c^3}{8\pi G M k_{\mathrm B}}.
\]

\paragraph{Zander-Funktional.}
\[
\Theta_Z(M) = \frac{c^3}{8\pi G M k_{\mathrm B}\hbar}.
\]
Endlicher Wirkungsfluss der Raumzeit pro Masseneinheit:
statt divergenter Temperatur entsteht ein stabiler Remnant bei $M\!\sim\!M_{\mathrm P}$.\cite{Zander2025_MemoryOfSpacetime}

\end{tcolorbox}

\vspace{0.8em}

% --- Box 4 ---
\begin{tcolorbox}[
  colback=white,
  colframe=black!70,
  arc=2mm,
  boxrule=0.5pt,
  title={\bfseries Kosmologische Skalierung aus $\sigma_{\mathrm P}$}]

\paragraph{Planck-Zell-Krümmung.}
\[
\Lambda_{\mathrm{cell}} = \frac{3}{\ell_{\mathrm P}^2}.
\]

\paragraph{Anzahl der Raumzeit-Zellen.}
\[
N_\sigma = \frac{R\,t}{\sigma_{\mathrm P}} = \frac{R\,t\,c^4}{\hbar G}.
\]

\paragraph{Makroskopische kosmologische Konstante.}
\[
\Lambda_{\mathrm{geo}} = \frac{\Lambda_{\mathrm{cell}}}{N_\sigma}
= \frac{3}{c R t}.
\]
Die scheinbare „Vakuumkatastrophe“ verschwindet:
lokale QFT-Vakuumenergie und beobachtete $\Lambda$ sind zwei Skalen derselben σ$_{\mathrm P}$-Geometrie.\cite{Zander2025_ParameterFreeUnification}
\end{tcolorbox}

\begin{figure}[ht]
    \centering
    \includegraphics[width=0.4\linewidth]{Universe.jpg}
    \caption{
        \textbf{Sphärische Darstellung eines quantisierten Universums.}
           }
\end{figure}

\clearpage
\section{Einleitung}

Am Ende des 20.~Jahrhunderts glaubte die Kosmologie, ihren Frieden mit dem Universum geschlossen zu haben.  
Saul Perlmutter, Adam Riess und Brian Schmidt\cite{Perlmutter1999,Riess1998}
 beobachteten, dass sich ferne Supernovae schwächer zeigten, als es eine rein gravitativ gebremste Expansion erlauben würde.  
Die Deutung: Das Universum beschleunigt sich.  
Diese scheinbare Anomalie wurde nicht als Hinweis auf eine unvollständige Theorie verstanden, sondern als Gelegenheit, eine fehlende Energieform zu postulieren — \emph{dunkle Energie}.
\newline
Damit begann eine der seltsamsten Episoden der Wissenschaftsgeschichte:  
Die theoretische Physik feierte die Entdeckung des „beschleunigten Kosmos“, ohne seine Ursache zu kennen, und das Standardmodell der Kosmologie, das sogenannte $\Lambda$CDM-Modell, erhob das Dunkle zur Norm.  
$\Lambda$ (Lambda) stand fortan für eine kosmologische Konstante, die in ihrer beobachteten Größe um 123 Größenordnungen kleiner war als der theoretische Vakuumwert der Quantenfeldtheorie.  
Das \emph{größte Missverhältnis der Physik} wurde kurzerhand zur empirischen Tugend erklärt.
\newline
Doch während Theorien sich selbst feierten, begann das Universum, zurückzuschauen.  
Mit dem \emph{James Webb Space Telescope} (JWST) erhielt die Menschheit einen Blick in die ersten 300~Millionen Jahre nach dem Urknall – und fand dort etwas, das nicht hätte existieren dürfen:  
voll entwickelte Galaxien mit hoher Metallizität, rotierenden Scheiben, und Strukturen, die nach allen LCDM-Simulationen Milliarden Jahre zu früh erschienen.  
Das Teleskop, das die Dunkle-Energie-Kosmologie bestätigen sollte, bringt die eleganteste Fiktion der modernen Astrophysik ins Wanken.
\newline
Die Dunkelheit, so zeigt sich, war nie eine Eigenschaft des Universums – sie war ein Symptom unserer Gleichungen.  
Nicht das All war unvollständig, sondern unser mathematisches Raster, in dem Raum und Zeit noch als unendliche, kontinuierliche Größen behandelt wurden.  
Die Lösung liegt nicht in einem zusätzlichen Feld, sondern in der Struktur der Raumzeit selbst.
\newline
In diesem Werk wird gezeigt, dass eine einfache, aber fundamentale Relation alle wesentlichen Phänomene verbindet:
\[
\sigma_{\mathrm P} = \ell_{\mathrm P}t_{\mathrm P} = \frac{\hbar G}{c^4}.
\]
Dieses Raumzeit-Wirkungsquantum — das kleinste mögliche Produkt aus Raum und Zeit — bildet die natürliche Grenze aller physikalischen Teilbarkeit.  
Aus dieser endlichen Struktur folgt eine geometrische, parameterfreie Erklärung der kosmologischen Beschleunigung:
\[
\Lambda_{\mathrm{geo}} = \frac{3}{c R t}.
\]
Sie beschreibt kein unbekanntes Energiefeld, sondern das Maß der endlichen Raumzeit selbst.
\newline
Während $\Lambda$CDM seine Dunkelheit aus fehlenden Daten und hypothetischen Teilchen bezog, leitet sich im $\sigma_{\mathrm P}$-Rahmen dieselbe Beschleunigung aus der quantisierten Geometrie ab – aus der Tatsache, dass Raumzeit nicht beliebig fein teilbar ist.  
Damit verschwindet das Vakuumproblem, die Hubble-Tension wird natürlich eingebettet, und die Beobachtungen des JWST \cite{JWST2023EarlyGalaxies} erscheinen nicht mehr als Anomalien, sondern als direkte Bestätigungen einer endlichen, selbstkonsistenten Raumzeit.
\newline
Das Universum ist nicht dunkel.
\newline  
Es ist quantisiert.

\subsection*{Über die Reinheit der Gleichung}

Einsteins Feldgleichung\cite{Einstein1915} war nie bloß eine Rechenvorschrift.
Sie ist eine Aussage über das Gleichgewicht der Natur:
\[
G_{\mu\nu} + \Lambda g_{\mu\nu}
= \frac{8\pi G}{c^4}\,T_{\mu\nu}.
\]
Links steht die Geometrie — der Raum selbst, wie er sich krümmt;
rechts die Materie — das, was ihn formt.
Einsteins Genie lag nicht im Symbol $\Lambda$,
sondern in der Symmetrie dieser Bilanz.
\newline
Im späten 20.~Jahrhundert wurde dieses Gleichgewicht gebrochen.
Um die Annahme der beschleunigte Expansion zu erklären,
verschob man $\Lambda$ auf die Seite der Materie
und nannte das Resultat \emph{Dunkle Energie}.
Die Gleichung blieb korrekt,
doch ihre Bedeutung wurde entweiht:
Ein geometrisches Prinzip wurde als Fluid gedeutet.
\newline
Im $\sigma_{\mathrm P}$-Rahmen wird diese Fehlbuchung berichtigt.
$\Lambda$ ist keine Energieform,
sondern die Signatur einer endlichen Raumzeit:
\[
\Lambda_{\mathrm{geo}} = \frac{3}{c R t}.
\]
Die Dunkle Energie verschwindet – 
nicht, weil sie falsch gemessen wurde,
sondern weil sie nie auf der richtigen Seite stand
und deshalb nicht existiert.
\newline
Die Raumzeit selbst ist die Quelle der Beschleunigung:
Sie dehnt sich nicht aus, weil etwas sie drängt,
sondern weil sie in jedem Moment neu wirkt.
Das Universum folgt keinem Druck,
sondern dem Maß seiner eigenen Geometrie:
\[
\sigma_{\mathrm P} = \frac{\hbar G}{c^4}.
\]
Damit kehrt die Gleichung zu ihrer Reinheit zurück.
Geometrie und Materie sind wieder zwei Seiten derselben Wirklichkeit –
nicht getrennt, sondern quantisiert verbunden.

\begin{quote}
\textit{Man darf die geometrische Seite der Gleichung nicht verändern – 
\newline 
denn sie ist nicht Modell, sie ist Gesetz.}  
\\[0.5em]
Und das hätten auch unsere späteren Nobelpreisträger wissen müssen.
\end{quote}

\vspace{1em}
\begin{center}
\begin{tikzpicture}[>=latex,scale=1.0]
\tikzstyle{eqbox}=[rectangle,rounded corners=3pt,draw=black!70,fill=white,
                   minimum width=12cm,minimum height=1.8cm,align=center,inner sep=6pt]
\tikzstyle{arrow}=[->,thick,black!70]

% --- Einstein 1915 ---
\node[eqbox] (einstein) at (0,5.5) {
\textbf{Einstein 1915}\\[0.4em]
$\displaystyle
G_{\mu\nu} + \Lambda g_{\mu\nu}
= \frac{8\pi G}{c^4} T_{\mu\nu}$\\[0.3em]
{\small Geometrie links – Materie rechts (symmetrisch, unverändert)}
};

% --- Perlmutter 1998 (ΛCDM) ---
\node[eqbox] (lcdm) at (0,2.5) {
\textbf{Perlmutter et al. 1998 — $\Lambda$CDM}\\[0.4em]
$\displaystyle
G_{\mu\nu}
= \frac{8\pi G}{c^4}
\big(T_{\mu\nu} + T^{(\Lambda)}_{\mu\nu}\big),
\qquad
T^{(\Lambda)}_{\mu\nu}
= \rho_\Lambda c^2\, g_{\mu\nu},
\quad
\rho_\Lambda = \frac{\Lambda c^2}{8\pi G}$\\[0.3em]
{\small Geometrischer Term als Fluid interpretiert (Buchhalter-Trick)}
};

% --- Zander 2025 ---
\node[eqbox] (zander) at (0,-0.5) {
\textbf{Zander 2025 — quantisierte Planck-kovariante Rekonstruktion}\\[0.4em]
$\displaystyle
\Big\langle \widehat{G}_{\mu\nu}
+ \Lambda_{\mathrm{geo}} g_{\mu\nu} \Big\rangle_{\sigma_P}
= \frac{8\pi G}{c^4}
\Big\langle \widehat{T}_{\mu\nu} \Big\rangle_{\sigma_P},
\qquad
\Lambda_{\mathrm{geo}} = \frac{3}{c R t}$\\[0.3em]
{\small Geometrie bleibt links – endliche Raumzeitwirkung $\sigma_P$ ersetzt das Fluid}
};

% --- Arrows ---
\draw[arrow] (einstein.south) -- node[right,align=left,xshift=5pt]
  {\small 1998: $\Lambda$ wird \\ \small als Energiedichte umgedeutet} (lcdm.north);
\draw[arrow] (lcdm.south) -- node[right,align=left,xshift=5pt]
  {\small 2025: $\sigma_P$-Rekonstruktion \\ \small der geometrischen Seite} (zander.north);

\end{tikzpicture}
\end{center}

\subsection*{Anmerkung zur aktuellen Beobachtungslage}

Seit den "bahnbrechenden" Messungen von \textcite{Perlmutter1999,Riess1998} galt die beschleunigte Expansion des Universums als experimentell bestätigt.  
Doch in den letzten Jahren mehren sich Zweifel an der Robustheit dieser Schlussfolgerung.  
Mehrere Studien weisen darauf hin, dass systematische Unsicherheiten in der Standardisierung von Typ~Ia-Supernovae – den sogenannten „Standardkerzen“ der Kosmologie – die Interpretation erheblich beeinflussen könnten \parencite{Tripp1998,Kim2020,Lee2022}.
\newline  
Eigenschaften der Wirtsgalaxien, Metallizitätseffekte und Selektionsverzerrungen führen zu Helligkeitsabweichungen, die mit einer scheinbaren Beschleunigung verwechselt werden können.  
\textcite{Trotta2022} und \textcite{Dam2023} betonen, dass die Signifikanz der beobachteten Beschleunigung drastisch sinkt, wenn man die Homogenitätsannahmen des $\Lambda$CDM-Modells aufgibt oder eine redshiftabhängige Kalibrierung zulässt.
\newline  
Selbst die \emph{Pantheon+}-Kompilation \parencite{Scolnic2022}, die bislang als robustester Supernova-Datensatz gilt, steht unter erneuter Prüfung, da einige Korrekturverfahren implizit die Existenz einer beschleunigten Expansion voraussetzen.
  
\clearpage

\section{Das ΛCDM-Modell – und wie James Webb und WMAP seine Grenzen zeigen}

\subsection*{Das Standardmodell der Kosmologie}

Das ΛCDM-Modell (\emph{Lambda Cold Dark Matter}) gilt seit über zwei Jahrzehnten als die erfolgreichste Beschreibung des Universums.  
Es kombiniert die Allgemeine Relativitätstheorie mit einem homogenen, isotropen Universum (Friedmann–Lemaître–Robertson–Walker-Metrik) und drei Hauptkomponenten:
\begin{itemize}
\item sichtbare baryonische Materie ($\Omega_{\mathrm b}$),
\item kalte Dunkle Materie ($\Omega_{\mathrm c}$),
\item und eine kosmologische Konstante $\Lambda$, interpretiert als Dunkle Energie.
\end{itemize}

Das Modell beschreibt die Expansion des Universums über die Friedmann-Gleichung:
\[
H^2(t) = \frac{8\pi G}{3}\,\rho_{\mathrm{tot}} - \frac{k c^2}{R^2} + \frac{\Lambda c^2}{3},
\]
wobei $\rho_{\mathrm{tot}} = \rho_{\mathrm b} + \rho_{\mathrm c} + \rho_{\mathrm r}$ die gesamte Energiedichte darstellt.  
Im ΛCDM-Modell wird $\Lambda$ als konstantes, raumfüllendes Fluid behandelt – die sogenannte \emph{Dunkle Energie} – mit negativer Druckdichte $p = -\rho_\Lambda c^2$, die die Expansion beschleunigt.  
\newline
Allerdings besitzt dieses Modell keinerlei physikalischen Ursprung für $\Lambda$.
Es ist ein empirischer Fitparameter, eingeführt, um Supernova-Daten zu beschreiben \parencite{Perlmutter1999,Riess1998}, ohne aus der Geometrie selbst abgeleitet zu sein.
Das macht ΛCDM zu einem Modell mit mindestens sechs freien Parametern:
\[
\{\Omega_{\mathrm b},\,\Omega_{\mathrm c},\,\Omega_\Lambda,\,H_0,\,n_s,\,A_s\},
\]
deren Werte so gewählt werden, dass sie die beobachtete Hintergrundstrahlung und Galaxienverteilung reproduzieren.

\vspace{1em}
% In der Präambel:
% \usepackage{tikz}
% \usetikzlibrary{decorations.markings}

\begin{figure}[h]
\centering
\begin{tikzpicture}[scale=1.0]

  % Achsen
  \draw[->, thick] (0,0) -- (10.5,0) node[below right]{\small Multipolmoment $\ell$};
  \draw[->, thick] (0,0) -- (0,4.5) node[left]{\small $D_\ell = \ell(\ell+1)C_\ell / 2\pi$};

  % ΛCDM-Peaks (schematische Kurve)
  \draw[thick, blue!60!black, smooth, samples=100, domain=0:10]
    plot(\x,{4*exp(-0.1*\x)*sin(deg(\x*60/180*pi))^2});

  % kleine Dämpfung rechts (Silence Tail)
  \draw[thick, blue!60!black, domain=6:10]
    plot(\x,{0.4*exp(-0.2*(\x-6))});

  % Beschriftung der Peaks
  \node[blue!50!black, font=\scriptsize] at (1.8,3.3) {1.\ Peak};
  \node[blue!50!black, font=\scriptsize] at (4.0,2.4) {2.\ Peak};
  \node[blue!50!black, font=\scriptsize] at (6.2,1.4) {3.\ Peak};

  % σ_P-Erklärung rechts
  \draw[dashed, red!60!black, thick] (9,0) -- (9,3.8);
  \node[red!60!black, font=\scriptsize, align=left, right] at (9.1,2.8)
    {σ$_{\mathrm P}$-Skala\\
     (Planck-Grenze)};
  
  % kleine Textlabels
  \node[font=\scriptsize, align=center, text width=3cm] at (2.2,4.3)
    {Akustische \\ Schwankungen \\ im baryonischen Plasma};
  \node[font=\scriptsize, align=center, text width=3cm, text=gray] at (7.6,0.9)
    {Dämpfung \\ kleiner Skalen};

\end{tikzpicture}

\caption{
Schematisches CMB-Leistungsspektrum.
Jeder Peak steht für eine stehende akustische Welle im frühen Universum.
Im klassischen ΛCDM werden Form und Lage der Peaks über sechs freie Parameter gefittet;
im σ$_{\mathrm P}$-Rahmen ergeben sie sich direkt aus der endlichen Raumzeitwirkung
$\sigma_{\mathrm P} = \hbar G / c^4$.
}
\label{fig:powerspectrum}
\end{figure}


\subsection*{Die natürliche Herleitung von Λ}

Im σₚ-Rahmen ergibt sich $\Lambda$ nicht als Fitgröße, sondern als geometrische Konsequenz einer endlichen Raumzeit:
\[
\Lambda_{\mathrm{geo}} = \frac{3}{c R t}.
\]
Hierbei stehen $R$ und $t$ für die aktuelle Größe und das Alter des Universums, während $c$ die Lichtgeschwindigkeit bleibt.  
Diese Relation folgt unmittelbar aus der Forderung, dass die Raumzeit keine unendliche Ausdehnung, sondern eine endliche Wirkung pro Raumzeitquant besitzt:
\[
\sigma_{\mathrm P} = \frac{\hbar G}{c^4}.
\]
Damit ist $\Lambda_{\mathrm{geo}}$ kein empirischer Parameter, sondern eine makroskopische Projektion der quantisierten Raumzeitstruktur.  
Die kosmische Beschleunigung entsteht also nicht durch eine mysteriöse Energieform, sondern durch die Geometrie selbst.

\subsection*{James Webb und das Ende von ΛCDM}

Das James Webb Space Telescope (JWST) wurde unter anderem mit dem Ziel entwickelt, die Frühzeit des Universums zu beobachten und die ΛCDM-Vorhersagen zu bestätigen \parencite{JWST2023Mission}.  
Doch das Gegenteil trat ein: Bereits die ersten Deep-Field-Beobachtungen zeigten Galaxien bei Rotverschiebungen von $z>10$, deren Masse und Entwicklung nach ΛCDM unmöglich erscheinen \parencite{Labbe2023,JWST2023EarlyGalaxies}.  
Diese Galaxien – darunter CEERS-93316, GLASS-z12, GN-z11, GHZ2/GLASS-z10 – sind zu groß, zu hell und zu strukturiert, um nur wenige hundert Millionen Jahre nach dem Urknall entstanden zu sein.

\vspace{1em}
\begin{center}
\includegraphics[width=0.7\textwidth]{highZ.png}\\
{\small Abb.~2: JWST-Beobachtungen massereicher Galaxien ($z \sim 9-12$), die nach ΛCDM nicht existieren dürften \parencite{Labbe2023}.}
\end{center}

Die einfachste Erklärung lautet:
\newline  
Diese Galaxien wirken nicht zu jung – das Modell, das sie beschreibt, ist falsch.  
Im natürlichem Rahmen liegt ihre Größe nicht in einem kosmischen Widerspruch, sondern in ihrer Position.  
Sie befinden sich am äußeren Rand des beobachtbaren Universums, also in einer Epoche, deren Licht uns erst jetzt erreicht.  
Je weiter eine Galaxie entfernt ist, desto weiter blicken wir in die Vergangenheit – und zugleich in Regionen, in denen die Raumzeit dichter, heißer und wirkungsreicher war.  
Was in der klassischen Kosmologie als „zu früh gebildet“ erscheint, ist hier schlicht eine Folge der endlichen Raumzeitwirkung:  
Diese Systeme sind groß, weil sie alt sind – sie tragen die Signatur einer jungen, energiegeladenen Raumzeit, nicht den Fehler eines Modells.


\subsection*{Von der Röhrenmetrik zur sphärischen Universum}

Zur Veranschaulichung des geometrischen Unterschieds kann die klassische FLRW-Metrik in 1+1 Dimensionen als „Röhre“ dargestellt werden – ein expandierendes, offenes Band der Zeitentwicklung.  
In der Empirie hingegen ergibt sich eine sphärische, geschlossenes Universum, in der kein Zentrum existiert, sondern jedes Ereignis auf derselben quantisierten Geometrie liegt.  

\vspace{1em}
\begin{center}
\includegraphics[width=0.8\textwidth]{tube.png}\\
{\small Abb.~3: 1+1D-Röhrenmetrik (klassisch, ΛCDM)}
\end{center}


\includegraphics[width=0.5\textwidth]{Planck Wmap.png}
\includegraphics[width=0.5\textwidth]{earth.jpg}\\
{\small Abb.~4: WMAP/Planck - CMB (Kosmische Hintergrundstrahlung) Afterglow wie in Abb. 3\\Abb.~5: Erde als bekannte Winkel-Tripel Darstellung.Hier erscheint die Erde Flach. Genauso wie die CMB in dieser Darstellung flach erscheint. Die CMB-Karte ist keine Scheibe. Sie ist eine Sphäre.}

\clearpage
\subsection*{Fazit: Der letzte Tanz der Röhre}

Das $\Lambda$CDM-Modell war kein Triumph erster Prinzipien,
sondern eine global erfolgreiche Notlösung:
ein sechsparametriger Fit mit zwei unsichtbaren Komponenten.
\emph{Dunkle Materie} -- seit Jahrzehnten gesucht, bis heute nicht direkt nachgewiesen.
\emph{Dunkle Energie} -- eingeführt als effektives Fluid,
um eine scheinbar beschleunigte Expansion zu modellieren.
Beide entstanden nicht aus Naturkonstanten,
sondern aus der Notwendigkeit, eine bereits gewählte Gleichungsseite zu retten.
\newline
Der entscheidende Schritt war konzeptionell:
\newline
Der ursprünglich geometrische Term $\Lambda g_{\mu\nu}$ wurde aus der linken Seite
von Einsteins Feldgleichung herausgelöst und als
\[
T^{(\Lambda)}_{\mu\nu} = -\frac{c^4}{8\pi G}\,\Lambda g_{\mu\nu}
\]
auf die rechte Seite verschoben -- interpretiert als „Energiedichte des Vakuums“.
Damit wurde ein Krümmungsterm der Geometrie zu einer Form von Materie erklärt.
Ein buchhalterischer Trick, der funktional ist,
aber die konzeptionelle Klarheit der Feldgleichungen untergräbt:
\[
\underbrace{G_{\mu\nu}}_{\text{Geometrie}}
= \frac{8\pi G}{c^4}\underbrace{\bigl(T_{\mu\nu} + T^{(\Lambda)}_{\mu\nu}\bigr)}_{\text{Materie + „Dunkles Fluid“}}.
\]

Die „1+1D-Röhrenmetrik“ und ihre Visualisierungen lieferten das passende Narrativ:
ein sich aufblasen\-des Gummituch, ein expandierender Gartenschlauch,
ein Universum, das „wegen“ eines negativen Drucks schneller läuft.
Solange Supernova-Daten, CMB-Fits und Linseneffekte kompatibel erschienen,
wirkte das Bild stabil genug, um lehrbuchtauglich zu werden.
\newline
Dann kam das \emph{James Webb Space Telescope (JWST)}.
\newline
Ein Observatorium im Milliardenmaßstab,
entwickelt, um das Standardmodell der Kosmologie zu präzisieren.
Stattdessen liefert es in den ersten Beobachtungsjahren Objekte,
die das $\Lambda$CDM-Narrativ unter Druck setzen:
massereiche, hoch strukturierte, metallreiche Galaxien in Rotverschiebungsbereichen,
in denen das Modell nur zaghafte Proto-Systeme erwartet.
Die vermeintlichen „zu frühen, zu großen“ Galaxien sind kein Skandal der Natur,
sondern ein Hinweis auf eine Überdehnung des Modells.
\newline
Die einfachste Lesart lautet:
Diese Systeme sind nicht „zu jung für ihre Masse“,
sondern unser kosmologisches Alters- und Expansionsmodell ist zu rigide.
Im $\sigma_{\mathrm P}$-Rahmen verschwindet das Paradoxon:
Strukturen erscheinen groß, weil sie alt sind --
weil sie am Rand unseres Beobachtungsfensters liegen
und die endliche, quantisierte Raumzeitwirkung ihre Entwicklung effizienter macht
als ein unendlich homogenisiertes Kontinuum.
\newline
In diesem Bild ist $\Lambda$ keine mysteriöse Flüssigkeit,
sondern das, was es bei Einstein ursprünglich war:
ein Maß globaler Geometrie.
Die effektive kosmologische Skala folgt als
\[
\Lambda_{\mathrm{geo}} = \frac{3}{c\,R\,t},
\]
direkt aus Planck-Skalen und Beobachtungsfenster $(R,t)$,
ohne freien Fitparameter, ohne hypothetisches Vakuumfluid.

\begin{quote}
\textit{Die Dunkle Energie verschwindet nicht, weil sie falsch gemessen wurde,\\
sondern weil sie nie auf der richtigen Seite der Gleichung stand und schlichtweg nicht existiert.}
\end{quote}

$\Lambda$CDM war eine beeindruckend erfolgreiche Arbeitsapproximation
für eine Phase der Datenlage.
\newline
Im Licht neuer Beobachtungen zeigt sich:
\newline
Nicht das Universum war „dunkel“,
sondern unsere buchhalterische Interpretation der Feldgleichungen.
\newline
Weniger Röhre, mehr 3d.
\newline
Weniger freie Parameter, mehr Naturkonstanten.
\newline
Weniger ad hoc-Fluide, mehr Geometrie.

\subsection*{Verantwortung, Geld und Mythos}

Wir sagen es klar: \(\Lambda\)CDM ist falsch.  
Es war eine arbeitsfähige Formallösung — ein sechsparametriger Fit, der
mit unsichtbaren Komponenten operierte und zur Grundlage eines Milliarden-dollar-
Wissenschaftsbetriebs wurde. Die Konsequenzen sind real: enorme Infrastruktur-
Investitionen, ganze Forschungsprogramme und eine Narrativ-Industrie, die
auf einem buchhalterischen Umzug des Λ-Terms beruhte.
\newline
Diese Rechnung hatte reale Kosten: fiskalische, intellektuelle und reputative.
Nobelpreise und hohe Forschungsetats belohnten ein Modell, das funktional
gut genug war, um jahrzehntelang zu dominieren — nicht, weil es prinzipiell
richtig gewesen wäre, sondern weil es in jener Datenlage am besten passte.
\newline
Jedoch nur in einem 1d Raum und einem Zeitpfeil.
Die größten Infrastrukturprojekte (CMB-Satelliten, große Observatorien,
Rechenzentren) waren Investitionen in ein Modellrahmenwerk; sie waren
wissenschaftlich fruchtbar, liefern Daten und Erkenntnisse — doch sie
verschlangen auch Ressourcen, die andernfalls in alternative Theorien
oder unabhängige Überprüfungen hätten fließen können.
\newline
Wenn ein Milliardenprogramm wie das \emph{James Webb Space Telescope}
dazu führt, dass das Standardmodell seiner ersten robusteren Tests
nicht standhält, dann ist das keine persönliche Niederlage,
sondern ein institutionelles Alarmsignal:
Geld, Prestige und Reputation dürfen niemals die intellektuelle
Neugier ersetzen, die Modelle immer wieder grundlegend in Frage stellt.
\newline
Kurz: Die Kosmologie hat ihre Rechnung gestellt — die Rechnung ist fällig.
\newline
Wir behalten uns vor, die historischen Entscheidungen pointiert zu benennen,
aber wissenschaftliche Kritik muss stets sauber, belegbar und fair bleiben.
\subsection*{Das Nichts, das zu viel wusste}

Die sogenannte \emph{Energiedichte des Vakuums} ist eine der seltsamsten Ideen,
die je in eine Gleichung gegossen wurden.
Man stelle sich vor: Der Raum ist leer – wirklich leer, frei von Teilchen und Feldern.
Und dann sagt die Theorie:
„Doch, da ist etwas. Nämlich Energie. Sehr viel sogar.“
\newline
Wie viel „Nichts“ passt ins „Nichts“?
Laut Quantenfeldtheorie: unendlich viel.
Und wie dicht ist dieses Nichts?
Etwa $10^{120}$ Mal dichter als alles, was wir je beobachtet haben.
\newline
Das entspricht einer Behauptung,
die Stille in einem Raum sei so laut, dass sie Mauern zum Einsturz bringen müsste –
wir hörten sie nur nicht, weil unsere Ohren zu unempfindlich sind.
\newline
Physikalisch bedeutet das:
Ein leerer Raum enthält nach dieser Rechnung genug Energie,
um unzählige Universen zu erzeugen.
Und doch sitzen wir ruhig darin, ohne dass der Stuhl explodiert.
\newline
Dass man diesen Rechenfehler nicht korrigierte,
sondern das Vakuum kurzerhand zur Energiequelle des Kosmos erklärte,
war ein genialer Akt kreativer Selbstrettung.
Einstein hatte $\Lambda$ ursprünglich als Krümmungsterm formuliert –
später stellte man ihn einfach auf die andere Seite der Gleichung,
kleidete ihn in die Sprache der Energie,
und ließ ihn als „Dunkle Energie“ Karriere machen.
\newline
Das Ergebnis war ein „Vakuum“, das
– ohne Teilchen, ohne Struktur, ohne messbare Signatur –
plötzlich Dichte, Druck und kosmische Dynamik besitzen sollte.
\newline
Das ist keine Physik im Sinne von Messbarkeit,
sondern eine bemerkenswerte Form theoretischer Buchhaltung:
\[
\text{Ich kann nichts sehen, nichts messen, nichts nachweisen – aber ich kann’s fitten.}
\]

\medskip
Denn genau das war es ursprünglich: ein Rechenfehler.
Die Quantenfeldtheorie summierte Nullpunktenergien über alle möglichen Feldmoden –
bis unendlich. Ohne Grenzwert, ohne physikalische Regularisierung.
Das Ergebnis war keine Messgröße, sondern ein divergenter Integralausdruck.
Man wusste das – und schrieb es trotzdem in die Gleichungen,
in der Hoffnung, die Natur möge es irgendwie „ausmitteln“.
\newline
Einstein hatte $\Lambda$ als Krümmungsterm verstanden,
nicht als Reservoir unermesslicher Energie.
Doch später stellte man ihn einfach auf die andere Seite der Gleichung,
kleidete ihn in die Sprache der Energie,
und ließ ihn als „Dunkle Energie“ Karriere machen.
\newline
Das Ergebnis war ein „Vakuum“, das
– ohne Teilchen, ohne Struktur, ohne messbare Signatur –
plötzlich Dichte, Druck und kosmische Dynamik besitzen sollte.
\clearpage
Das Nichts als Energiequelle: eine Erfolgsgeschichte in sechs Nobelpreisen und null Teilchen.
Die Geschichte beginnt mit der Annahme, dass das Vakuum – traditionell „leer“ – eine enorme Energiedichte besitzt.
Aus einer quantenfeldtheoretischen Fußnote wurde über die Jahrzehnte eine tragende Säule der modernen Kosmologie.
Spätestens seit den Supernova-Messungen von 1998 gilt die Dunkle Energie als bevorzugte Erklärung
für jede beobachtete kosmische Beschleunigung.
Doch weder Laborversuche noch Teilchenbeschleuniger oder Astrometrie liefern je einen direkten Hinweis darauf.
Stattdessen wuchs die Theorie durch Fits und Parameterpflege.
Sechs Nobelpreise, unzählige Simulationen – aber kein Teilchen, kein Nachweis.
So wird aus einem mathematischen Restterm eine kosmologische Substanz.

\subsection{Das Vakuumenergie‐Desaster — oder: Wie viel „Nichts“ passt ins Nichts?}

Kaum ein Konzept zeigt so deutlich die Kluft zwischen Theorie und Beobachtung wie die sogenannte \emph{Vakuumenergie}.
Die Quantenfeldtheorie postuliert, dass selbst der leerste Raum voller Nullpunktschwingungen ist.
Addiert man alle bis zur Planckgrenze, ergibt sich:
\begin{equation}
  \rho_{\mathrm{vac}}^{\mathrm{QFT}}
  \sim \frac{E_P}{\ell_P^3}
  \approx 10^{113}\,\si{\joule\per\metre\cubed}.
\end{equation}
Ein Kubikzentimeter „Nichts“ enthielte nach dieser Rechnung mehr Energie als eine Galaxie.
Und doch bleibt der Kosmos bemerkenswert ruhig.

Beobachtungen zeigen dagegen:
\begin{equation}
  \rho_{\mathrm{vac}}^{\mathrm{obs}} \approx 5\times10^{-10}\,\si{\joule\per\metre\cubed},
\end{equation}
eine Abweichung von rund $10^{122}$ Größenordnungen – das größte Missverhältnis der modernen Physik.

\subsubsection*{Der Buchhaltungsfehler des Jahrhunderts}

Die Ursache liegt nicht in der Physik selbst, sondern in der Annahme eines unendlichen Kontinuums.
Die Quantenfeldtheorie zählt unendlich viele Freiheitsgrade pro Raumzeitpunkt,
als könne die Natur beliebig fein geteilt werden.
Doch Raum und Zeit sind keine glatten Flächen – sie bestehen aus endlichen Einheiten:
\begin{equation}
  \sigma_P = \ell_P t_P = \frac{\hbar G}{c^4}.
\end{equation}
Dieses Raumzeit-Quantum ist die kleinste messbare Wirkungseinheit des Universums.
Es setzt der Zählung der Freiheitsgrade eine physikalische Grenze.

\subsubsection*{Die Maßfrage des Universums}

Das beobachtbare Universum besitzt eine endliche Ausdehnung $R$ und ein endliches Alter $t$.
Je nach Messmethode variieren diese Werte leicht – so wie zwei Maßbänder, die beide korrekt sind,
aber an unterschiedlichen Enden ansetzen.
Die Kombination $R t$ beschreibt die „Raumzeit-Fläche“ unseres Beobachtungsfensters.

\vspace{0.5em}
\noindent
Die drei gängigen Kalibrierungen lauten:
\begin{itemize}
  \item \textbf{Planck (2018):} $H_0 = 67.4~\si{km/s/Mpc}$, $R = 1.37\times10^{26}~\si{m}$, $t = 13.8~\si{Gyr}$.
  \item \textbf{SH0ES (2022):} $H_0 = 73.0~\si{km/s/Mpc}$, $R = 1.23\times10^{26}~\si{m}$, $t = 13.4~\si{Gyr}$.
  \item \textbf{Mittelwert:} $H_0 = 70.2~\si{km/s/Mpc}$, $R = 1.31\times10^{26}~\si{m}$, $t = 13.6~\si{Gyr}$.
\end{itemize}
Alle drei beschreiben dasselbe Universum – nur mit leicht unterschiedlichen Maßstäben.
Die Unterschiede zeigen sich nicht in der Physik, sondern in der Definition des Beobachtungsfensters.

\subsubsection*{Endliche Maßstäbe statt unendlicher Integrale}

Setzt man die jeweilige Skala in Beziehung zur Planck-Zelle,
\begin{equation}
  N_\sigma = \frac{R\,t}{\sigma_P},
\end{equation}
so erhält man die Anzahl der quantisierten Raumzeit-Zellen im beobachtbaren Universum:
\begin{align*}
  N_\sigma^{\mathrm{Planck}} &\approx 5.97\times10^{43} / 8.71\times10^{-79} \approx 6.8\times10^{121},\\
  N_\sigma^{\mathrm{SH0ES}} &\approx 5.2\times10^{43} / 8.71\times10^{-79} \approx 6.0\times10^{121},\\
  N_\sigma^{\mathrm{Mittel}} &\approx 5.6\times10^{43} / 8.71\times10^{-79} \approx 6.4\times10^{121}.
\end{align*}
Das Universum besteht also aus etwa $10^{122}$ elementaren Raumzeit-Quanten.
Dies ist der natürliche Reduktionsfaktor, der die scheinbare Diskrepanz der Quantenfeldtheorie auflöst.

\subsubsection*{Die natürliche Vakuumdichte aus Geometrie}

Die effektive Vakuumenergiedichte ergibt sich nicht aus einem Feld,
sondern aus der Geometrie der endlichen Raumzeit:
\begin{equation}
  \rho_{\mathrm{vac}}^{\mathrm{geo}}
  = \frac{3c^{3}}{8\pi G R t}.
\end{equation}
Trägt man hier die verschiedenen Beobachtungsfenster ein, ergibt sich:

\begin{table}[h!]
\centering
\caption*{\textbf{Geometrische Vakuumdichte für verschiedene Beobachtungsfenster}}
\renewcommand{\arraystretch}{1.2}
\setlength{\tabcolsep}{6pt}
\begin{tabular}{lccc}
\hline
\textbf{Quelle} & $R\,t$ [m·s] & $\Lambda_{\mathrm{geo}}=\frac{3}{cRt}$ [m$^{-2}$] & $\rho_{\mathrm{vac}}^{\mathrm{geo}}$ [J/m$^3$] \\ \hline
Planck (2018) & $5.97\times10^{43}$ & $1.68\times10^{-52}$ & $5.3\times10^{-10}$ \\
SH0ES (2022)  & $5.23\times10^{43}$ & $1.91\times10^{-52}$ & $6.0\times10^{-10}$ \\
Mittelwert    & $5.60\times10^{43}$ & $1.79\times10^{-52}$ & $5.6\times10^{-10}$ \\
JWST (2024)   & $6.00\times10^{43}$ & $1.67\times10^{-52}$ & $5.1\times10^{-10}$ \\ \hline
\end{tabular}
\end{table}

Alle Werte liegen im Bereich der beobachteten kosmologischen Energiedichte –
und zwar ohne freie Parameter oder zusätzliche Felder.
Der Unterschied zwischen Planck und SH0ES entspricht nur einer geometrischen Variation des Produktes $(R t)$ um etwa 0.13.
Die „Hubble-Tension“ ist damit keine physikalische, sondern eine Maßstabsfrage:
zwei unterschiedliche Lineale, dieselbe Raumzeit.

\subsubsection*{Das Verhältnis der Maßstäbe}

Die klassische Diskrepanz von \(10^{122}\) ergibt sich direkt aus dem Verhältnis zwischen makroskopischer und mikroskopischer Skala:
\begin{equation}
  \frac{\rho_{\mathrm{vac}}^{\mathrm{QFT}}}{\rho_{\mathrm{vac}}^{\mathrm{geo}}}
  = \frac{R\,t}{\sigma_P} \approx 10^{122}.
\end{equation}
Der vermeintliche Katastrophenfaktor ist also kein Rätsel, sondern das natürliche Maß der endlichen Raumzeit.

\begin{quote}
\centering
\textit{
Die Dunkle Energie ist keine Substanz,\\
sondern das Maß der eigenen Geometrie.}
\end{quote}

\paragraph{Schlussfolgerung.}
\emph{
Die Differenz zwischen theoretischer und beobachteter Vakuumenergie
entspricht exakt der Anzahl quantisierter Raumzeit-Zellen im Universum.
Damit ist die kosmologische Konstante
\[
\Lambda_{\mathrm{geo}} = \frac{3}{cRt}
\]
keine Energieform, sondern die Signatur einer endlichen, gequantelten Raumzeit.
Planck, SH0ES und JWST messen nicht unterschiedliche Universen –
sie messen dasselbe σ$_P$-geometrische Prinzip durch leicht verschobene Fenster.}

\clearpage
\subsection{Wie man ernsthaft nach der „Dichte des Vakuums“ fragen kann — und was dabei schiefläuft}

Nehmen wir zwei Wörter:

\begin{description}
  \item[\textbf{Dichte:}] Klassisch definiert als Masse pro Volumen oder Energie pro Raum. 
  Ein Maß für etwas, das \emph{da ist}.
  \item[\textbf{Vakuum:}] Klassisch verstanden als Leere. 
  Kein Teilchen, kein Feld, kein Etwas. 
  Ein Maß für das, was \emph{nicht da ist}.
\end{description}

Nun versuche, beides in denselben Satz zu packen, ohne innerlich zusammenzubrechen:

\begin{quote}
\centering
\textit{„Wie groß ist die Dichte der Leere?“}
\end{quote}

Herzlichen Glückwunsch: 
Man hat soeben ein semantisches Oxymoron erschaffen.  
Es ist, als würde man fragen:
\begin{itemize}
  \item Wie warm ist absolute Kälte?
  \item Wie schwer ist Bedeutungslosigkeit?
  \item Wie viel Ton wiegt Stille?
\end{itemize}

Und doch stellen Menschen mit Doktortiteln genau diese Frage.  

\subsection*{Kategoriefehler mit Einheiten}

Die Frage nach der \emph{Dichte des Vakuums} ist keine physikalische, 
sondern eine kategoriale Verwechslung.  
Man fragt nach einer Eigenschaft, wo kein Träger existiert.  
Das ist, als wollte man wissen, wie salzig eine Pause ist.  
\newline
Das Vakuum trägt keine Energie – es ermöglicht sie.  
Die Dichte gehört dem, was \emph{in} ihm geschieht, nicht dem Raum selbst.


\begin{figure}[htbp]
    \centering
    \includegraphics[width=0.5\textwidth]{Universe2.jpg}
    \caption{Visualisierung der kosmischen Mikrowellenhintergrundstrahlung (CMB) als sphärische Struktur basierend auf WMAP-Daten.Diese Darstellung abstrahiert die typischen 2D-Kartenprojektionen und illustriert die CMB als isotrope Strahlung, die das frühe Universum in alle Richtungen durchdringt.}
    \label{fig:universe2}
\end{figure}


\clearpage
\section{Das Dunkle-Materie-Narrativ – Trägheit, Gravitation und die Geometrie der Rotation}

\subsection*{Ursprung des Problems}

Die Geschichte der „Dunklen Materie“ begann 1933, als Fritz Zwicky bei der Untersuchung des Coma-Galaxienhaufens
eine Diskrepanz zwischen sichtbarer Masse und gravitativer Bindungskraft bemerkte \parencite{Zwicky1933}.
Er folgerte, dass es eine unsichtbare Masse geben müsse, um die beobachteten Umlaufgeschwindigkeiten zu erklären.
Vier Jahrzehnte später bestätigte Vera Rubin mit präzisen Rotationskurven einzelner Spiralgalaxien,
dass deren äußere Sterne sich zu schnell bewegen, um durch die sichtbare Masse gehalten zu werden \parencite{Rubin1980}.
Seither gilt die Hypothese:
\[
M_{\text{total}} = M_{\text{baryonisch}} + M_{\text{dunkel}},
\]
wobei der zweite Term nie direkt beobachtet wurde.
Trotz intensiver Suche (LUX, Xenon1T, LHC, DAMA/LIBRA, AMS-02)
wurde kein einziges Dunkelmaterieteilchen nachgewiesen.

\subsection*{Die Rotationskurve als Fenster in die Geometrie}

In klassischen Newton-Gleichungen nimmt die Rotationsgeschwindigkeit mit wachsendem Radius ab:
\[
v(r) = \sqrt{\frac{G M(r)}{r}}.
\]
Beobachtet wird jedoch:
\[
v(r) \approx \text{konstant},
\]
was einer radialen Beschleunigung
\[
g_{\text{obs}}(r) = \frac{v^2}{r}
\]
entspricht, die deutlich über dem durch sichtbare Masse erklärbaren
\[
g_{\text{bar}}(r) = \frac{G M_{\text{bar}}(r)}{r^2}
\]
liegt.
Um diesen Unterschied zu überbrücken, postuliert man Dunkle Materie mit einem Haloprofil,
meist in der Form eines Navarro–Frenk–White-Potentials~\cite{Navarro1997}.
Doch diese Profile sind rein phänomenologisch: Sie passen Kurven, ohne die Ursache zu erklären.

\subsection*{Das universelle Beschleunigungsmaß \( g_\ast = cH \)}

Die SPARC-Datenbank (Spitzer Photometry \& Accurate Rotation Curves)~\cite{Lelli2016}
zeigt, dass sich sämtliche Galaxien – unabhängig von Masse, Morphologie oder Umgebung –
durch eine einzige empirische Relation beschreiben lassen:
\[
g_{\text{obs}} = \frac{g_{\text{bar}}}{1 - e^{-\sqrt{g_{\text{bar}}/g_\ast}}}.
\]
Der Parameter \(g_\ast\) ist universell und entspricht numerisch
\[
g_\ast \approx 1.2\times10^{-10}\,\si{\metre\per\second\squared}.
\]
Das ist bemerkenswert nahe an
\[
cH_0 \approx (3.0\times10^{8})(2.3\times10^{-18})
\simeq 7\times10^{-10}\,\si{\metre\per\second\squared},
\]
also der durch Lichtgeschwindigkeit und Hubble-Rate bestimmten
kosmischen Grenzbeschleunigung.
Dies legt nahe, dass die „fehlende Masse“ gar keine lokale Eigenschaft ist,
sondern eine Folge der globalen Raumzeitgeometrie.

\subsection*{Geometrische Herleitung aus \texorpdfstring{$\sigma_P$}{σₚ}}

Im quantisierten σₚ-Rahmen ergibt sich die beobachtete Skalierung
ohne neue Parameter aus der Geometrie selbst:
\[
\sigma_P = \ell_P t_P = \frac{\hbar G}{c^4},
\qquad
\Lambda_{\mathrm{geo}} = \frac{3}{cRt}.
\]
Da Gravitation im Mittel durch die Krümmung pro Raumzeitquant vermittelt wird,
entsteht auf makroskopischer Skala ein Grenzgradient:
\[
g_\ast = \frac{c}{t_H} = cH,
\]
wobei \(t_H = 1/H\) die Hubble-Zeit ist.
Dies ist genau die Beschleunigung,
bei der die lokale Raumzeitwirkung einer Planck-Zelle
dem makroskopischen Expansionsfluss entspricht:
\[
\sigma_P^{-1} : (R,t) \;\Rightarrow\; g_\ast = \frac{c}{t_H}.
\]
Damit ist der Übergang von Newton’scher zu „flacher“ Rotation
kein dynamisches Rätsel, sondern eine geometrische Schwelle der Kausalität:
Ab \(g_{\text{bar}} \lesssim g_\ast\)
wirkt nicht mehr die lokale, sondern die globale Geometrie der Raumzeit.

\subsection*{Beobachtungsbestätigung – RAR und BTFR}

Diese Relation erscheint in zwei unabhängigen empirischen Gesetzen:

1. Radial Acceleration Relation (RAR):
   \[
   g_{\text{obs}} = f(g_{\text{bar}}) \;\text{mit}\;
   f(g) \approx \frac{g}{1 - e^{-\sqrt{g/g_\ast}}}.
   \]
   Alle bekannten Galaxien folgen derselben Kurve \parencite{McGaugh2016}.

2. Baryonic Tully–Fisher Relation (BTFR):
   \[
   M_{\mathrm{bar}} \propto v^4.
   \]
   Setzt man \(v^4 = G M_{\mathrm{bar}} g_\ast\),
   folgt dieselbe Grenzbeschleunigung \(g_\ast = cH\).
   Das ist kein Zufall, sondern die natürliche Konsequenz einer
   endlichen, skalenabhängigen Raumzeitwirkung.


\subsection*{Gravitation ist skalenbegrenzt, nicht „fehlend“}

Die Rotationskurven galaktischer Systeme erfordern keine unsichtbare Substanz.
Sie zeigen lediglich, dass Gravitation – als geometrisches Phänomen – 
nicht beliebig tief ins Schwache extrapoliert werden darf.
Bei Beschleunigungen unterhalb
\[
g_\ast = cH
\]
tritt die Raumzeit selbst als aktiver Akteur auf.
Das, was wir „Dunkle Materie“ nannten,
war nur das Echo der endlichen Raumzeitwirkung auf kosmischer Skala.

\begin{quote}
\centering
\textit{
Nicht die Masse fehlt –\\
uns fehlte das Verständnis für das Maß.}
\end{quote}


\clearpage

\begin{table}[h!]
\centering
\caption*{\textbf{Fensterabhängigkeit von $H_0$, $g_\ast$ und $\rho_{\mathrm{vac}}^{\mathrm{geo}}$}}
\renewcommand{\arraystretch}{1.25}
\begin{tabular}{lcccc}
\hline
\textbf{Quelle} & $H_0$ [km/s/Mpc] & $g_\ast = cH_0$ [m/s$^2$] & $\rho_{\mathrm{vac}}^{\mathrm{geo}}$ [J/m$^3$] & Bemerkung \\ \hline
Planck 2018 & 67.4 & $6.5\times10^{-10}$ & $2.1\times10^{-9}$ & CMB-basiert \\
SH0ES 2024 & 73.0 & $7.1\times10^{-10}$ & $2.5\times10^{-9}$ & Supernovae-basiert \\
Mittelwert & 70.2 & $6.8\times10^{-10}$ & $2.3\times10^{-9}$ & geometrischer Mittelwert \\ \hline
\end{tabular}
\end{table}

\subsubsection*{Einfach erklärt: Warum die Zahlen so erstaunlich gut passen}

Was hier passiert, ist eigentlich verblüffend einfach.
\newline
Man nehme die Ausdehnungsgeschwindigkeit des Universums, den sogenannten Hubble‐Wert~$H_0$.
Er sagt, wie schnell sich zwei weit entfernte Galaxien voneinander entfernen – etwa
70 Kilometer pro Sekunde pro Megaparsec Abstand.
Das klingt nach viel, ist aber kosmisch winzig:
in SI‐Einheiten sind das gerade einmal $2\times10^{-18}$ pro Sekunde.
\newline
Wenn man diesen Wert mit der Lichtgeschwindigkeit multipliziert, erhält man
\[
g_\ast = c\,H_0 \approx 7\times10^{-10}\,\si{\metre\per\second\squared}.
\]
Diese Zahl beschreibt eine fundamentale Grenze:
die kleinste messbare Beschleunigung im Universum.
Unterhalb dieses Wertes „spürt“ ein Körper nicht mehr nur lokale Gravitation,
sondern die Krümmung der gesamten Raumzeit.
\newline
Genau diese Schwelle taucht überall wieder auf – 
in den Rotationskurven der Galaxien, in der Dynamik von Sternhaufen
und sogar in der großräumigen Struktur des Kosmos.
Sie ist die feine Naht, an der Gravitation und Kosmologie
denselben Takt schlagen.
\newline
Und wenn man aus demselben Hubble‐Wert die theoretische Energiedichte
des Universums berechnet, erhält man:
\[
\rho_{\mathrm{vac}}^{\mathrm{geo}} = \frac{3c^3}{8\pi G R t}
\approx 2\times10^{-9}\,\si{\joule\per\metre\cubed}.
\]
Das ist – ganz ohne Dunkle Energie, ohne exotische Teilchen – fast exakt
der Wert, den Satelliten wie \emph{Planck} und \emph{WMAP} tatsächlich messen.
\newline
Das Universum verrät uns damit etwas sehr Einfaches:
Es braucht keine verborgene Kraft, um zu wachsen.
Seine Expansion ist eine Folge seiner endlichen Geometrie.
Je nachdem, welches „Beobachtungsfenster“ man wählt – 
ob die langsamere Planck‐Skala oder die schnellere SH0ES‐Messung –
ändert sich nur der Takt, nicht die Melodie.

\begin{quote}
\centering
\textit{
Das Universum tanzt immer zum gleichen Rhythmus.\\
Nur unsere Messinstrumente hören verschiedene Tempi.}
\end{quote}

Für den Laien mag es poetisch klingen, 
aber physikalisch bedeutet es:
\newline  
Raum und Zeit selbst bestimmen die Werte, die wir messen.  
Sobald man akzeptiert, dass Raumzeit nicht unendlich fein teilbar ist,
fallen alle „Feinabstimmungen“ und „Dunklen Kräfte“ von selbst aus der Rechnung.

\subsection{Der galaktische Exzess – oder wie man Dunkelheit simuliert}

Seit Jahrzehnten predigt die Kosmologie, dass über 80~\% der Materie im Universum „unsichtbar“ sei.  
Sie müsse da sein, sonst würden die Gleichungen nicht stimmen.  
So entstand eine ganze Schattenwelt aus Dunkler Materie, Dunkler Energie und Dunklen Erklärungen –  
eine kosmologische Theaterkulisse für alles, was sich mit normaler Physik nicht rechnen ließ.

\medskip
\noindent
Dann kam das galaktische Zentrum.  
Dort beobachtete man eine erhöhte Gammastrahlung – den sogenannten \emph{Galactic Center Excess (GCE)}:\\„Das muss Dunkle Materie sein! Sie vernichtet sich selbst und leuchtet dabei!“ 
\newline
Klingt spektakulär, bis man kurz nachdenkt.
\clearpage
\subsection*{Die ganz triviale Alternative}

Das galaktische Zentrum ist schlicht der dichteste Ort der Milchstraße.  
Dort wimmelt es von Sternen, Neutronensternen, Pulsaren, Akkretionsscheiben und Supernovae.  
Die Dichte ist etwa 200-mal höher als im galaktischen Diskus.  
Wo also mehr Sterne sind, explodieren mehr Sterne.  
Ganz ohne Teilchenphysik, ganz ohne Mystik.

\begin{figure}[h!]
\centering
\includegraphics[width=0.85\linewidth]{andromeda.jpg}
\caption{
\textbf{Andromeda (M31) als Beispiel für ein normales galaktisches Zentrum.}
Die hell leuchtende Mitte ist kein Hinweis auf „selbstzerstörende Dunkle Materie“,
sondern schlicht der Ort höchster Sterndichte:
Dort häufen sich Supernovae, Akkretionsscheiben und Pulsare.
Die Physik ist banal – mehr Sterne, mehr Energie, mehr Licht.
\\[0.3em]
\textit{Zum Vergleich:}
\href{https://www.aip.de/en/news/milkyway-gammaray-darkmatter-annihilation/}
{AIP (2025) – „Milky Way shows gamma ray excess due to dark matter annihilation“\cite{AIP2025_DMNews}}.
}
\label{fig:andromeda_center}
\end{figure}



\medskip
\noindent
Man kann das sogar ausrechnen:
\[
\Gamma \propto \rho_\star^2,
\]
die Rate der Ereignisse wächst mit dem Quadrat der Sterndichte.  
Wir haben gezeigt, dass dieser Effekt den „Exzess“ um Faktoren von \(10^2\text{–}10^3\)
verstärkt — exakt das, was die Satelliten sehen.  
\subsection*{Supercomputer gegen Taschenrechner}

Um diesen simplen Zusammenhang zu „bestätigen“,  
führten Muru et al.\cite{Muru2025} auf dem SuperMUC-Cluster
eine Simulation mit 8192$^3$ Teilchen durch.  
Das Ziel: den Gammastrahlen-Exzess zu erklären –  
durch selbstzerstörende Dunkle Materie.
\newline
Ergebnis: ein leicht elliptisches Leuchten im Zentrum,
mit Achsenverhältnis \(b/a \approx 0.7\).  
Unsere einfache Rechnung mit klassischer Gravitation, Einstein’scher Krümmung und endlicher Raumzeitwirkung?  
\emph{Achsenverhältnis \(b/a \approx 0.7\).}  
Gleiche Form, gleiche Skala – null Dunkelheit. \cite{Zander2025_SuperMUC}

\begin{quote}
\centering
\textit{
SuperMUC rechnete Wochen.\\
Mein Browser fünf Sekunden.}
\end{quote}
\clearpage

\subsection*{Wenn Simulation zum Selbstzweck wird}

Natürlich wird weiter simuliert.  
Mit neuen Parametern, noch feineren Gittern und noch dichteren Zitaten.  
Das $\sigma_P$-Modell arbeitet mit \(6/6\) Quellen;  
die moderne Dunkle-Materie-Forschung mit \(57/9\) pro Paper.  
Bibliometrisch brillant – physikalisch redundant.

\begin{quote}
\centering
\textit{
Man kann Milliarden Rechenoperationen investieren,\\
um zu zeigen, was Newton schon wusste: Gravitation zieht an.}
\end{quote}

\subsection*{Was bleibt}

Am Ende bleibt eine einfache Erkenntnis:
Das Zentrum der Milchstraße leuchtet heller,
weil dort mehr Sterne sind.
Die Dichte steigt, die Ereignisrate steigt, die Strahlung steigt.  
Das nennt man Physik, nicht Mystik.  
Wer dafür Dunkle Materie braucht,
braucht vielleicht nur helleres Licht.

\begin{quote}
\centering
\textit{
Dunkle Materie erklärt, warum Galaxien existieren sollten.\\
Die Sterne erklären, warum sie wirklich existieren.}
\end{quote}

\medskip
\noindent
\textbf{Fazit:}  
Alles, was die Dunkle Materie erklären sollte,  
ergibt sich aus endlicher Geometrie, baryonischer Masse  
und einfacher Gravitation.  
Der Rest ist Rhetorik – und Rechenzeit.

\begin{tcolorbox}[colback=gray!4,colframe=black!40!white,title=\textbf{Wie man Dunkle Materie erfindet – in drei einfachen Schritten},sharp corners,enhanced]
\begin{enumerate}
  \item \textbf{Beobachte etwas, das du nicht verstehst.}\\
  Zum Beispiel, dass Sterne in Galaxien zu schnell rotieren oder dass es im Zentrum der Milchstraße heller ist, als erwartet.  
  (Tipp: Je spektakulärer, desto besser.)

  \item \textbf{Erfinde etwas Unsichtbares, das alles erklärt.}\\
  Gib ihm einen Namen mit „dunkel“ – das klingt mysteriös und modern.  
  Dunkle Materie, Dunkle Energie, Dunkle Logik – alles geht.  
  Wichtig: Es darf weder messbar noch überprüfbar sein.

  \item \textbf{Simuliere es mit möglichst vielen Parametern.}\\
  Starte ein HPC-Projekt, füttere 8192$^3$ Teilchen in deinen Code,  
  und publiziere ein Paper mit 60 Zitaten und 20 Autoren.  
  Wenn die Ergebnisse dem Erwarteten widersprechen: erhöhe die Auflösung.
\end{enumerate}

\begin{center}
\textit{Glückwunsch! Du hast soeben das Unsichtbare sichtbar gemacht – auf Kosten der Steuerzahler.}
\end{center}
\end{tcolorbox}

\subsection*{Für alle, die es lieber einfach wollen}

Stell dir vor, die Milchstraße ist eine Stadt.  
In der Innenstadt – also im galaktischen Zentrum – leben die meisten Menschen (Sterne).  
Draußen am Rand (der galaktische Diskus) ist es viel leerer.  
Jetzt zähl mal, wie oft irgendwo ein Feuerwerk stattfindet.  
In der Innenstadt kracht es ständig, weil dort viele Menschen eng beieinander wohnen.  
Draußen auf dem Land passiert fast nie etwas.
\newline
Das ist die ganze Geschichte des sogenannten \emph{Gamma-Ray Excess}.
\newline
Im Zentrum der Milchstraße gibt es ungefähr 200-mal mehr Sterne pro Raumvolumen als im Rest.  
Wenn zwei Sterne oder kompakte Objekte manchmal zusammenstoßen oder verschmelzen,  
dann steigt die Häufigkeit solcher Ereignisse mit dem \emph{Quadrat} der Dichte –  
weil du bei doppelter Bevölkerungsdichte viermal so viele Nachbarn hast,  
die dich theoretisch anrempeln können.
\clearpage
Das ergibt:
\[
\text{Aktivität im Zentrum} \;\approx\; (150^2\text{–}200^2)\times 0.01 \;\approx\; 200\text{–}400\times \text{mehr als im Rest.}
\]

Also:  
Wenn draußen auf dem galaktischen Land alle paar Jahrzehnte mal ein Gamma-Blitz passiert,  
dann gibt’s im Zentrum einen alle 25–50 Jahre – völlig normal.  
Oder, wie man’s einfacher sagen kann:

\begin{quote}
\centering
\textit{Je mehr Sterne du stapelst, desto öfter knallt’s.}
\end{quote}

Das ist alles, was die aufwendige Supercomputer-Simulation von Muru et al.\ gezeigt hat –  
nur eben mit 8192$^3$ Teilchen und 10 Millionen CPU-Stunden.  
Aber das Ergebnis bleibt das gleiche, das man auf einem Bierdeckel bekommt:  
Das galaktische Zentrum ist heller, weil es voller ist.  
Das ist keine „selbstzerstörende Dunkle Materie“ –  
das ist schlicht ein überfüllter kosmischer Großstadtverkehr.


\section{Schlussfolgerung: A Requiem for $\Lambda$CDM — Danke, JWST}

Das Standardmodell der Kosmologie war kein Irrtum, sondern eine Zwischenetappe.  
Ein Versuch, Ordnung in ein Universum zu bringen, das größer, älter und subtiler ist, als unsere damaligen Gleichungen erlaubten.  
ΛCDM funktionierte — solange wir nur sahen, was es erklären konnte.
\newline
Doch dann kam das \emph{James Webb Space Telescope}.  
Ein Instrument von beispielloser Präzision, gebaut, um ΛCDM zu bestätigen –  
und das stattdessen zeigte, dass die Theorie ihre eigene Beobachtung überlebt hat.  
Es fand Galaxien, die zu früh zu groß waren, Schwarze Löcher, die zu schnell wuchsen,  
und Strukturen, die mehr an Reife als an Anfänge erinnerten.  
Es hielt dem Kosmos den Spiegel vor – und der Spiegel sprach zurück.

\begin{quote}
\centering
\textit{
JWST hat nicht das Universum verändert,\\
sondern unser Vertrauen in einfache Erklärungen.}
\end{quote}

Was also, wenn die Fehler gar nicht im All lagen,  
sondern in unserer mathematischen Sehschärfe?  
Einstein, Planck, Heisenberg und Minkowski haben uns die Werkzeuge gegeben –  
doch zu ihrer Zeit fehlten die experimentellen und rechnerischen Mittel,  
um ihre Ideen konsequent zusammenzudenken.  
Wenn man ihre Gleichungen heute vollständig ernst nimmt –  
mit endlicher Raumzeit, mit Wirkungsquanten, mit Kausalität als Geometrie –  
dann verschwinden all jene Paradoxien,  
die das 20. Jahrhundert für unvermeidlich hielt.  
Keine Dunkle Energie. Keine Dunkle Materie.  
Nur die Konsequenz der Endlichkeit.
\newline
Das ist keine neue Theorie, sondern die Vollendung einer alten.  
Nicht σ$_P$ ersetzt ΛCDM –  
sondern die Rückkehr zu dem, was Einstein selbst gemeint hatte:  
dass Geometrie und Energie zwei Seiten derselben Wirklichkeit sind.
σ$_P$ ist kein Modell, es ist reine Dimensionsanalyse. 
Die nummerischen Werte sind nicht von belang. Stattdessen bestimmen die gemessenen Werte der Dimensionen, mit welcher Stärke sich die Konstanten gegenseitig beeinflußen.
Die durch SI festgelegten Werte der Konstanten gelten nur als konventioneller Richtwert ihm σ$_P$-Rahmen. Jedoch nicht als absolut.   




\clearpage

\printbibliography[title={Literaturverzeichnis}]





\end{document}


