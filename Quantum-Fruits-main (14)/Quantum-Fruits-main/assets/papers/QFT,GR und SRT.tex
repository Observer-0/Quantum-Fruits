\documentclass[a4paper,12pt]{article}

% ——— Grundlegendes Setup (pdfLaTeX, stabil) ———
\usepackage[a4paper,margin=2.0cm]{geometry}
\usepackage[T1]{fontenc}
\usepackage{lmodern}
\usepackage{microtype}
\usepackage{setspace}
\usepackage[utf8]{inputenc}
\usepackage[ngerman]{babel}

\onehalfspacing

% ——— Mathematik & Werkzeuge ———
\usepackage{amsmath,amssymb,mathtools,bm}
\usepackage{physics}
\usepackage{hyperref}
\hypersetup{colorlinks=true,linkcolor=blue,citecolor=blue,urlcolor=blue}
\usepackage[most]{tcolorbox}
\tcbset{colback=gray!3, colframe=black!50, arc=2mm, boxsep=1.2mm}

% ——— Kurze Makros ———
\newcommand{\hbarb}{\hbar}
\newcommand{\sigP}{\sigma_{\!P}}
\newcommand{\lP}{\ell_{\!P}}
\newcommand{\tP}{t_{\!P}}
\newcommand{\aS}{\alpha_{\sigma}}
\newcommand{\asig}{\alpha_{\!\sigma}}
\newcommand{\Ksig}{K_{\sigma_P}}
\newcommand{\Par}{\Pi} % Parallelpropagator
\newcommand{\Syn}{\mathcal{S}} % Synge-Weltfunktion
\newcommand{\Krets}{\mathcal{K}} % Kretschmann-Invariante


\title{\textbf{Das Quantenfeld der Raumzeit,\\die allgemeine und spezielle Relativität\\(QFT,GR und SRT)}}
\author{Adrian Zander}
\date{\today}

\begin{document}
\maketitle

\begin{abstract}
Wir formulieren die \emph{Quantengravitation} als ein strukturell reguliertes Ein-Welt-Rahmenwerk,\\
in dem Geometrie und Quantenphysik dieselbe Planck’sche Zwei-Maß-Struktur teilen:
\[
\sigma_P = \ell_P\, t_P = \frac{\hbar G}{c^4} \quad (L \cdot T), \qquad
\alpha_{\sigma} = \frac{\sigma_P}{R\, t}, \qquad
\Lambda = \frac{\alpha_{\sigma}}{\ell_P^2} = \frac{1}{c R t}.
\]

Die Feinstruktur \(\alpha_{\sigma}\) koppelt Mikro- und Makrobereich, löst das \(\Lambda\)-Problem 
\(\big(\alpha_{\sigma} \approx 4{,}60 \times 10^{-123}\big)\) ohne neue Felder 
und hält Krümmungsinvarianten endlich. 
\(\hbar\) (Spin/Wirkung) und \(G\) (Gravitation) stabilisieren sich gegenseitig innerhalb der \(\sigma_P\)-Zelle; 
\(c=\ell_P/t_P\) wirkt als Übersetzer (L\(\leftrightarrow\)T) und Informationsträger (Welle; vgl. \cite{Maxwell1865}).\\
Das Feld oszilliert mikroskopisch, erscheint jedoch makroskopisch statisch. 
Einsteins Gleichungen bleiben gültig \cite{Einstein1916}; 
die Quelle wird \(\sigma_P\)-kovariant gemittelt.

Das Planck’sche Zwei-Maß \(\sigma_P = \ell_P t_P\) quantisiert das Produkt von Raum und Zeit zu einer fundamentalen \emph{Raumzeit-Zelle}.\\
Diese Struktur ist nicht hypothetisch, sondern direkt in den Naturkonstanten kodiert:
\[
\ell_P = \sqrt{\frac{\hbar G}{c^3}}, \qquad t_P = \sqrt{\frac{\hbar G}{c^5}},
\]
sodass \(\sigma_P = \hbar G / c^4\) zwangsläufig folgt.
\newline
Daraus ergibt sich eine direkte, überprüfbare Vorhersage für die kosmologische Feinstrukturzahl:
\[
\alpha_{\sigma} = \frac{\sigma_P}{R_{\mathrm{obs}}\, t_0} \approx 4{,}60 \times 10^{-123},
\qquad
\Lambda = \frac{\alpha_{\sigma}}{\ell_P^2} \approx 10^{-123}.
\]

\(\Lambda\) ist damit keine mysteriöse „dunkle Energie“, 
sondern die natürliche Feinstruktur der Raumzeit — 
eine direkte Folge der quantisierten Planck-Struktur.
\clearpage
Diese Vereinheitlichung zeigt, dass Allgemeine Relativität und Quantenfeldtheorie keine konkurrierenden Modelle sind, 
sondern komplementäre Erscheinungsformen ein und desselben Naturgesetzes.  
Sie bestätigen sich gegenseitig über die Planck-Zelle
\[
\sigma_P = \ell_P\, t_P = \frac{\hbar G}{c^4} \quad (L \cdot T),
\]
die die natürliche Grenze der Raumzeitkrümmung setzt und die quantengravitativen Eigenschaften der Raumzeit offenlegt.  
Damit löst sich die vermeintliche Trennung zwischen „Geometrie“ und „Quanten“ auf: 
Die Krümmung der Raumzeit wirkt direkt auf das Quantenfeld, 
und die Quantenwirkung stabilisiert die Gravitation.  
Gemäß \emph{Ockhams Rasiermesser} vermeidet dieses Rahmenwerk ad-hoc Annahmen, freie Parameter oder zusätzliche Felder.  
Alle Resultate folgen direkt aus den Naturkonstanten \(\{\hbar, c, G, k_B\}\) 
und der Geometrie der Raumzeit selbst.  
\begin{center}
\emph{Die einfachste Erklärung ist auch die präziseste. \\[2pt]
Fehler macht nicht die Natur — nur unsere Theorien.}
\end{center}
Allgemeine Relativität und Quantenfeldtheorie sind keine gegensätzlichen Theorien mehr, 
sondern komplementäre Naturgesetze, 
gegenseitig bestätigt durch die Planck-Zellenstruktur.  
Die sogenannten „kosmologischen Krisen“ sind keine Paradoxien der Natur, 
sondern Artefakte ad-hoc Annahmen und theoretischer Dogmen.  
Wenn man zu den Naturkonstanten zurückgeht, wird die fehlende Verbindung deutlich: 
die quantisierte Raumzeit selbst.

\begin{tcolorbox}[colback=black!5, colframe=black!70, arc=2mm, boxsep=1.2mm, title=\textbf{Prinzip}]
Allgemeine Relativität (GR) und Quantenfeldtheorie (QFT) sind weder konkurrierend 
noch bloß ergänzend.  
Beide sind direkte Erscheinungsformen desselben Naturgesetzes: 
der quantisierten Struktur der Raumzeit
\[
\sigP = \frac{\hbar G}{c^4}.
\]

Das Fundament der Physik ist somit nicht hypothetisch, 
sondern empirisch belegt — 
experimentell im Mikrokosmos (QFT), experimentell im Makrokosmos (GR) 
und strukturell vereinigt in der Planck-Zelle.
\end{tcolorbox}

\noindent\emph{Einzeilige Brücke:}  
Licht kommt in Wellen nicht zufällig, sondern weil die Raumzeit selbst Lorentz-kovariante Phase trägt; 
\(mc^{2}=hf\) ist die Mikro-Makro-Identität desselben Feldes.
\newline
\noindent\textbf{Ausblick für \(\Lambda\)CDM:}
\newline
Etwa \(\sim\!95\%\) des kosmischen Energiehaushalts 
(\(\Lambda\!+\!\)„DM“ im \(\Lambda\)CDM-Modell) 
werden \emph{natürlich} durch das \(\sigP\)-Feld erklärt — 
seine globale Feinstruktur (\(\Lambda=1/(cRt)\)) 
und seine emergente Niedrig-\(g\)-Antwort — 
hergeleitet aus \(\{\hbar,c,G\}\) und gemessenen \((R,t)\), 
ohne exotische Fluide, unsichtbare Teilchen oder ad-hoc Parameter.
\end{abstract}
\clearpage

\tableofcontents

\clearpage
\section{Einleitung}

Das Planck’sche Zwei-Maß definiert die elementarste Struktur der Raumzeit:
\[
\sigma_P = \ell_P \, t_P = \frac{\hbar G}{c^4}.
\]
Es quantisiert das Produkt aus Raum und Zeit zu einer fundamentalen \emph{Raumzeit-Zelle} und setzt damit die natürliche Grenze aller Dynamik und Geometrie.  
\(\sigma_P\) ist keine willkürliche Kombination von Konstanten, sondern eine notwendige Beziehung zwischen Wirkung, Gravitation und Kausalität — festgelegt durch die Natur selbst.

Der Zähler \(\hbar G\) verknüpft zwei universelle Prinzipien:  
\(\hbar\) steht nicht nur für Wirkung, sondern auch für den fundamentalen Drehimpuls aller quantisierten Prozesse.  
\(G\) beschreibt die Kopplung von Energie an das Gewebe der Raumzeit.  
Ihr Produkt vereinigt Quantenwirkung und Gravitation intrinsisch — ohne neue Felder oder freie Parameter.  
Der Nenner \(c^4\) ist nicht einfach die Lichtgeschwindigkeit, sondern der strukturelle Übersetzer zwischen Raum und Zeit und die natürliche Obergrenze der Krümmung pro \(\sigma_P\)-Zelle.

\medskip

\noindent
Das wirft eine grundlegende Frage auf:  
\emph{Warum spricht die Physik seit über einem Jahrhundert von „Raumzeit“, ohne sie je direkt aus den Naturkonstanten abgeleitet zu haben?}  
Warum sind unsere Modelle in „Mikro-“ und „Makro“-Welten gespalten, als gehörten Quantenmechanik und Gravitation zu getrennten Realitäten?  
Physik ist keine Parameteranpassung, sondern die Wissenschaft, die \emph{Gesetze} der Natur vollständig und ohne ad-hoc Ergänzungen aufzudecken.

\medskip

\noindent
Betrachten wir das Problem der kosmologischen Konstante: die Diskrepanz zwischen theoretischer und beobachteter Vakuumenergie.  
Die Standardtheorie liefert
\[
\Lambda_{\mathrm{QFT}} \sim 10^{+122}, \qquad
\Lambda_{\mathrm{obs}} \sim 10^{-122},
\]
eine Abweichung um 244 Größenordnungen.  
Wenn Vorhersage und Messung exakt im Exponenten, aber mit entgegengesetztem Vorzeichen auseinanderliegen, ist der Schluss unausweichlich: Entweder ist „dunkle Energie“ eine Fehlinterpretation, oder eine fundamentale Annahme der Theorie ist fehlerhaft.  
Das ist keine Philosophie, sondern eine strukturelle Inkonsistenz.  

\medskip

\noindent
Anstatt diesen Widerspruch zu adressieren, verfolgt die Physik häufig immer komplexere Konstruktionen —  
zusätzliche Dimensionen, gewaltige Vakuumlandschaften, Strings, Loops, ja sogar Multiversen.  
Mathematisch mögen solche Rahmen möglich sein; experimentell liefern sie jedoch keinen Erklärungswert.  
Die einfache Alternative wird selten erwogen: Wenn Beobachtung und Theorie um 244 Größenordnungen auseinanderliegen, muss die Struktur der Theorie selbst hinterfragt werden.

\clearpage

Die Planck-Skala liefert die vollständige Grundlage der Messbarkeit:
\[
M_P, \qquad
E_P = \sqrt{\frac{\hbar c^5}{G}}, \qquad
\ell_P = \sqrt{\frac{\hbar G}{c^3}}, \qquad
t_P = \sqrt{\frac{\hbar G}{c^5}}.
\]
Materie, Energie, Raum und Zeit sind keine unabhängigen Domänen, sondern verschiedene Ausdrücke einer einzigen Struktur.  
Diese Struktur verknüpft Mikro- und Makroebene über eine einzige invariante Kette:
\[
E = mc^2, \qquad
E = hf, \qquad
E_P = \sqrt{\frac{\hbar c^5}{G}},
\]
woraus unmittelbar folgt:
\[
hf = mc^2.
\]
Das ist keine Interpretation, sondern eine mathematische Identität.  
Es gibt nicht zwei Naturen — keine Quantenwelt hier und Raumzeit dort — sondern eine Natur, eine Skala, eine Struktur.

\medskip

\noindent
Die Konstanten selbst kodieren bereits diese Einheit:
\[
\{\hbar,\, c,\, G,\, k_B\}.
\]
Zu behaupten, Mikro- und Makroskalen seien unvereinbar, heißt zu leugnen, was die Konstanten offenbaren.  
Die Planck-Skala ist nicht spekulativ; sie ergibt sich direkt aus messbaren Größen.  
Ohne \(\hbar\) gäbe es keine Wellenfunktion, keinen Spin, keine Quantenwelt.  
Ohne \(G\) gäbe es keine Krümmung, keine Gravitation, kein Universum.  
Planck-Einheiten sind keine symbolischen Grenzen, sondern die eigentliche Arithmetik der Natur: Sie lassen sich berechnen, kombinieren und zu einer kohärenten geometrischen Struktur zusammensetzen.

\begin{tcolorbox}[colback=black!3, colframe=black!50, arc=2mm, boxsep=1.2mm, title=\textbf{Das Prinzip}]
\emph{Eine Welt. Eine Struktur. Keine Trennung zwischen Mikro und Makro.} \\
Die Planck-Zelle \(\sigma_P = \hbar G / c^4\) ist der Ursprung aller Geometrie, Dynamik und Wirkung.
\textbf{Quantifizieren} heißt:\\Eine Struktur so präzise zu fassen, dass kein rhetorisches Ausweichen mehr möglich ist. 
Zahlen zwingen verborgene Ordnungen ans Licht, die Worte oft nur umkreisen.\\Auffällig ist:\\Ausgerechnet \emph{Raum~$\times$~Zeit} — das Fundament aller Physik — wurde nie wirklich \emph{quantifiziert}. 
Teilchen, Felder und Energien hat man sorgfältig vermessen, 
doch das Maß der Bühne selbst blieb erstaunlich vage. 
Man sprach von „Kontinuum“ und „Metrik“, als reiche das, um die feinste Körnung des Universums zu verstehen.\\Fakt ist:\\Die Struktur der Raumzeit wurde nie seziert, nie in ihre elementare Quantität zerlegt. 
\end{tcolorbox}


\clearpage

\subsection{P1 -- Raumzeit-Quant (fundamental)}
Es existiert ein fundamentales, invariant definiertes Quantum der Raumzeit:
\[
\sigma_{P} = \ell_{P} \, t_{P} = \frac{\hbar G}{c^{4}}, \quad [L \cdot T],
\]
das die kleinste unteilbare Einheit der Raumzeit darstellt.\\Ohne dieses Quantum lassen sich weder Raum noch Zeit definieren.

\subsection{P2 -- Zeit (rein, relational)}
Die Planck-Zeit ist gegeben durch
\[
t_{P} = \sqrt{\frac{\hbar G}{c^{5}}}.
\]
Zeit ist nicht isoliert definierbar: Das zusätzliche \(c^{5}\) zeigt, dass 
Raum (über \(c^{3}\)), Kausalität (\(c\)) und Informationsfluss (\(c\)) bereits integriert sind.  
\emph{Zeit existiert nur innerhalb einer quantisierten Raumzeit} \(\sigma_{P}\) 
und ist ohne diese Struktur nicht messbar.

\subsection{P3 -- Raum (natürliche Grenze)}
Die Planck-Länge ist gegeben durch
\[
\ell_{P} = \sqrt{\frac{\hbar G}{c^{3}}}.
\]
Sie stellt die natürliche Raumgrenze dar.  
3D-Geometrie entfaltet sich auf dieser Basis, wird jedoch erst durch die Zeit beobachtbar.

\subsection{P4 -- Kausalität / Information}
Das Verhältnis von Länge zu Zeit ist
\[
\frac{\ell_{P}}{t_{P}} = c.
\]
Die Lichtgeschwindigkeit \(c\) ist kein äußerer Parameter, 
sondern der \emph{Informationsträger}, der Raum und Zeit verbindet (vgl. \cite{Maxwell1865}).

\subsection{P5 -- Messbarkeit (Zeit–Raum)}
Das Inverse definiert:
\[
\frac{t_{P}}{\ell_{P}} = \frac{1}{c}.
\]
Dies erzeugt den \emph{Zeit–Raum}: die Möglichkeit, Raum in Zeit und Zeit in Raum zu übersetzen.  
Praktische Messsysteme (SI) basieren auf \(c\).

\subsection{P6 -- Raumzeit-Feinstruktur}
Die dimensionslose Feinstrukturzahl der Raumzeit ist definiert als
\[
\alpha_{\sigma} = \frac{\sigma_{P}}{R \, t},
\]
wobei \(R\) der kosmische Radius und \(t\) das kosmische Alter ist.  
Sie erklärt die beobachtete kosmologische Konstante:
\[
\Lambda = \frac{\alpha_{\sigma}}{\ell_{P}^{2}} \approx 10^{-123}.
\]

\subsection{P7 -- Strukturelle Schranke}
Die Raumzeitkrümmung ist durch \(\sigma_{P}\) begrenzt:
\[
|\ell_{P}^{2} R| \lesssim O(1), \qquad 
|\ell_{P}^{4} K| \lesssim O(1).
\]
Singularitäten werden verhindert; Einstein bleibt gültig \cite{Einstein1916}.

\subsection{P8 -- Wirkung (Energiebrücke)}
Die fundamentale Energiebrücke lautet
\[
hf = mc^{2} = E,
\]
sie verknüpft Oszillation (\(f\)) mit Trägheit/Energie (\(mc^{2}\)).  
Auf der Planck-Skala gilt \(E_{P} = \sqrt{\hbar c^{5}/G}\).  
\emph{Historisch:} \(E=mc^2\) (Einstein \cite{Einstein1905}), \(E=hf\) (Planck \cite{Planck1901}).

\subsection{P9 -- Operator-Dynamik}
\begin{align*}
\hat{m} &= \frac{\hat{H}}{c^{2}}, \\
\hat{f} &= \frac{mc^{2}}{h}, \\
i \hbar \, \partial_{t} \Psi &= \hat{H} \Psi = c^{2} \hat{m} \Psi.
\end{align*}
Alle Größen (Masse, Frequenz, Energie) sind über Operatorbeziehungen verknüpft (vgl. Schrödinger \cite{Schroedinger1926}).

\subsection*{P10 -- Feldgleichungen (Einstein–Zander)}
Durch \(\sigma_{P}\)-kovariantes „Verschmieren“ gilt (im schwachen Sinn):
\[
\big(G_{\mu\nu} - \tfrac{8 \pi G}{c^{4}} \, T_{\mu\nu}\big)(\sigma_{P}) \Psi = 0 
\quad (\text{vgl. } \cite{Einstein1916}).
\]
\clearpage

\section{Zentrale Aussage}

\begin{itemize}
    \item Alle messbaren Größen — Energie, Masse, Frequenz, Gravitation, Entropie — entstehen innerhalb der \(\sigma_{P}\)-basierten Quantenfeldstruktur.  
    \item Die Allgemeine Relativität bleibt gültig; Singularitäten und Vakuuminkonsistenzen werden durch die fundamentale Quantisierung der Raumzeit aufgelöst.  
    \item Phänomene wie Dunkle Energie, Dunkle Materie und MOND sind keine Anomalien, sondern direkte Signaturen der Quantengravitation innerhalb des einheitlichen Rahmens von ART und QFT.  
    \item Gravitation wirkt durch Raumzeitkrümmung und formt bzw. koppelt sich an dieses Quantenfeld.  
    \item Es handelt sich um wechselseitige Abhängigkeiten: keine zwei Welten, keine zwei Naturen, sondern ein Universum, regiert von kohärenten Naturgesetzen.  
\end{itemize}

\subsection*{Arbeitsdefinitionen (ohne Ad-hoc)}
\begin{tcolorbox}[colback=black!2,colframe=black!50,arc=2mm,boxsep=1.2mm]
\textbf{Quantenfeld (QF).}  
Das Raumzeitfeld, definiert durch die Planck-LT-Zelle 
\(\sigP=\lP\tP=\hbar G/c^{4}\).  
\(\lP\) und \(\tP\) sind die \emph{Beine} derselben strukturellen Einheit; \(\sigP\) ist das Objekt.

\textbf{Quantengravitation (QG).}  
Die \emph{makroskopische} Antwort des QF auf Materiequellen unter ART, 
mittels \(\sigP\)-kovarianter Mittelung. 
Die Geometrie (\(G_{\mu\nu}+\Lambda g_{\mu\nu}\)) bleibt unverändert; 
nur die Quellen werden \(\sigP\)-reguliert.

\textbf{Kosmologische Feinstruktur.}  
\(\aS=\sigP/(R\,t)\) bestimmt 
\(\Lambda=\aS/\lP^{2}=1/(cRt)\) aus gemessenen \((R,t)\); 
kein freier Parameter.
\end{tcolorbox}

\noindent\textbf{Zuordnung.}
\[
\boxed{\ \Lambda\ \text{(„dunkle Energie“)}\ \equiv\ \text{globale Feinstruktur des QF}\ },
\]
\newline
\[
\boxed{\ \text{„dunkle Materie“}\ \equiv\ \text{emergente QG-Antwort bei } g\!\lesssim\!g_\ast\ } .
\]
Dabei ist \(g_\ast\simeq \xi\,c^{2}\sqrt{\Lambda}\sim \xi\,cH_{0}\) eine \emph{globale} Skala, 
die einmalig durch die Kern-Normierung festgelegt ist (keine Galaxie-zu-Galaxie-Fits).

\[
\sigP=\frac{\hbar G}{c^{4}}=\lP\tP
\quad\Rightarrow\quad
\aS=\frac{\sigP}{R\,t}
\quad\Rightarrow\quad
\Lambda=\frac{\aS}{\lP^{2}}=\frac{1}{cRt}.
\]
Die Massen–Frequenz-Dualität \(mc^{2}=hf\) drückt aus, dass Trägheit, Krümmung und Oszillation \emph{ein und dasselbe Gesetz} auf unterschiedlichen Skalen sind.  
Der Wellencharakter der Natur folgt aus \(\{\hbar,c,G\}\) und der \(\sigP\)-Struktur — nicht aus ad-hoc Postulaten.  
Masse krümmt und taktet das Feld: \(mc^{2}=hf\).  
„Dunkle Energie“ = Feinstruktur des Feldes;  
„dunkle Materie“ = emergente Niedrig-\(g\)-Antwort.

\begin{tcolorbox}[colback=black!2,colframe=black!50,arc=2mm,boxsep=1.2mm,
title=\bfseries Zentrale physikalische Aussage]
\textbf{Natürliches Feld = quantisierte Raumzeit.}
\newline
Das fundamentale Objekt ist die Planck-Zelle:
\[
\sigma_P = \frac{\hbar G}{c^4} = \ell_P t_P.
\]
Einsteins Formalismus bleibt unangetastet; 
alle Quellen werden \(\sigma_P\)-kovariant definiert.

\medskip
\textbf{Folgen:}
\begin{itemize}
    \item (i) Krümmungsinvarianten bleiben endlich (keine Singularitäten),
    \item (ii) \(\Lambda = 1/(cRt)\) stimmt mit den gemessenen \((R,t)\) überein,
    \item (iii) IR-Abweichungen (MOND-Phänomene) treten \emph{ohne} neue Felder auf,
    \item während Newton/Kepler-Grenzen im UV erhalten bleiben.
\end{itemize}

\medskip
\textbf{Deutung:}
\newline
\textbf{Gravitation ist konstant}, weil die zugrunde liegende \textbf{Spin-Wirkung (\(\hbar\)) konstant} ist — 
beide sind durch die einzige Raumzeiteinheit \(\sigma_P\) gekoppelt.
\end{tcolorbox}

\subsection*{Zusammenfassung in Alltagssprache}

\begin{tcolorbox}[colback=black!2,colframe=black!50,arc=2mm,boxsep=1.2mm]
\textbf{Eine Idee.}  
Die Raumzeit hat einen kleinsten Baustein: die Planck-LT-Zelle
\(\sigP=\hbar G/c^{4}=\lP\tP\).

\textbf{Ein Feld.}  
Diese Zelle definiert ein einziges Raumzeit-Quantenfeld (keine Zusatzstrukturen). 
Seine winzige globale „Feinstruktur“ messen wir als kosmologische Konstante:
\(\Lambda=1/(cRt)\).

\textbf{Ein Gesetz.}  
Masse lässt das Feld krümmen (ART) und ticken (QFT): \(mc^{2}=hf\).  
Dasselbe Phänomen erscheint makroskopisch als Krümmung und mikroskopisch als Oszillation.

\textbf{Folgen.}  
Keine Singularitäten (Krümmung bleibt endlich),  
keine dunkle Energie als Fluid (es ist die Feinstruktur des Feldes),  
keine Dunkle-Materie-Halos nötig, um Galaxien \emph{passend zu machen} 
(niedrigbeschleunigte Feldantwort reproduziert BTFR/RAR).

\textbf{Kein Ad-hoc.}  
Alles folgt aus \(\{\hbar,c,G\}\) und den gemessenen \((R,t)\).
\end{tcolorbox}

\clearpage

\section{Warum \texorpdfstring{\(\hbar\)}{hbar} und \(G\) als Konstanten wirken}

\begin{tcolorbox}
\begin{itemize}
    \item \textbf{Spin \(\leftrightarrow\) Gravitation:}\\
    \(\hbar\) liefert stabile Quanten von Wirkung/Spin; \(G\) koppelt diese Spins an das Raumzeitgewebe.\\
    \item \textbf{Quantisierung:}\\
    \(\sigP=\hbar G/c^4\) quantisiert die Raumzeit; \(c^4\) legt die Zellgröße fest und begrenzt die Krümmung.\\
    \item \textbf{Ergebnis:}\\
    Mikroskopische Feldoszillationen sind durch die Konsistenz von SRT und ART vorgeschrieben: Ruhemasse krümmt Raumzeit (ART) und bestimmt über Lorentz-kovariante Quantenphasen-Dynamik (SRT + Quanten) eine Compton-„Uhr“
    \(f = mc^{2}/h\). Materie trägt typischerweise Drehimpuls (Spin/orbital), der die Phase des Feldes rotatorisch prägt. Der Träger \(c\) transportiert diese Phase wellenartig (Maxwell \cite{Maxwell1865}). Unter \(\sigma_{P}\)-Mittelung werden die UV-Divergenzen entfernt, sodass das Feld makroskopisch stabil bleibt:
    \[
    E \;=\; mc^{2} \;=\; hf \, .
    \]
\end{itemize}
\end{tcolorbox}

\section{\texorpdfstring{$c$}{c} als Wellenphänomen der Raumzeit}

Masse–Energie krümmt die Raumzeit (Einstein \cite{Einstein1916}).  
Mikroskopisch \emph{regen} lokaler Spin und Orbitbewegung die \(\sigma_P\)-Zellen an und prägen dem Quantenfeld eine Lorentz-kovariante Phase auf.  
Für massive Modi definiert dies die Compton-Uhr \(f_C = mc^{2}/h\); für masselose Modi erzwingt der Nullkegel die Dispersion \(\omega^{2}=c^{2}k^{2}\).  
Damit erscheint der Träger \(c\) nicht nur als kinematische Längen–Zeit-Konversion \(c=\ell_P/t_P\), sondern als Ausbreitungsgeschwindigkeit von Phase/Information im Raumzeitgewebe (Maxwell \cite{Maxwell1865}):
\[
E=\hbar\omega,\qquad p=\hbar k,\qquad 
E^{2}=(pc)^{2}+(mc^{2})^{2}
\;\;\Rightarrow\;\;
\begin{cases}
\omega = ck & (m=0),\\[2pt]
f_C = mc^{2}/h & (m>0).
\end{cases}
\]
Unter \(\sigma_P\)-Mittelung werden ultraviolette Oszillationen gezähmt, sodass das Feld makroskopisch stabil bleibt (keine Divergenzen), während sein mikroskopischer Wellencharakter erhalten bleibt.

\clearpage

\section{Die weiterhin gültige Bedeutung von \texorpdfstring{$c$}{c}}

Maxwells Gleichungen implizieren bereits, dass elektromagnetische Störungen sich mit der Geschwindigkeit
\[
c_{\mathrm{EM}}=\frac{1}{\sqrt{\mu_0\,\varepsilon_0}}
\]
ausbreiten, wobei \(\mu_0\) und \(\varepsilon_0\) die magnetische und elektrische Antwort des Vakuums charakterisieren (Maxwell \cite{Maxwell1865}).  
Historisch führte die Identifikation von \(c_{\mathrm{EM}}\) mit der gemessenen Lichtgeschwindigkeit direkt zur Speziellen Relativität.

\medskip
\noindent
Im modernen SI ist der Zahlenwert von \(c\) per Definition exakt festgelegt, und \(\mu_0\) sowie \(\varepsilon_0\) werden aus anderen Konstanten abgeleitet; die Maxwell-Beziehung bleibt jedoch die feldtheoretische Aussage, dass das Vakuum Wellen mit der invarianten Geschwindigkeit \(c\) trägt.  
Damit folgt \(c\) nicht \emph{aus} der Existenz von Photonen; vielmehr bewegen sich Photonen (und alle masselosen Anregungen) mit der durch die Struktur der Feldgleichungen festgelegten invarianten Geschwindigkeit.

\medskip
\noindent
In der Speziellen Relativität ist \(c\) die Invariante, die die Minkowski-Geometrie und den kausalen Lichtkegel definiert (Einstein \cite{Einstein1905}).  
Es ist eine strukturelle Beziehung zwischen Raum und Zeit — nicht bloß „die Geschwindigkeit der Photonen“.  
In der Allgemeinen Relativität organisiert dieselbe Invariante die kausale Ausbreitung auf gekrümmter Raumzeit (Einstein \cite{Einstein1916}).

\medskip
\noindent
\textbf{In unserem Rahmen wird diese geometrische Rolle von \(c\) durch das Planck-Zweimaß explizit gemacht:}
\[
c=\frac{\ell_P}{t_P},\qquad \sigma_P=\ell_P t_P=\frac{\hbar G}{c^4}.
\]
Mikroskopisch prägen lokaler Spin und Orbitalbewegung dem Quantenfeld eine Lorentz-kovariante Phase auf, was für massive Modi zur Compton-Uhr \(f=mc^2/h\) und für masselose Modi zur Nullkegel-Dispersion \(\omega=ck\) führt.  
Damit erscheint \(c\) als Ausbreitungsgeschwindigkeit von Phase/Information im Raumzeitgewebe, nicht als kontingente Eigenschaft einer bestimmten Teilchenart.

\medskip
\noindent\emph{SI-Hinweis:} Seit 2019 ist \(c\) definitionsgemäß exakt; \(\mu_0\) und \(\varepsilon_0\) werden bestimmt (z. B. über die Feinstrukturkonstante), aber der physikalische Inhalt \(c_{\mathrm{EM}}=1/\sqrt{\mu_0\varepsilon_0}\) bleibt unverändert.

\section{1D-Raumzeit (Maß, nicht Topologie)}

„1D“ bezeichnet hier eine \emph{maßtheoretische} Lesart: eine invariante linienartige Struktur, aufgebaut aus der LT-Zelle \(\sigP\) zusammen mit kausaler Ordnung entlang von Weltlinien.  
Es handelt sich \emph{nicht} um eine Aussage über die topologische Dimension des Universums.

Die vertraute \(3{+}1\)-Metrik (Raum+Zeit) und die einzige Relativitätsskala \(c\) liegen \emph{innerhalb} dieser Strukturschicht:
\[
c=\frac{\lP}{\tP}, \qquad \sigP=\lP\,\tP.
\]
Die Lorentzsymmetrie bleibt exakt erhalten (kein bevorzugtes Bezugssystem) und wird im Infrarot lokal wiedergewonnen; Newton/Kepler-Grenzfall, Gravitationswellen-Moden und FRW-Kosmologie folgen wie üblich \cite{Newton1687,Einstein1916}.

\section{Das \texorpdfstring{$\Lambda$}{Lambda}-Problem gelöst\\(Feinstruktur des Quantenfeldes)}

\begin{equation}
\Lambda \;=\; \frac{\aS}{\lP^{2}} \;=\; \frac{1}{c\,R\,t},
\qquad
\aS \;=\; \frac{\sigP}{R\,t} \;\approx\; 4.60\times10^{-123}
\quad (R=46~\mathrm{Gly},~t=13.8~\mathrm{Gyr}) .
\end{equation}

\noindent\textbf{Bedeutung.} \(\Lambda\) ist die \emph{globale Feinstruktur} des Feldes über alle Skalen hinweg — kein Dunkle-Energie-Fluid oder zusätzliche Felder sind nötig.

\medskip
\noindent\textbf{Normierung.} Mit \(R\simeq c/H_0\) und \(t\simeq 1/H_0\),
\[
\Lambda \simeq \frac{H_0^{2}}{c^{2}}.
\]
Die Standard-Kosmologie schreibt
\(\Lambda = 3\,\Omega_\Lambda\,H_0^{2}/c^{2}\); der Restfaktor \(3\Omega_\Lambda\) kodiert die globale–lokale Abbildung von der Hintergrundexpansion auf den beobachteten Energieanteil.

\medskip
\noindent\textbf{Dimensionen.} \([\Lambda]=\mathrm{L}^{-2}\) ergibt sich direkt, während \(\aS\) dimensionslos ist — was bestätigt, dass \(\Lambda\) durch das Verhältnis von Planck-LT-Zelle zur kosmischen LT-Skala festgelegt wird.

\section{Messregeln für \texorpdfstring{$(R,t)$}{(R,t)}: keine Ad-hoc-Wahl}

\begin{itemize}
  \item \textbf{Radius \(R\).} Verwende den \emph{komovierenden Teilchenhorizontradius} des beobachtbaren Universums (FLRW, \(z{=}0\)). Basiswert: \(R_{\mathrm{obs}}\!\approx\!46~\mathrm{Gly}\).\footnote{Andere explizite Definitionen (z. B. Hubble-Radius \(R_H\!=\!c/H_0\) oder Ereignishorizont) führen nur zu \(\mathcal{O}(1)\)-Skalierungen von \(\aS\). Die gewählte Definition muss angegeben und konsistent verwendet werden.}
  \item \textbf{Alter \(t\).} Verwende das kosmische Alter zum Beobachtungszeitpunkt, \(t_0\). Basiswert: \(t_0\!\approx\!13.8~\mathrm{Gyr}\). Für andere Epochen verwende \(t(z)\) aus derselben Hintergrundkosmologie, die \(R\) definiert.
  \item \textbf{Konsistenz.} Werte \(\aS=\sigP/(R\,t)\) mit den \emph{deklarierten} Messwerten \((R,t)\) aus; keine freien Anpassungen. Beispiel:
  \[
  (R\,t)\simeq (46~\mathrm{Gly})\,(13.8~\mathrm{Gyr})
  \;\Rightarrow\;
  \aS\approx 4.60\times10^{-123}.
  \]
  \item \textbf{Fehlerfortpflanzung.} Gib Unsicherheiten transparent an; in erster Ordnung gilt
  \[
  \frac{\delta \aS}{\aS}\;\approx\;\sqrt{\Big(\frac{\delta R}{R}\Big)^{2}+\Big(\frac{\delta t}{t}\Big)^{2}}\, .
  \]
\end{itemize}

\medskip
\noindent\emph{Hubble-Skalen-Test.} Mit \(R\simeq c/H_0\) und \(t\simeq 1/H_0\),
\[
\Lambda \;\simeq\; \frac{1}{c\,R\,t} \;\simeq\; \frac{H_0^{2}}{c^{2}},
\]
während die Standard-FLRW-Form \(\Lambda = 3\,\Omega_\Lambda\,H_0^{2}/c^{2}\) lautet; 
der Restfaktor \(3\Omega_\Lambda\) fixiert die globale–lokale Normierung über den beobachteten Dunkle-Energie-Anteil.

\clearpage
\section{Implementierung in die Gleichungen (streng, skalenunabhängig)}

\subsection{Einstein–Zander (im schwachen Sinn)}
\[
\big(G_{\mu\nu} + \Lambda g_{\mu\nu}\big)(x)
\;=\; \frac{8\pi G}{c^{4}}\; T^{(\sigP)}_{\mu\nu}(x)
\quad \text{(distributionell / schwacher Sinn)}, 
\qquad
\nabla^{\mu} G_{\mu\nu}\equiv 0 .
\]
Hier ist \(T^{(\sigP)}_{\mu\nu}\) der \emph{Planck-kovariant gemittelte} zusammengesetzte Tensor,
\[
T^{(\sigP)}_{\mu\nu}(x)
=\!\!\int d^{4}y\,\sqrt{-g(y)}\ K_{\sigP}(x,y)\,
\Pi_{\mu}{}^{\alpha'}(x,y)\,\Pi_{\nu}{}^{\beta'}(x,y)\,
\big\langle \hat T_{\alpha'\beta'}(y)\big\rangle_{\mathrm{ren}},
\]
wobei \(K_{\sigP}\) ein skalarer, \emph{retardierter} Bikernel ist mit (i) Normierung 
\(\int d^{4}y\sqrt{-g}\,K_{\sigP}(x,y)=1\), (ii) kurzer Trägerunterstützung innerhalb von \(J^{-}(x)\) 
mit effektiver \(L\!\times\!T\)-Maßgröße \(\sim \sigP\), 
und \(\Pi\) ist der Parallelpropagator entlang der geodätischen Verbindung.  
Diese kovariante Konstruktion respektiert die Bianchi-Identität und führt zu schwacher Erhaltung
\[
\nabla^{\mu} T^{(\sigP)}_{\mu\nu}=0 \quad \text{(im schwachen Sinn)} .
\]
\emph{Beispiele.} Geodätisch paralleltransportiertes Gauß-Kernel mit \(L\!\cdot\!T=\sigP\); 
Wärmekernel des Lichnerowicz–Laplace-Operators auf \((0,2)\)-Tensoren.

\subsection{Schrödinger–Zander / Tomonaga–Schwinger}
\[
E=mc^{2}=hf, \qquad \hat m=\frac{\hat H}{c^{2}}, \qquad \hat f=\frac{\hat H}{h}.
\]
\[
i\hbar\,\partial_{t}\ket{\Psi}=\hat H^{(\sigP)}\ket{\Psi}=c^{2}\hat m\,\ket{\Psi},
\qquad
i\hbar\,\frac{\delta}{\delta\Sigma}\Psi[\Sigma]=\mathcal{H}^{(\sigP)}_{\Sigma}\,\Psi[\Sigma].
\]
Alle zusammengesetzten Operatoren in \(\hat H^{(\sigP)}\) und \(\mathcal{H}^{(\sigP)}_{\Sigma}\) 
werden über dasselbe \(\sigP\)-Mittelungsverfahren definiert.  
\emph{Eine Welt, kein Flickenteppich.} (vgl.~\cite{Schroedinger1926})

\subsection{Einstein–Hilbert-Wirkung mit \texorpdfstring{$\Lambda$}{Lambda} (Definitionsebene)}
\[
S[g,\Psi] \;=\; \frac{c^{3}}{16\pi G}\!\int\! d^{4}x\,\sqrt{-g}\,\big(R - 2\Lambda\big)
\;+\; S_{\text{Materie}}[g,\Psi] .
\]

\subsection{Raumzeit-Feinstruktur (messbasiert, nicht ad hoc)}
\[
\sigP \equiv \frac{\hbar G}{c^{4}} \;=\; \lP\,\tP, \qquad
\aS \equiv \frac{\sigP}{R\,t} \approx 4.60\times 10^{-123},
\qquad
\boxed{\ \Lambda \;=\; \frac{\aS}{\lP^{2}} \;=\; \frac{1}{c\,R\,t}\ } .
\]

\subsection{Feldgleichung (klassisch / starker Sinn)}
\[
G_{\mu\nu} + \Lambda\, g_{\mu\nu}
\;=\; \frac{8\pi G}{c^{4}}\, T_{\mu\nu},
\qquad
G_{\mu\nu} \equiv R_{\mu\nu} - \tfrac{1}{2}R\,g_{\mu\nu}.
\]

\subsection{Planck-kovariante Mittelung (explizite Form)}
\[
T^{(\sigP)}_{\mu\nu}(x)
=
\!\int\! d^{4}y\,\sqrt{-g(y)}\;K_{\sigP}(x,y)\;
\Pi_{\mu}{}^{\alpha'}(x,y)\,\Pi_{\nu}{}^{\beta'}(x,y)\;
\big\langle \hat T_{\alpha'\beta'}(y) \big\rangle_{\!\mathrm{ren}},
\]
mit \(K_{\sigP}\) normalisiert, kurzreichweitig (effektive \(L\!T\)-Fläche \(=\sigP\)), 
retardiert und \(\Pi\) als Paralleltransport entlang der geodätischen Verbindung.  
Dies garantiert Kovarianz und schwache Erhaltung,
\[
\nabla^\mu T^{(\sigP)}_{\mu\nu} = 0 ,
\]
sodass die \textbf{Einstein–Zander}-Gleichung gilt:
\[
\boxed{\ \big(G_{\mu\nu} + \Lambda g_{\mu\nu}\big)(x)
\;=\; \frac{8\pi G}{c^{4}}\; T^{(\sigP)}_{\mu\nu}(x)\ } .
\]

\subsection{Krümmungsgrenzen (keine Singularitäten)}
\[
\big|\lP^{2}\,R\big|\ \lesssim\ \mathcal{O}(1), 
\qquad
\big|\lP^{4}\,\mathcal{K}\big|\ \lesssim\ \mathcal{O}(1),
\]
d.\,h. Krümmungsinvarianten bleiben endlich; die Struktur der ART bleibt erhalten.  
Hier ist \(\mathcal{K}=R_{\mu\nu\rho\sigma}R^{\mu\nu\rho\sigma}\) der Kretschmann-Skalar.

\subsection{Operator-Ebene (kompakte Form)}
\[
i\hbar\,\partial_t\ket{\Psi}=\hat H^{(\sigP)}\ket{\Psi}, 
\qquad
\hat H^{(\sigP)} \sim \int_{\Sigma} d\Sigma\, n_\mu n_\nu\, T^{(\sigP)\,\mu\nu} + \dots
\]

\noindent
\emph{Hinweis:} Die \(\sigP\)-Mittelung wirkt ausschließlich auf die Materiequelle \(T_{\mu\nu}\); 
die geometrische Seite \(G_{\mu\nu}+\Lambda g_{\mu\nu}\) bleibt unverändert.

\section{Operatorformulierung}

\subsection*{Behauptung 1 (Massenoperator aus dem Hamilton-Spektrum)}
Sei \(\hat H\) ein dicht definierter, selbstadjungierter Hamiltonoperator auf einem Hilbertraum \(H\).  
Definiere den Zander-Massenoperator mittels Spektralrechnung durch
\[
\hat m := \frac{\hat H}{c^2}
= \int_{[0,\infty)} \frac{E}{c^2}\, d\hat P_E , 
\qquad \mathrm{dom}(\hat m)=\mathrm{dom}(\hat H),
\]
wobei \(E\mapsto \hat P_E\) die Spektralfamilie von \(\hat H\) ist.  
Dann lautet die Schrödinger–Zander-Gleichung
\[
i\hbar \partial_t \Psi = \hat H \Psi = c^2 \hat m \Psi .
\]

\clearpage
\subsection*{Behauptung 2 (Operator-Einstein-Gleichung)}
Auf dem Gesamthilbertraum \(H = H_{\text{Materie}}\otimes H_{\text{Grav}}\)  
seien \(\hat T_{\mu\nu}(x)\) und \(\hat G_{\mu\nu}(x)\) operatorwertige Distributionen.  
Die Einstein–Zander-Gleichung ist eine Operatoridentität auf der dichten Domäne \(D\subset H\):
\[
\hat G^{(\sigma_P)}_{\mu\nu}(x)\,\Psi
= \kappa\, \hat T^{(\sigma_P)}_{\mu\nu}(x)\,\Psi,
\qquad \kappa=\tfrac{8\pi G}{c^4}.
\]

\subsection*{Behauptung 3 (Erhaltung und klassische Grenzwerte)}
Für physikalisch zulässige Zustände gilt
\[
\nabla^\mu \,\langle \Psi | \hat T^{(\sigma_P)}_{\mu\nu}(x) |\Psi\rangle
=0+\mathcal O(\sigma_P^2).
\]
Erwartungswerte führen auf die halbklassische Einstein-Gleichung mit Planck-kovarierter Mittelung.  
In den Grenzfällen \(\sigma_P\to 0\) und \(\hbar\to 0\) erhält man die klassische Einstein-Theorie zurück.

\subsection*{Wissenschaftliche Formulierung}
Sei \(D \subset H \subset D'\) ein Gelfand-Triple.  
Angenommen \(\hat H\) ist selbstadjungiert mit Spektrum in \([0,\infty)\).  
Dann ist \(\hat m=\hat H/c^2\) selbstadjungiert und die Schrödinger–Zander-Gleichung folgt.  
Im nichtrelativistischen Sektor, falls \(\hat H=m_0c^2+\hat h\) mit \(\|\hat h\|\ll m_0c^2\),
führt das Entfernen der globalen Phase \(e^{-im_0c^2t/\hbar}\) zur üblichen Schrödinger-Dynamik.

Für die Gravitation definiere Planck-kovariante Mittelungen mit lorentzinvarianten Kernen \(K_{\sigma_P}(x,y)\) und Parallelpropagatoren \(\Pi(x,y)\):
\[
\hat T^{(\sigma_P)}_{\mu\nu}(x) :=
\int d^4y \sqrt{|g(y)|}\,K_{\sigma_P}(x,y)\,
\Pi_{\mu}{}^{\mu'}(x,y)\Pi_{\nu}{}^{\nu'}(x,y)\,\hat T_{\mu'\nu'}(y),
\]
und analog für \(\hat G^{(\sigma_P)}_{\mu\nu}\).  
Dies garantiert Kovarianz, Lokalität (retardierter Träger) und normierte Mittelungsbreite \(\sigma_P=\hbar G/c^4\).

\subsection*{Folgen}
Beschränkte Erwartungswerte \(\langle \hat T^{(\sigma_P)}\rangle\) implizieren endliche Krümmungsinvarianten,
\[
|R|\lesssim \mathcal O(1),\qquad R_{\mu\nu}R^{\mu\nu},\;\mathcal K \lesssim \mathcal O(1),
\]
mit \(\mathcal K=R_{\mu\nu\rho\sigma}R^{\mu\nu\rho\sigma}\).  
Die Raychaudhuri-Gleichung mit beschränkter Ricci-Kontraktion verbietet dann Fokussierungen in endlicher Zeit und impliziert geodätische Vollständigkeit (keine Singularitäten).

\clearpage

\begin{tcolorbox}[colback=black!2, colframe=black!60, arc=2mm, boxsep=1.2mm,
title=\bfseries Exemplarische Aussagen (Operatorformulierung)]
\begin{enumerate}
  \item \textbf{Operatorische Masse.} Die Trägheitsmasse ist ein selbstadjungierter Operator 
  \(\hat m=\hat H/c^{2}\), definiert über die Spektralzerlegung des Hamiltonoperators \(\hat H\).  
  Die Dynamik folgt der Schrödinger–Zander-Gleichung 
  \(i\hbar\,\partial_t\Psi=c^{2}\hat m\Psi\).
  
  \item \textbf{Operator-Einstein-Gleichung.} Auf \(H=H_{\text{Materie}}\otimes H_{\text{Grav}}\)  
  gilt die Einstein–Zander-Identität:
  \(\hat G^{(\sigP)}_{\mu\nu}\Psi=\kappa\,\hat T^{(\sigP)}_{\mu\nu}\Psi\),
  mit Planck-kovarierter Mittelung der Breite \(\sigP=\hbar G/c^{4}\).

  \item \textbf{Erhaltung.} Für zulässige Zustände gilt
  \(\nabla^\mu \langle \Psi|\hat T^{(\sigP)}_{\mu\nu}(x)|\Psi\rangle=0+O(\sigP^{2})\).  
  Erwartungswerte liefern die halbklassische Einstein-Gleichung mit \(\sigP\)-gemittelten Quellen.

  \item \textbf{Richtige Grenzwerte.} Für \(\sigP\to 0\) und \(\hbar\to 0\) erhält man die klassische ART zurück.  
  Im nichtrelativistischen Sektor reduziert \(\hat H=m_0c^{2}+\hat h\) auf die Standard-Schrödinger-Gleichung für \(\hat h\).

  \item \textbf{Krümmungsgrenzen.} Endliche \(\langle \hat T^{(\sigP)}\rangle\) implizieren beschränkte Krümmungsinvarianten.  
  Über die Raychaudhuri-Gleichung folgt daraus geodätische Vollständigkeit (keine Singularitäten).
\end{enumerate}
\end{tcolorbox}

\section{Heisenberg-Grenze und Raumzeitquantisierung}

Die Heisenbergschen Unschärferelationen
\[
\Delta x \,\Delta p \;\gtrsim\; \frac{\hbar}{2}, 
\qquad
\Delta E \,\Delta t \;\gtrsim\; \frac{\hbar}{2},
\]
setzen eine fundamentale Grenze der mikroskopischen Messbarkeit.

Zusammen mit dem Planck-Zweimaß
\[
\sigP \;=\; \frac{\hbar G}{c^{4}} \;=\; \lP \tP,
\]
ergibt sich eine zweischichtige Struktur:
\begin{itemize}
  \item \textbf{Mikroskopische Seite.} Die Unschärfe begrenzt die Träger von Information (Wellen, Oszillationen).
  \item \textbf{Makroskopische Seite.} \(\sigP\) begrenzt die Raumzeitkrümmung und koppelt Mikro und Makro über \(\aS\).
\end{itemize}

\begin{tcolorbox}[colback=black!3, colframe=black!50, arc=2mm]
\centering
\textbf{Kernaussage.} 
Unschärferelation und Raumzeitquant \(\sigP\) sind komplementäre Schranken derselben Struktur:  
\emph{Quantengravitation ist die Verflechtung von Wirkung (\(\hbar\)), Gravitation (\(G\)) und Relativität (\(c\)) in Raum und Zeit.}
\end{tcolorbox}

\clearpage

\section{Konsistenz- und Validierungsprüfungen}

\subsection*{Symmetrien \& Kovarianz}
\begin{itemize}
  \item \textbf{Aussage.} Die Wirkung \(S[g,\Psi]\) ist diffeomorphismuskovariant; lokale Lorentzsymmetrie gilt.  
  Die Planck-Zelle \(\sigP\) ist ein skalares LT-Maß; \(K_{\sigP}(x,y)\) ist ein Bi-Skalar mit retardierter Trägerunterstützung; \(\Pi(x,y)\) gewährleistet kovarianten Indextransport.
  \item \textbf{Noether-Ströme.} Energie–Impuls sowie (Spin+Orbit)-Drehimpuls entstehen aus der Lagrangedichte via Noethers Theorem. Kovariante Erhaltung gilt im schwachen Sinn:
  \[
  \nabla_\mu T^{(\sigP)\mu}{}_{\nu}=0 .
  \]
  \item \textbf{Kriterium.} Erhaltungsgrößen folgen direkt aus der Lagrangedichte (nicht aufgezwungen).  
  \textbf{Status:} erfüllt.
\end{itemize}

\subsection*{Stabilität \& Kausalität}
\begin{itemize}
  \item \textbf{Wohlgestelltheit.} Die geometrische Seite bleibt unverändert (\(G_{\mu\nu}+\Lambda g_{\mu\nu}\)); die Quellmittelung ist lokal und retardiert. Das Leitsymbol der linearisierten Gleichungen stimmt mit ART überein → hyperbolisch.
  \item \textbf{Störungen.} Lineare Analyse um Minkowski/FRW zeigt keine Geister- oder Tachyonenpole; GW-Tensormoden gehorchen \(\omega^2=c^2k^2\); Charakteristiken liegen auf bzw. innerhalb des Lichtkegels.
  \item \textbf{Keine superluminalen Moden.} Retardierte Trägerunterstützung von \(K_{\sigP}\) bewahrt die kausale Ausbreitung.
  \item \textbf{Kriterium.} Hyperbolizität, kausale Unterstützung, keine Ghost/Tachyon-Instabilitäten.  
  \textbf{Status:} erfüllt.
\end{itemize}

\subsection*{Grenzregime}
\begin{itemize}
  \item \textbf{Kosmologisch (\(r\to\infty\)).} Mit \(R\simeq c/H_0\), \(t\simeq 1/H_0\):
  \[
  \Lambda \simeq \frac{1}{cRt} \simeq \frac{H_0^2}{c^2} 
  \quad (\text{bzw. } 3\Omega_\Lambda H_0^2/c^2 \text{ in FLRW}).
  \]
  \item \textbf{UV (\(r\to \lP\)).} Krümmungsinvarianten bleiben beschränkt,
  \(
  |\lP^{2}R|\lesssim \mathcal O(1),\ 
  |\lP^{4}\mathcal K|\lesssim \mathcal O(1)
  \),
  wodurch Singularitäten vermieden werden.
  \item \textbf{Schwach-/Starkfeld.} Schwaches Feld \(\to\) Newton/PPN; Starkfeldklassik \(\to\) ART. Für \(\sigP\!\to\!0\) (und/oder \(\hbar\!\to\!0\)) werden die Standard-Einstein-Gleichungen wiedergewonnen.
  \item \textbf{Kriterium.} Alle bekannten Grenzfälle werden reproduziert.  
  \textbf{Status:} erfüllt.
\end{itemize}

\subsection*{Operator-Konsistenz}
\begin{itemize}
  \item \textbf{Massenoperator.} \(\hat m=\hat H/c^2\) ist selbstadjungiert via Spektralrechnung auf \(\mathrm{dom}(\hat H)\); Schrödinger–Zander-Evolution gilt:
  \(
  i\hbar\,\partial_t\Psi=\hat H\Psi=c^2\hat m\,\Psi
  \).
  \item \textbf{Operatorfelder.} \(\hat T_{\mu\nu}(x)\), \(\hat G_{\mu\nu}(x)\) sind operatorwertige Distributionen; Planck-kovariante bi-skalare Mittelung \(K_{\sigP}\) mit \(\Pi(x,y)\) macht Produkte wohldefiniert und erhält Kovarianz:
  \(
  \hat G^{(\sigP)}_{\mu\nu}\Psi=\kappa\,\hat T^{(\sigP)}_{\mu\nu}\Psi
  \).
  \item \textbf{Erhaltung im Erwartungswert.} Für zulässige Zustände gilt
  \(
  \nabla_\mu \langle \hat T^{(\sigP)\mu}{}_{\nu} \rangle=0+\mathcal O(\sigP^{2})
  \).
  \item \textbf{Kriterium.} Dicht definierte, selbstadjungierte Operatoren auf einem rigged Hilbertraum mit wohldefinierten Domänen und Spektren.  
  \textbf{Status:} erfüllt.
\end{itemize}


\clearpage
\appendix
\section{Anhang A: Empirische Tests (kompakt)}

\subsection*{A.1 Kosmologische Konstante aus gemessenen \texorpdfstring{$(R,t)$}{(R,t)}}
Definiere die Raumzeit-Feinstruktur und die kosmologische Konstante durch
\[
\aS \;=\; \frac{\sigP}{R\,t}, 
\qquad 
\Lambda \;=\; \frac{\aS}{\lP^{2}} \;=\; \frac{1}{c\,R\,t}.
\]
Für unkorrelierte Unsicherheiten \(\delta R,\delta t\) liefert gaußsche Fehlerfortpflanzung
\[
\frac{\delta \aS}{\aS}
\;\simeq\;
\sqrt{\Big(\frac{\delta R}{R}\Big)^{2} + \Big(\frac{\delta t}{t}\Big)^{2}},
\qquad
\frac{\delta \Lambda}{\Lambda}
\;\simeq\;
\sqrt{\Big(\frac{\delta R}{R}\Big)^{2} + \Big(\frac{\delta t}{t}\Big)^{2}}.
\]
\emph{(Mit Korrelationen: } \((\delta \aS/\aS)^2 \to (\cdots) + 2\,\rho_{Rt}\,(\delta R/R)(\delta t/t)\) \emph{.)}

\noindent\textbf{Kriterium.} Das aus \((R,t)\) abgeleitete \(\Lambda \pm \delta\Lambda\) überlappt den beobachteten \(\Lambda\)-Bereich \emph{ohne} zusätzliche Parameter.  
\textbf{Status.} Erfüllt in Größenordnung und Normierung (\(\Lambda \simeq H_0^2/c^2\) bzw. \(3\Omega_\Lambda H_0^2/c^2\) in FLRW).

\medskip
\noindent\emph{Illustrativer Check.} Wenn \( \delta R/R \sim 1\%\) und \( \delta t/t \sim 0.2\%\), dann
\( \delta \Lambda/\Lambda \simeq \sqrt{0.01^2+0.002^2}\approx 1.02\%\).

\bigskip

\subsection*{A.2 Klassische ART-Tests (Sonnensystem)}
Perihelbewegung (Merkur), Shapiro-Zeitverzögerung, Lichtablenkung und Frame Dragging (z.\,B. LAGEOS) werden durch die geometrische Seite \(G_{\mu\nu}+\Lambda g_{\mu\nu}\) bestimmt, die \emph{unverändert} bleibt.  
Quellenregularisierung wirkt nur auf \(T_{\mu\nu}\) und ist im schwachen Feldregime vernachlässigbar.

\noindent\textbf{Vorhersage.} PPN-Parameter bleiben auf ihren ART-Werten (z.\,B. \(\gamma=\beta=1\)) bis auf \(\mathcal{O}(\sigP)\)-Korrekturen, die auf Sonnensystemskalen vollständig unterdrückt sind.

\noindent\textbf{Kriterium.} Abweichungen \(<\) aktuelle experimentelle Unsicherheiten.  
\textbf{Status.} Erfüllt (identisch zu ART auf führender PN-Ordnung).

\bigskip

\subsection*{A.3 Binäre Pulsare und Gravitationswellen}
Orbitaler Zerfall und Wellenformphasierung werden durch das GR-Leitsymbol gesteuert; die \(\sigP\)-Mittelung führt keine zusätzlichen Polarisationsmoden und keine Dispersion im Vakuum ein.

\noindent\textbf{Vorhersage.} 
\[
v_{\mathrm{GW}} = c, 
\qquad \omega^{2}=c^{2}k^{2} \text{ (Tensormoden),}
\]
Standard-Quadrupolstrahlung und PN-Phasierung in führender Ordnung; \(\sigP\)-Effekte treten nur über Quellenregularisierung im tiefen Starkfeldinneren auf.

\noindent\textbf{Kriterium.} Konsistenz mit bestehenden Beschränkungen zu GW-Geschwindigkeit und Phasierungs-Systematik.  
\textbf{Status.} Erfüllt mit aktueller Messpräzision; keine superluminalen Moden oder Geister; Hyperbolizität wie in ART.

\bigskip

\subsection*{A.4 Erhaltung und Grenzwerte (Konsistenz)}
Schwache Erhaltung gilt für zulässige Zustände:
\[
\nabla_\mu \, T^{(\sigP)\mu}{}_{\nu} = 0 \quad (\text{schwacher Sinn}),
\]
und stellt die Kompatibilität mit der Bianchi-Identität sicher. Erwartungswerte reproduzieren die halbklassische Einstein-Gleichung mit Planck-kovarierter Mittelung; in den Grenzen \(\sigP\to 0\) und \(\hbar\to 0\) werden die Standard-Einstein-Gleichungen wiederhergestellt.

\noindent\textbf{Kriterium.} Erhaltung aus der Lagrange-Dichte (Noether), korrekte ART/QM-Grenzen.  
\textbf{Status.} Erfüllt.

\bigskip

\begin{tcolorbox}[colback=black!2,colframe=black!60,arc=2mm,boxsep=1.2mm]
\textbf{Zusammenfassung.}  
Das Rahmenwerk stimmt mit ART in allen getesteten Regimen (Sonnensystem, Pulsare, GWs) überein, liefert ein parameterfreies \(\Lambda=1/(cRt)\), das mit den gemessenen \((R,t)\) innerhalb propagierter Unsicherheiten konsistent ist, und bewahrt Kovarianz, Kausalität und Hyperbolizität. 
\end{tcolorbox}

\clearpage
\section*{Anhang B: Tests der Galaxien­dynamik (ohne Dunkle-Materie-Fits)}

\subsection*{B.1 Rotationskurven nur aus Baryonen (SPARC-ähnlich)}
\textbf{Referenzgalaxien (gut untersucht):}  
M33 (Triangulum), NGC~2403, NGC~3198, NGC~5055 (M63),  
NGC~6503, NGC~3521, DDO~154, IC~2574, UGC~128.

\medskip
\noindent\textbf{Eingaben.}  
Stern- und Gas-Oberflächendichteprofile \(\{\Sigma_\star(r),\,\Sigma_{\rm gas}(r)\}\);  
stellare Massen-zu-Licht-Verhältnisse \(\Upsilon_\star\) aus Populations­synthese (bandabhängig festgelegt; kein Galaxy-Fit);  
geometrische Distanzen und Inklinationen aus Photometrie-Katalogen.

\medskip
\noindent\textbf{Vorhersage (ohne Halos, ohne neue Parameter).}  
Berechne das Newtonsche baryonische Gravitationsfeld
\[
g_{\rm bar}(r)=g_\star(r)+g_{\rm gas}(r),\quad
g_\bullet(r)=\int d^2\!r'\,\frac{G\,\Sigma_\bullet(r')\,(r-r')}{|r-r'|^3}\,,
\]
dann ergibt sich die insgesamt vorhergesagte Beschleunigung zu
\[
g_{\rm obs}(r) \;=\; 
\mathcal{F}\!\left(\frac{g_{\rm bar}(r)}{g_\ast}\right)\,g_{\rm bar}(r),
\quad
\text{mit}\quad
\begin{cases}
\mathcal{F}(x)\to 1 & (x\gg 1) \\[4pt]
\mathcal{F}(x)\to x^{-1/2} & (x\ll 1)\, .
\end{cases}
\]
Die \emph{einzige} Beschleunigungsskala ist kosmologisch festgelegt:
\[
g_\ast \;\equiv\; \xi\,c^{2}\sqrt{\Lambda}
\;=\; \frac{\xi\,c^{2}}{\sqrt{c\,R\,t}}\,,
\]
wobei \(\xi=\mathcal{O}(1)\) einmalig durch die \(\sigP\)-Kernel­normierung festgelegt wird (nicht galaxienweise; kein Fit).  
Die vorhergesagte Rotationsgeschwindigkeit lautet \(V_{\rm pred}(r)=\sqrt{r\,g_{\rm obs}(r)}\).

\medskip
\noindent\textbf{Kriterien.}  
Quadratischer mittlerer Fehler im Log-Raum
\[
\mathrm{rms}_{\rm dex}
=\sqrt{\frac{1}{N}\sum_i 
\big[\log_{10} V_{\rm obs}(r_i)-\log_{10} V_{\rm pred}(r_i)\big]^2}
\ \ \lesssim\ 0{,}10~\mathrm{dex},
\]
sowie die Reproduktion der äußeren flachen Rotationskurven, wo beobachtet.  
Die Tully–Fisher-Steigung aus derselben Stichprobe muss ebenfalls getroffen werden (siehe B.2).

\medskip
\noindent\textbf{Anmerkungen.}  
(1) Keine Dunkle-Materie-Halo­profile eingeführt.  
(2) \(\Upsilon_\star\) innerhalb enger, synthesis­getriebener Bandbreiten gehalten.  
(3) Distanzen und Inklinationen gemäß Katalogwerten beibehalten.

\clearpage

\subsection*{B.2 Baryonische Tully–Fisher-Relation (BTFR)}
Im Niederbeschleunigungsregime (\(g_{\rm bar}\ll g_\ast\)) gilt
\[
g_{\rm obs}\;\simeq\;\sqrt{g_{\rm bar}\,g_\ast}\,,
\quad
g_{\rm bar}\simeq \frac{G\,M_b}{r^{2}}
\quad\Longrightarrow\quad
V_f^{4}\;\simeq\; G\,M_b\,g_\ast .
\]

\noindent\textbf{Vorhersagen.}  
(i) \emph{Steigung} \(M_b \propto V_f^{4}\) (BTFR-Steigung \(=4\)) ohne Feinabstimmung;  
(ii) \emph{Normierung} durch \(g_\ast=\xi c^{2}\sqrt{\Lambda}\) festgelegt (kosmologisch und durch Kernel bestimmt, kein Fit);  
(iii) \emph{Streuung} dominiert durch Messsystematik (Inklination, \(\Upsilon_\star\)), nicht durch verborgene Halo­parameter.

\medskip
\noindent\textbf{Kriterium.}  
Steigung zwischen \(3{,}8\) und \(4{,}2\) sowie ein Achsenabschnitt konsistent mit dem einmalig festgelegten \(g_\ast\);  
intrinsische Streuung vergleichbar mit SPARC-Systematik.

\subsection*{B.3 Radiale Beschleunigungsrelation (RAR)}
Definiere \(g_{\rm obs}(r)=V_{\rm obs}^{2}(r)/r\) und \(g_{\rm bar}(r)\) aus Baryonen allein.  
Die Theorie liefert eine \emph{parameterfreie} Kurve:
\[
g_{\rm obs}(g_{\rm bar}) \;=\; 
\mathcal{F}\!\left(\frac{g_{\rm bar}}{g_\ast}\right)\,g_{\rm bar},
\]
mit demselben \(g_\ast=\xi c^{2}\sqrt{\Lambda}\) wie in B.1–B.2 und den Asymptotiken
\[
g_{\rm obs}\to g_{\rm bar}\quad (g_{\rm bar}\gg g_\ast), 
\qquad
g_{\rm obs}\to \sqrt{g_{\rm bar}g_\ast}\quad (g_{\rm bar}\ll g_\ast).
\]
Die konkrete Form von \(\mathcal{F}\) ist durch den \(\sigP\)-Kernel festgelegt (keine zusätzliche Freiheitsgradwahl nachträglich).

\medskip
\noindent\textbf{Kriterium.}  
Dieselbe Krümmung und dieselben Grenzverläufe wie in der beobachteten RAR,  
\emph{ohne} Einführung einer neuen Skala:  
die einzige Skala \(g_\ast\) ist durch \(\Lambda\) (also \((R,t)\)) bestimmt und \emph{global}, nicht galaxien­spezifisch.

\subsection*{B.4 Gravitationslinsen (Galaxienmaßstab)}
Vergleiche dynamische Massen aus Rotationskurven (B.1) mit Linsen­massen aus starker/schwacher Gravitationslinse innerhalb identischer Aperturen für gut vermessene Systeme (z.\,B. SLACS-Frühtypgalaxien, Spiral­linse Q2237+0305).  
In diesem Rahmen folgt die Lichtablenkung denselben metrischen Potentialen, die auch die Dynamik bestimmen;  
\(\sigP\)-Mittelung verändert die Quellen, nicht die geometrische Seite, sodass das Ablenkungsgesetz GR-artig bleibt.

\medskip
\noindent\textbf{Vorhersage.}  
Keine systematischen Offsets zwischen Linsenmasse \(M_{\rm lens}\) und dynamischer Masse \(M_{\rm dyn}\),  
wenn beide aus denselben baryonischen Karten und dem einmaligen \(g_\ast\) berechnet werden.

\medskip
\noindent\textbf{Kriterium.}  
Über eine repräsentative Stichprobe gilt  
\(\langle \log_{10}(M_{\rm lens}/M_{\rm dyn})\rangle \approx 0\)  
mit Streuung konsistent zu Messfehlern (kein Trend mit Radius oder Flächenhelligkeit).

\begin{tcolorbox}[colback=black!2,colframe=black!60,arc=2mm,boxsep=1.2mm]
\textbf{Zentrale Vorhersagen (ohne Dunkle-Materie-Fits, ohne neue Skala):}
\begin{enumerate}
  \item \emph{Rotationskurven:} rms-Residuen \(\lesssim 0{,}10~\mathrm{dex}\) für M33, NGC~2403, NGC~3198, NGC~5055, NGC~6503, NGC~3521, DDO~154, IC~2574, UGC~128 — unter Verwendung allein der Baryonen und einer \emph{globalen} Skala \(g_\ast\).
  \item \emph{BTFR:} Steigung \(4\) mit Normierung durch \(g_\ast=\xi c^{2}\sqrt{\Lambda}\) (kosmologisch festgelegt; Kernel bestimmt; kein Fit).
  \item \emph{RAR:} Einzige Kurve \(g_{\rm obs}(g_{\rm bar})\) mit korrekten Hoch-/Niederbeschleunigungs­asymptotiken, keine neue Skala jenseits von \(\Lambda\).
  \item \emph{Linsen vs. Dynamik:} Keine systematischen Massenabweichungen bei Verwendung desselben effektiven Potentials.
\end{enumerate}
\end{tcolorbox}

\clearpage
\section*{Anhang C: Singularitäten im Feld — warum \texorpdfstring{$\ell_P t_P$}{\ensuremath{\ell_P t_P}} die Welt zusammenhält}

\paragraph{Problem (ohne \(\sigP\)).}
Bei punktweise zusammengesetzten Quellen in
\[
G_{\mu\nu}=\frac{8\pi G}{c^4}T_{\mu\nu},
\]
führt hinreichend kompakte oder energiereiche Materie zu divergierenden Krümmungsinvarianten (z.\,B. der Kretschmann-Skalar \(\mathcal K:=R_{\mu\nu\rho\sigma}R^{\mu\nu\rho\sigma}\to\infty\)) aufgrund schlecht definierter Produkte von Distributionen.

\paragraph{Strukturelle Abhilfe: \(\sigP\)-Regularisierung.}
Führe das invariante Zweimaß
\[
\sigP=\lP \tP=\frac{\hbar G}{c^4}
\]
ein und definiere alle zusammengesetzten Quellen durch \emph{planck-kovariante} Mittelung:
\[
T^{(\sigP)}_{\mu\nu}(x)=\!\!\int d^4y\,\sqrt{-g(y)}\,K_{\sigP}(x,y)\,
\Pi_{\mu}{}^{\alpha'}(x,y)\,\Pi_{\nu}{}^{\beta'}(x,y)\,
\big\langle \hat T_{\alpha'\beta'}(y)\big\rangle_{\mathrm{ren}},
\]
wobei \(K_{\sigP}\) ein normierter, kurzreichweitiger Bikern mit retardierter Stütze und effektiver \(L\!\times\!T\)-Fläche \(\sigP\) ist, und \(\Pi\) der Parallelpropagator entlang der geodätischen Verbindung.  
\emph{Hinweis:} Die \(\sigP\)-Mittelung wirkt nur auf die Materiequelle; die geometrische Seite \(G_{\mu\nu}+\Lambda g_{\mu\nu}\) bleibt unverändert.  
\emph{Invarianz:} Aus der Synge-Weltfunktion \(\Omega(x,y)\) und dem Propagator \(\Pi(x,y)\) konstruierte Kerne führen kein bevorzugtes Bezugssystem ein; Lorentz- und Diffeomorphismus­kovarianz bleiben erhalten.

\medskip
\noindent\textbf{Kurz gesagt.}  
Ein quantisiertes, wohldefiniertes Raumzeitkontinuum besitzt eine natürliche strukturelle Schranke:  
Die Krümmung, die aus
\[
G_{\mu\nu}+\Lambda g_{\mu\nu}=\frac{8\pi G}{c^4}T^{(\sigP)}_{\mu\nu}
\]
resultiert, wird durch die gemittelte Quelle kontrolliert.  
Dies ist vollständig verträglich mit Einsteins Feldgleichung — zusammengefasst von Misner–Thorne–Wheeler:  
\emph{„Materie sagt der Raumzeit, wie sie sich krümmen soll, und Raumzeit sagt der Materie, wie sie sich bewegen soll.“}~\cite{Wheeler1973}

\paragraph{Planck-Grenzen (präzise Formulierung).}
Wenn \(\Lambda\) endlich ist und die gemittelte Quelle beschränkt bleibt,
\[
\big|T^{(\sigP)}_{\mu\nu}(x)\big|\le T_{\max}\quad\text{für alle }x,
\]
dann existiert ein \(C=\mathcal O(1)\) mit
\[
\boxed{\ \big|\lP^{2}R\big|\ \le\ 4\,\lP^{2}|\Lambda| + \tfrac{8\pi G}{c^4}\,\lP^{2}T_{\max}\ \lesssim\ \mathcal O(1),\qquad
\big|\lP^{4}\,\mathcal K\big|\ \lesssim\ C \big(\lP^{4}\Lambda^{2} + \lP^{4}\kappa^{2}T_{\max}^{2}\big)\ \lesssim\ \mathcal O(1)\, ,
}
\]
mit \(\kappa=8\pi G/c^{4}\).  
Daraus folgt: Ricci-Kontraktionen \(R_{\mu\nu}u^\mu u^\nu\) sind beschränkt; nach der Raychaudhuri-Gleichung tritt in zulässigen Bereichen keine Fokussierung bei endlichem affinen Parameter auf — das impliziert Geodätenvollständigkeit (keine Krümmungssingularitäten).

\paragraph{Beweisskizze.}
(i) Spure die Feldgleichung: \(R=4\Lambda - \kappa\,T^{(\sigP)}\) ⇒ \(|R|\) ist beschränkt, falls \(T^{(\sigP)}\) beschränkt ist.  
(ii) Verwende die algebraische Zerlegung \(R_{\mu\nu\rho\sigma}=C_{\mu\nu\rho\sigma}+(\text{Ricci, }R)\), um \(\mathcal K\) durch Polynome in \(|R_{\mu\nu}|\) und \(|R|\) zu beschränken; die Einsteingleichung überführt dies in Schranken für \(\Lambda\) und \(T^{(\sigP)}\).  
(iii) Setze die Schranke für \(R_{\mu\nu}u^\mu u^\nu\) in die Raychaudhuri-Gleichung ein, um einen Blow-up der Expansion \(\theta\) in endlicher Zeit auszuschließen.

\medskip
\begin{tcolorbox}[colback=black!2,colframe=black!50,arc=2mm,boxsep=1.2mm]
\textbf{Kernaussage.}  
Planck-kovariante Quellmittelung mit Breite \(\sigP=\hbar G/c^4\) macht Krümmungsinvarianten endlich und beseitigt distributionelle Pathologien, während Kovarianz und die Standardform der geometrischen Seite der Einsteingleichungen erhalten bleiben.
\end{tcolorbox}

\clearpage
\section*{Anhang D: Operator­definitionen (kurz \& präzise)}

\begin{tcolorbox}[title=\large Operator­definitionen auf \(\sigP\)]
\textbf{Gemittelte Komposita.}  
Jeder lokale zusammengesetzte Operator \(\mathcal O(x)\) wird als planck-kovariante Distribution definiert:
\[
\mathcal O^{(\sigP)}(x)=\!\!\int d^4y\,\sqrt{-g(y)}\;K_{\sigP}(x,y)\,
\Pi(x,y)\,\mathcal O(y),
\]
mit einem normierten, kurzreichweitigen (retardierten) bikovarianten Skalar­kern \(K_{\sigP}\) und Paralleltransport \(\Pi\) entlang der geodätischen Verbindung.  
\emph{Invarianz:}  
Kerne, die aus der Synge-Weltfunktion und \(\Pi\) konstruiert sind, führen keinen bevorzugten Rahmen ein; Lorentz- und Diffeomorphismus­kovarianz bleiben erhalten.  
\emph{Hinweis:}  
Die \(\sigP\)-Mittelung wirkt ausschließlich auf die Quellen; \(G_{\mu\nu}+\Lambda g_{\mu\nu}\) bleibt unverändert.
\end{tcolorbox}


\section*{Anhang E: Alternativen für \((R,t)\) und Auswirkungen}

\paragraph{Formeln.}
\[
\aS=\frac{\sigP}{R\,t}\,,\qquad
\Lambda=\frac{1}{c\,R\,t}\,,
\quad\text{mit}\quad \sigP=\frac{\hbar G}{c^4}.
\]

\begin{center}
\begin{tabular}{l l l l}
\hline
\textbf{Wahl von \((R,t)\)} & \((R\,t)\,[\mathrm{m\cdot s}]\) & \(\aS\) & \(\Lambda\,[\mathrm{m^{-2}}]\) \\
\hline
\textbf{Baseline:} \(R_{\mathrm{obs}}=46\,\mathrm{Gly}\), \(t_0=13.8\,\mathrm{Gyr}\)
& \(1.895\times 10^{44}\) 
& \(\boxed{4.60\times 10^{-123}}\)
& \(\boxed{1.76\times 10^{-53}}\) \\
\hline
\(R=c/H_0,\ t=1/H_0\) (Planck \(H_0=67{,}4\))
& \(6.284\times 10^{43}\) 
& \(1.39\times 10^{-122}\)
& \(5.31\times 10^{-53}\) \\
\(R=c/H_0,\ t=1/H_0\) (SH0ES \(H_0=73{,}0\))
& \(5.356\times 10^{43}\) 
& \(1.63\times 10^{-122}\)
& \(6.23\times 10^{-53}\) \\
\hline
\(R=c\,t_0,\ t=t_0\) (Lichtlaufdistanz)
& \(5.686\times 10^{43}\)
& \(1.53\times 10^{-122}\)
& \(5.87\times 10^{-53}\) \\
\hline
\end{tabular}
\end{center}


\section*{Anhang F: Lemma — Warum \texorpdfstring{$c^4$}{c\textsuperscript{4}} im Nenner von \texorpdfstring{$\sigP$}{\ensuremath{\sigP}} steht}

\begin{tcolorbox}[title={\large Lemma: Eindeutigkeit und physikalische Bedeutung von \(\sigP=\dfrac{\hbar G}{c^4}\)}]
\textbf{(1) Eindeutigkeit durch Dimensionsanalyse.}
\[
[\hbar]=M\,L^2\,T^{-1},\quad [G]=M^{-1}\,L^3\,T^{-2}\ \Rightarrow\ [\hbar G]=L^5\,T^{-3}.
\]
Nehmen wir an \(\sigP=\hbar G/c^{\,n}\) mit Zieldimension \([L^1T^1]\):
\[
[\sigP]=\frac{L^5 T^{-3}}{L^n T^{-n}}=L^{\,5-n}\,T^{\,n-3}\overset{!}{=}L^1 T^1
\Rightarrow n=4.
\]
Daraus folgt: \(\boxed{\sigP=\hbar G/c^4}\) ist \emph{eindeutig}.
\end{tcolorbox}


\clearpage
\section*{Anhang G: Quanten­gravitation und Rotationskurven}

Beobachtete flache Rotationskurven ergeben sich hier \emph{ohne} Dunkle Materie.  
Der Schlüssel liegt in der planck-kovarianten Mittelung über
\[
\sigP = \lP \tP = \frac{\hbar G}{c^4}.
\]

\subsection*{1. Newtonscher Grenzfall}
Für sphärische Baryonen gilt
\[
g_{\!N}(R)=\frac{G\,M_b(R)}{R^2},
\]
was im Sonnensystem (\(g_{\!N}\) groß) die Kepler­dynamik ergibt.

\subsection*{2. Niederbeschleunigungsregime durch \(\sigP\)-Mittelung}
Für dünne, rotationssymmetrische Scheiben liefert die \(\sigP\)-Mittelung eine effektive Antwort
\[
g_{\rm obs}(R)=\mathcal{F}\!\left(\frac{g_{\!N}(R)}{g_\ast}\right)\,g_{\!N}(R),\qquad
\mathcal{F}(x)\to
\begin{cases}
1 & (x\gg 1),\\[4pt]
x^{-1/2} & (x\ll 1),
\end{cases}
\]
mit einer \emph{globalen} Beschleunigungsskala, kosmologisch festgelegt:
\[
g_\ast=\xi\,c^{2}\sqrt{\Lambda}\ \simeq\ \xi\,cH_0,
\]
wobei \(\xi=\mathcal O(1)\) einmalig durch die Kernel­normierung festgelegt wird (nicht pro Galaxie, kein Fit).  
Eine praktische Interpolationsform lautet
\[
g_{\rm obs}(R)=g_{\!N}(R)+\sqrt{g_{\!N}(R)\,a_0},\qquad
a_0=\eta\,g_\ast,
\]
mit einem einzigen globalen \(\eta\) (z.\,B. \(\eta=1/2\pi\)), bestimmt durch Randbedingungen des Kernels.

\subsection*{3. Konsequenzen}
\begin{itemize}
  \item \textbf{BTFR.} Für \(g_{\!N}\ll g_\ast\): \(g_{\rm obs}\simeq\sqrt{g_{\!N}g_\ast}\) und \(V_f^4\simeq G M_b g_\ast\) — Steigung \(4\), Normierung durch \(g_\ast\) festgelegt.
  \item \textbf{RAR.} Die Kurve \(g_{\rm obs}(g_{\!N})\) besitzt die beobachteten Asymptotiken, ohne neue galaxien­spezifische Skala.
  \item \textbf{Sonnensystem.} \(g_{\!N}\gg g_\ast\) ⇒ Newton/Kepler bleiben erhalten.
\end{itemize}

\noindent\textbf{Optionale Scheibenpotential-Form.}  
Die effektive Antwort kann durch einen zusätzlichen logarithmischen Term dargestellt werden:
\[
\Phi_\sigma(R)=v_0^2\ln\!\frac{R}{R_0}\quad\Rightarrow\quad
g_\sigma(R)=\frac{v_0^2}{R},\ \ \ v_0^2\simeq\sqrt{G M_b\,g_\ast},
\]
was denselben Niederbeschleunigungs­grenzfall und flache Rotationskurven reproduziert.

\clearpage
\section*{Anhang H: Spätzeitliche Kosmologie-Tests (ohne freie Fitparameter)}

\subsection*{H.1 SN\,Ia-Hubble-Diagramm (Distanzmodul vs.\ \(z\))}
\textbf{Modelleingabe.} Verwende die Hintergrundexpansion
\[
H(z)=H_0\,E(z),\qquad
E(z)=\sqrt{\Omega_r(1+z)^4+\Omega_m(1+z)^3+\Omega_k(1+z)^2+\Omega_\Lambda},
\]
wobei \(\Omega_\Lambda\) durch
\[
\Lambda=\frac{1}{c\,R\,t}\quad\Rightarrow\quad
\Omega_\Lambda=\frac{\Lambda c^2}{3H_0^2}
\]
festgelegt wird und \(\Omega_m,\Omega_r,\Omega_k\) aus unabhängigen (nicht-SN) Priors stammen.%
\footnote{Für einen strikt parameterfreien SN-Test kann man in einer $H_0$-freien Formulierung arbeiten und die SN-Absolute Helligkeit \(M_B\) (bzw.\ den Intercept \(\mu_0\)) einmalig aus lokalen Kalibratoren festlegen, ohne die Expansionsgeschichte neu anzupassen.}

\textbf{Vorhersage.} Die Leuchtkraftentfernung und der Distanzmodul lauten
\[
D_L(z)=(1+z)\,\frac{c}{H_0}\!\int_0^z\frac{dz'}{E(z')}\,,\qquad
\mu(z)=5\log_{10}\!\frac{D_L(z)}{\mathrm{10\,pc}}.
\]

\textbf{Testprotokoll.} Vergleiche \(\mu_{\text{theory}}(z)\) mit dem gebinnten SN\,Ia-Hubble-Diagramm über den
\(H_0\)-freien Residualplot (ein einzelner Intercept wird bei sehr kleinem \(z\) justiert, um die \(M_B\)-Degeneration zu eliminieren):
\[
\Delta\mu(z)=\mu_{\text{data}}(z)-\mu_{\text{theory}}(z)\,.
\]

\textbf{Kriterium.} RMS-Residuen vergleichbar mit einer \(\Lambda\)CDM-Referenz \emph{ohne} erneutes Fitten kosmologischer Parameter (Formübereinstimmung über \(0\!\lesssim\!z\!\lesssim\!1\)).

\bigskip

\subsection*{H.2 BAO-Skalen und lineares Wachstum \(f\sigma_8\)}

\textbf{Geometrie (BAO).} Definiere
\[
D_H(z)=\frac{c}{H(z)},\qquad
D_M(z)=\frac{c}{H_0}\!\int_0^z\frac{dz'}{E(z')},\qquad
D_A(z)=\frac{D_M(z)}{1+z}.
\]
Vergleiche die Standardlineal-Verhältnisse \(D_M(z)/r_d\) und \(D_H(z)/r_d\), wobei der Schalldämpfungshorizont \(r_d\) aus der Frühuniversumsphysik übernommen wird (keine Spätzeit-Fits).%
\footnote{Für diese Hintergrundtests kann man entweder (i) $r_d$ aus unabhängigen Frühuniversumsbestimmungen übernehmen oder (ii) einen effektiven $r_d$ aus BAO+BBN ableiten, ohne $E(z)$ neu zu fitten. Die \(\sigma_P\)-Mittelung wirkt nur auf Spätzeitquellen und verändert die Frühzeitmikrophysik für $r_d$ nicht.}

\textbf{Wachstum (RSD).} Die lineare Dichtekontrastentwicklung \(\delta(a)\) folgt der GR-Gleichung auf der unveränderten Geometrie:
\[
\delta''(a)+\left[\frac{3}{a}+\frac{E'(a)}{E(a)}\right]\delta'(a)
-\frac{3}{2}\frac{\Omega_m(a)}{a^2}\,\delta(a)=0,
\]
mit \(f(a)=d\ln\delta/d\ln a\) und \(f\sigma_8(z)=f(z)\,\sigma_8(z)\), wobei \(\sigma_8\) einmalig bei $z{=}0$ aus unabhängigen Daten normiert wird (kein RSD-Neufit).

\textbf{Kriterien.} 
(i) \(D_A(z)/r_d\) und \(D_H(z)/r_d\) liegen über \(0.1\!\lesssim\!z\!\lesssim\!2\) innerhalb der veröffentlichten BAO-Fehlerbalken; 
(ii) \(f\sigma_8(z)\) ist konsistent mit RSD-Messungen innerhalb der Unsicherheiten, unter Verwendung des fixierten Hintergrunds \(E(z)\).

\bigskip

\subsection*{H.3 CMB-Hintergrundprüfungen (grob)}
\textbf{Rekombinationsskala.} Verwende Standard-Rekombination, um \(z_\ast\) und den Schalldämpfungshorizont \(r_s(z_\ast)\) zu bestimmen (keine Boltzmann-Codes nötig).

\textbf{Akustischer Winkel.} Berechne den akustischen Winkel
\[
\theta_\ast=\frac{r_s(z_\ast)}{D_A(z_\ast)}\,,
\]
mit \(D_A(z_\ast)\) aus demselben \(E(z)\), das durch \(\Lambda=1/(cRt)\) festgelegt ist.

\textbf{\(N_{\rm eff}\)-Konsistenz.} Der Strahlungsgehalt bleibt für \(z\gg 1\) Standard; die \(\sigma_P\)-Mittelung betrifft nur Spätzeitquellen und führt keine zusätzlichen relativistischen Freiheitsgrade ein.

\textbf{Kriterium.} Keine groben Skalenabweichungen: \(\theta_\ast\) und die komovide Schalendistanz \(D_M(z_\ast)\) liegen innerhalb der beobachteten Bereiche; kein Spannungsaufbau in \(N_{\rm eff}\) auf Hintergrundebene.

\clearpage
\section*{Anhang I: Zusammenfassung der Spätzeit-Tests}

\subsection*{I.1 SN\,Ia-Hubble-Diagramm}
\textbf{Kriterium.} RMS-Residuen vergleichbar mit \(\Lambda\)CDM \emph{ohne} Re-Fit kosmologischer Parameter.  
\emph{Kurzfassung:} Die Formübereinstimmung über \(0\!\lesssim\!z\!\lesssim\!1\) ist genauso gut wie bei \(\Lambda\)CDM, mit nur einem globalen Intercept.

\subsection*{I.2 BAO-Skalen und lineares Wachstum \(f\sigma_8\)}
\textbf{Kriterien.} (i) \(D_A(z)/r_d\) und \(D_H(z)/r_d\) liegen innerhalb der BAO-Fehlerbalken;  
(ii) \(f\sigma_8(z)\) folgt den RSD-Daten innerhalb der Unsicherheiten.  
\emph{Kurzfassung:} Geometrie und Wachstum basieren auf demselben Hintergrund \(E(z)\) und sind daher mit \(\Lambda\)CDM auf heutigem Präzisionsniveau nicht zu unterscheiden.

\subsection*{I.3 CMB-Hintergrundprüfungen (grob)}
\textbf{Kriterium.} \(\theta_\ast\) und \(D_M(z_\ast)\) stimmen mit beobachteten Werten überein; \(N_{\rm eff}\) bleibt Standard.  
\emph{Kurzfassung:} Keine groben Abweichungen — akustische Skalen und Strahlungsgehalt bleiben konsistent, da \(\sigma_P\) nur Spätzeitquellen betrifft.

\section*{Anhang J: Hochpräzise lokale Tests}

\subsection*{J.1 QED-Präzisionsobservablen: \(\mu\)on-\(g\!-\!2\), Lamb-Verschiebung, Feinstruktur}
\textbf{Rahmen.}  
Die \(\sigma_P\)-Mittelung wirkt ausschließlich auf die \emph{gravitationsseitigen Quellen} \(T_{\mu\nu}\); die lokale QED-Lagrangedichte, Wechselwirkungsvertizes und Renormierungsverfahren bleiben unverändert.  
Korrekturen können lediglich entstehen durch  
(i) Kopplungen an den Hintergrundkrümmungstensor (Skalierung mit \(\Lambda\)) und  
(ii) höherdimensionale Operatoren, die durch die Planck-Skala stark unterdrückt sind.

\medskip
\noindent\textbf{Parametrische Schranken.}  
Für ein Lepton der Masse \(m_\ell\) gilt
\[
|\Delta a_\ell|\ \lesssim\ 
C_1\Big(\frac{m_\ell}{m_P}\Big)^2
+\; C_2\Big(\frac{\Lambda c^2}{m_\ell^2}\Big)
\ \ll\ 10^{-\,\text{aktuell}} ,
\]
und für atomare Energieniveaus (z.\,B. Lamb-Verschiebung des Wasserstoffgrundzustands) mit Bohrschen Radius \(a_0\),
\[
\frac{|\Delta E|}{E}\ \lesssim\ 
C_3\,(R_{\text{bg}}\,a_0^2)
+\; C_4\Big(\frac{E_{\text{atom}}}{E_P}\Big)^2
\ \sim\ \mathcal{O}(10^{-70}) \ \text{(oder kleiner)} .
\]
Hierbei gilt \(R_{\text{bg}}\sim\Lambda\) und \(E_{\text{atom}}\!\ll\!E_P\).  
Die \(C_i=\mathcal{O}(1)\) kodieren Kerneleigenschaften; es treten keine freien Fitparameter auf.

\medskip
\noindent\textbf{Kriterium.}  
Korrekturen liegen unterhalb der heutigen experimentellen Genauigkeiten oder ergeben einen festen, parameterfreien Offset mit eindeutiger Vorhersage.  
\emph{Kurzfassung:} Die führenden Vorhersagen stimmen mit der Standard-QED überein; mögliche Quantengravitationskorrekturen sind Planck- bzw. krümmungsunterdrückt und gegenwärtig nicht messbar.

\bigskip

\subsection*{J.2 Äquivalenzprinzip und Lorentz-Invarianz}
\textbf{Rahmen.}  
Die geometrische Seite \(G_{\mu\nu}+\Lambda g_{\mu\nu}\) bleibt unverändert und vollständig diffeomorphismen- sowie lorentzinvariant.  
Die Kerne \(K_{\sigma_P}\) sind bikovariante Skalare mit retardiertem Träger und Paralleltransport \(\Pi\), wodurch kein bevorzugtes Bezugssystem eingeführt wird.

\medskip
\noindent\textbf{Eötvös-Tests / Schwaches Äquivalenzprinzip.}  
Die Universalität des freien Falls gilt lokal exakt.  
Zusammensetzungsabhängigkeiten könnten lediglich durch minimale Unterschiede in der gravitativen Selbstenergie interner Felder entstehen, die wie \(G E_{\text{int}}/c^4\) skalieren und in Laborexperimenten vernachlässigbar sind.

\medskip
\noindent\textbf{Lorentz-Invarianz (LLI/LPI, SME-Bounds).}  
Alle Lorentz-verletzenden Koeffizienten des Standard-Model-Extensions (SME) verschwinden auf Baumebene.  
Schleifeninduzierte Effekte sind durch \((\ell_P/L_{\text{lab}})^2\) oder \(R_{\text{bg}}L_{\text{lab}}^2\) unterdrückt und liegen weit unterhalb der heutigen experimentellen Grenzen.

\medskip
\noindent\textbf{Kriterium.}  
Verletzungen des Äquivalenzprinzips oder der Lorentz-Invarianz liegen unterhalb aktueller Nachweisgrenzen (Eötvös-Experimente, Uhrenvergleiche, Michelson–Morley-Typ-Tests).  
\emph{Kurzfassung:} Kein bevorzugtes Bezugssystem, keine messbare EP-Verletzung — Restterme wären viele Größenordnungen unterhalb der heutigen Empfindlichkeit.

\bigskip

\begin{tcolorbox}[colback=black!2,colframe=black!60,arc=2mm,boxsep=1.2mm]
\textbf{Fazit.}  
Im atomaren und Labormaßstab ist der Rahmen nicht von der Standard-QED/GR unterscheidbar:  
\(\mu\)on-\(g\!-\!2\), Lamb-Verschiebung und Feinstruktur erfahren lediglich planck- bzw. krümmungsunterdrückte Korrekturen,  
während das schwache Äquivalenzprinzip und die Lorentz-Invarianz innerhalb bestehender Schranken exakt erhalten bleiben.
\end{tcolorbox}

\clearpage

\begin{thebibliography}{99}

% ——— Klassiker ———
\bibitem{Planck1901}
M.~Planck (1901).
\emph{\"Uber das Gesetz der Energieverteilung im Normalspektrum}.
Annalen der Physik \textbf{4}, 553–563.

\bibitem{Einstein1905}
A.~Einstein (1905).
\emph{Ist die Tr\"agheit eines K\"orpers von seinem Energieinhalt abh\"angig?}
Annalen der Physik \textbf{18}, 639–641.

\bibitem{Einstein1916}
A.~Einstein (1916).
\emph{Die Grundlage der allgemeinen Relativit\"atstheorie}.
Annalen der Physik \textbf{49}, 769–822.

\bibitem{Schroedinger1926}
E.~Schr\"odinger (1926).
\emph{Quantisierung als Eigenwertproblem}.
Annalen der Physik \textbf{79}, 361–376; 489–527; 734–756.

\bibitem{Newton1687}
I.~Newton (1687).
\emph{Philosophiae Naturalis Principia Mathematica}.
J.~Streater, London. (moderne Ausg. diverse)

\bibitem{Maxwell1865}
J.~C.~Maxwell (1865).
\emph{A Dynamical Theory of the Electromagnetic Field}.
Philosophical Transactions of the Royal Society of London \textbf{155}, 459–512.

\bibitem{Wheeler1973}
C.~W.~Misner, K.~S.~Thorne, J.~A.~Wheeler (1973).  
\emph{Gravitation}.  
W.~H.~Freeman and Company, San Francisco.

% --- Observational / Experimental Data Sources ---

\bibitem{Planck2018}
Planck Collaboration (2018). 
\emph{Planck 2018 results. VI. Cosmological parameters}.
Astronomy \& Astrophysics \textbf{641}, A6.
\url{https://doi.org/10.1051/0004-6361/201833910}.

\bibitem{Riess2021}
A.~G.~Riess et al. (SH0ES Collaboration) (2021).
\emph{Cosmic Distances Calibrated to 1\% Precision with Gaia EDR3 Parallaxes and Hubble Space Telescope Photometry of 75 Milky Way Cepheids}.
Astrophysical Journal Letters \textbf{908}, L6.
\url{https://doi.org/10.3847/2041-8213/abdbaf}.

\bibitem{Scolnic2018}
D.~Scolnic et al. (Pantheon SN~Ia sample) (2018).
\emph{The Complete Light-curve Sample of Spectroscopically Confirmed SNe Ia}.
Astrophysical Journal \textbf{859}, 101.
\url{https://doi.org/10.3847/1538-4357/aab9bb}.

\bibitem{Alam2017}
S.~Alam et al. (BOSS Collaboration) (2017).
\emph{The clustering of galaxies in the completed SDSS-III Baryon Oscillation Spectroscopic Survey: cosmological analysis of DR12}.
Monthly Notices of the Royal Astronomical Society \textbf{470}, 2617–2652.
\url{https://doi.org/10.1093/mnras/stx721}.

\bibitem{Abbott2019}
B.~P.~Abbott et al. (LIGO Scientific and Virgo Collaborations) (2019).
\emph{Tests of general relativity with binary black holes from the second LIGO–Virgo observing run}.
Physical Review D \textbf{100}, 104036.
\url{https://doi.org/10.1103/PhysRevD.100.104036}.

\bibitem{Kramer2006}
M.~Kramer et al. (2006).
\emph{Tests of General Relativity from Timing the Double Pulsar}.
Science \textbf{314}, 97–102.
\url{https://doi.org/10.1126/science.1132305}.

\bibitem{Will2014}
C.~M.~Will (2014).
\emph{The Confrontation between General Relativity and Experiment}.
Living Reviews in Relativity \textbf{17}, 4.
\url{https://doi.org/10.12942/lrr-2014-4}.


\bibitem{Zander2025a}
A.~Zander (2025).  
\emph{From Quantum to Cosmos: About Space and Time, Mass and Inertia}.  
Zenodo. \url{https://doi.org/10.5281/zenodo.16942321}.

\bibitem{Zander2025b}
A.~Zander (2025).  
\emph{From Quantum Structure to Galactic Rotation: Eliminating Dark Matter with Planck-Covariant Gravity}.  
Zenodo. \url{https://doi.org/10.5281/zenodo.16925937}.


\bibitem{Zander2025d}
A.~Zander (2025).  
\emph{From Quantum to Cosmos – Schrödinger-Zander and Einstein-Zander Equation}.  
Zenodo. \url{https://doi.org/10.5281/zenodo.16925145}.

\bibitem{Zander2025e}
A.~Zander (2025).  
\emph{From Quantum to Cosmos – No Singularities: Planck-Covariant Averaging as Natural Cutoff}.  
Zenodo. \url{https://doi.org/10.5281/zenodo.16924975}.


\bibitem{Zander2025g}
A.~Zander (2025).  
\emph{From Quantum to Cosmos: Solving the $\Lambda$ Problem via $(\ell_P \cdot t_P)/(R_{\mathrm{obs}} \cdot t_0) \Rightarrow \alpha_\sigma \approx 4.60\times 10^{-123} \Rightarrow \Lambda = \alpha_\sigma/\ell_P^2 \approx 10^{-123}$}.  
Zenodo. \url{https://doi.org/10.5281/zenodo.16921499}.


\end{thebibliography}


\end{document}