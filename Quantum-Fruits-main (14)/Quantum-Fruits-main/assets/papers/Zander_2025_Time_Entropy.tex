\documentclass[12pt,a4paper,twoside]{article}

\usepackage[ngerman]{babel}
\usepackage[utf8]{inputenc}
\usepackage[T1]{fontenc}
\usepackage{amsmath,amssymb,amsfonts}
\usepackage{mathrsfs}
\usepackage{geometry}
\usepackage{fancyhdr}
\usepackage{titlesec}
\usepackage{abstract}

% Geometrie und Layout
\geometry{top=25mm, bottom=25mm, left=25mm, right=25mm}
\setlength{\parindent}{0em}
\setlength{\parskip}{0.5em}

% Header
\pagestyle{fancy}
\fancyhf{}
\fancyhead[LE,RO]{\thepage}
\fancyhead[RE]{Zander et al.}
\fancyhead[LO]{Zeit und Entropie als abgeleitete Größen}

% Titel-Formatierung
\title{\textbf{Zeit, Entropie und Irreversibilität als abgeleitete Größen quantisierter Raumzeit}\\
\large Eine wirkungsbasierte Formulierung ohne vorausgesetzten Zeitparameter}

\author{Adrian Zander}
\date{\today}

\begin{document}

\maketitle

\begin{abstract}
\noindent
In der etablierten Thermodynamik und der Allgemeinen Relativitätstheorie werden Zeit und Entropie häufig als a priori existente Strukturen oder statistische Emergenzen behandelt. Diese Arbeit postuliert stattdessen einen fundamentaleren Ansatz: Die Quantisierung der Raumzeit durch eine minimale Wirkungszelle $\sigma_P = \hbar G / c^4$. Wir zeigen, dass Zeitordnung und Entropie keine intrinsischen Eigenschaften des Universums sind, sondern aus der Zählung diskreter Wirkungsbelegungen (Ticks) hervorgehen. Die thermodynamische Temperatur wird hierbei als Irreversibilitätsrate $\mathcal{R}$ neu definiert, was eine skaleninvariante Beschreibung von makroskopischer Materie bis hin zu Schwarzen Löchern ermöglicht und das Informationsparadoxon durch eine unitäre Buchhaltung auflöst.
\end{abstract}

\hrule
\vspace{1em}

\section{Einführung: Das Problem des externen Zeitparameters}

Die konventionelle Physik beschreibt Dynamik als Funktion der Zeit $f(t)$. Dies setzt die Existenz einer kontinuierlichen, externen Zeitvariable voraus. In Regionen extremer Gravitation oder an der Planck-Skala verliert dieser Zeitbegriff jedoch seine operative Bedeutung.

Der hier vorgestellte Ansatz (Zander-Framework 2025) invertiert die Logik: Nicht die Zeit ermöglicht Veränderung, sondern diskrete Veränderung (Wirkung) erzeugt Zeit. Physikalisch real ist ausschließlich die Wirkung $\Delta A$. Zeit und Entropie sind Buchhaltungsgrößen dieser Wirkung, skaliert durch die Boltzmann-Konstante $k_B$.

\section{Axiomatik der Wirkungszellen}

\subsection{Das fundamentale Quartett $\{\hbar, c, G, k_B\}$}
Das Modell postuliert, dass die physikalische Realität vollständig durch den Satz $\{\hbar, c, G, k_B\}$ beschrieben wird. Dabei bündeln $\hbar, c, G$ die geometrische Dynamik, während $k_B$ den Wechselkurs zwischen Information und physikalischer Energie definiert.

\subsection{Die fundamentale Zelle $\sigma_P$}
Wir definieren die Raumzeit nicht als kontinuierliches Mannigfaltigkeit, sondern als ein Gitter minimaler Wirkungsaufnahmekapazität. Die Größe der Elementarzelle bestimmt sich aus:
\begin{equation}
    \sigma_P = \frac{\hbar G}{c^4} \quad [\text{m} \cdot \text{s}].
\end{equation}
Diese Größe repräsentiert die minimale Kopplung von Quantenmechanik ($\hbar$) und Gravitation ($G$) in der Raumzeitstruktur.

\subsection{G als emergenter informationaler Gradient}
In diesem Framework ist die Gravitationskonstante $G$ kein fundamentaler Input, sondern eine abgeleitete Größe des $k_B$-Sättigungsgradienten über die Elementarzellen:
\begin{equation}
    G = \frac{\sigma_P c^4}{\hbar} = \frac{\hbar c^3}{W}.
\end{equation}
Die Krümmung der Raumzeit emergiert dort, wo Information ($k_B$) einen Gradienten in der Wirkungsbelegung erzeugt. Gravitation ist der geometrische Ausdruck des Bestrebens des Feldes, diesen informationalen Druck auszugleichen.

\subsection{Der Tick-Index $i$}
Jeder physikalische Prozess mit Energie $E$ und Dauer $\Delta t$ belegt eine endliche Anzahl dieser Zellen. Wir definieren den dimensionslosen Tick-Index $i$:
\begin{equation}
    i = \frac{\Delta A}{\sigma_P} = \frac{E \cdot \Delta t}{\sigma_P}.
\end{equation}
Dabei ist $i$ streng ganzzahlig zu verstehen ($\Delta A \ge \sigma_P$), im makroskopischen Limes wird es als kontinuierlich genähert.

\section{Entropie und Zeit als Ableitungen}

\subsection{Entropie als Zählgröße}
Entropie $S$ wird in diesem Formalismus von ihrer statistischen Interpretation (Maß der Unordnung) befreit. Sie ist definiert als das kumulative Inventar der belegten Wirkungszellen:
\begin{equation}
    S \equiv k_B \cdot N_{\text{ticks}} = k_B \sum_{k} i_k.
\end{equation}
Ein System, das keine Wirkung umsetzt ($i=0$), besitzt keine Entropiedynamik. Informationsverlust ist ausgeschlossen, da jeder Tick $i$ eine physikalische Spur in der Matrix hinterlässt (Unitarität).

Für ein System mit der Masse $m$ und einer Referenzfrequenz $f$ (z.B. die fundamentale Oszillation der Zelle) definieren wir die \textbf{relativen Tick-Ratios}:
\begin{equation}
    S1 = \frac{mc^2}{hf}, \quad s2 = \frac{hf}{mc^2}.
\end{equation}
Dabei stellt $S1$ die Anzahl der Wirkungsquanten dar, die dem energetischen Äquivalent der Masse entsprechen. Diese dimensionslosen Größen bilden ein duales Paar, das die Skalierung zwischen Teilcheneigenschaften und Wellendynamik beschreibt.

Auf dieser Basis lässt sich die \textbf{Zander-Entropie ($S_Z$)} als fundamentale Erweiterung der thermischen Maße klassifizieren. Während Boltzmann ($S_B = k_B \ln W$) die Zustandszahl und von Neumann ($S_{vN} = -k_B \text{tr}(\rho \ln \rho)$) die Unschärfe der Besetzung beschreiben, misst $S_Z$ die \textbf{energetische Moden-Verteilung}:
\begin{equation}
    S_Z = k_B \ln \left( \frac{E_m}{E_{hf}} \right).
\end{equation}
Sie ist das direkte Maß für die geometrische Bremslast (Mass-Burden). Ein Anstieg von $S_Z$ markiert den Übergang von freien, oszillatorischen Wirkungsquanten in statische, gravitative Trägheit und bildet somit den natürlichen \textbf{Pfeil der Zeit} im Zander-Framework.

\subsection{Das Zwei-Niveau-Toymodell}
Betrachten wir ein System mit zwei Basisvorgängen: Einem hochenergetischen Spin-Prozess ($|hf\rangle$) und einem trägen Masse-Zustand ($|mc^2\rangle$). Die Evolution des Erwartungswerts $\langle \hat{S}_{\sigma_P} \rangle$ beschreibt den Übergang von kohärenter Quantenwirkung hin zur dissipativen Trägheit. Dieser Prozess ist die fundamentale Quelle der Irreversibilität: Zeit entsteht dort, wo Wirkung in Masse „gebremst“ wird.

\subsection{Zeit als Ordnungsrelation}
Da $t$ nicht fundamental vorausgesetzt wird, ergibt es sich als Sekundärgröße aus der Wirkungsdichte. Die messbare Dauer $\Delta t$ ist das Verhältnis von akkumulierter Wirkung zu Systemenergie:
\begin{equation}
    \Delta t = \sigma_P \frac{i}{E}.
\end{equation}
Zeit ist somit nichts anderes als die sequentielle Ordnung irreversibler Ticks. Ein "Anhalten der Zeit" entspricht dem Zustand $\dot{S} = 0$.

\section{Irreversibilität statt Temperatur}

Die klassische Definition der Temperatur $T^{-1} = \partial S / \partial E$ suggeriert einen thermischen Gleichgewichtszustand. Wir ersetzen diesen Begriff durch die \textbf{Irreversibilitätsrate} $\mathcal{R}$, die den gerichteten Wirkungsfluss beschreibt.

Aus der Definition von $S$ und $i$ folgt:
\begin{equation}
    \frac{dS}{dE} = k_B \frac{\partial}{\partial E} \left( \frac{E \Delta t}{\sigma_P} \right) \approx \frac{k_B \Delta t}{\sigma_P}.
\end{equation}
Daraus definieren wir $\mathcal{R}$ als die Energieabgabe pro Entropieeinheit:
\begin{equation}
    \mathcal{R} \equiv \frac{dE}{dS} = \frac{\sigma_P}{k_B \Delta t}.
\end{equation}
\textbf{Interpretation:}
\begin{itemize}
    \item Für große $\Delta t$ (makroskopische, träge Systeme) geht $\mathcal{R} \to 0$. Das System verhält sich reversibel.
    \item Für $\Delta t \to t_P$ (Planck-Zeit) erreicht $\mathcal{R}$ das Maximum. Das System strahlt maximal irreversibel.
\end{itemize}

\section{Anwendung: Skaleninvarianz und Schwarze Löcher}

\subsection{Der makroskopische Grenzfall (Low-Load)}
Für gewöhnliche Materie der Masse $m$ ist die gravitative Kopplung $\chi(m) = G m^2 / (\hbar c^3)$ vernachlässigbar. Der Tick-Index bleibt weit unter dem Sättigungslimit:
\begin{equation}
    i \ll i_{\text{max}} = \frac{c^4}{G}.
\end{equation}
Die Raumzeit-Metrik bleibt klassisch, die Irreversibilitätsrate $\mathcal{R}$ ist durch die molekulare Zeitskala bestimmt und entspricht der thermischen Umgebungstemperatur.

\subsection{Das Schwarze Loch (Saturation)}
Ein Schwarzes Loch repräsentiert den Zustand, in dem die Wirkungsdichte die strukturelle Kapazität der Raumzeit erreicht ($i \to i_{\text{max}}$).
Die Bekenstein-Hawking-Entropie ergibt sich direkt aus der Zählung der Oberflächenzellen:
\begin{equation}
    N_{\text{ticks}} \approx 4\pi \frac{r_s^2}{\ell_P^2} = 4\pi \chi(M).
\end{equation}
Die Hawking-Strahlung ist in diesem Bild keine thermische Fluktuation, sondern die geometrisch erzwungene Irreversibilität $\mathcal{R}_{\text{BH}}$. Da $\Delta t \sim r_s/c$, folgt:
\begin{equation}
    \mathcal{R}_{\text{BH}} \propto \frac{\sigma_P}{k_B (GM/c^3)} \propto \frac{1}{M}.
\end{equation}
Dies reproduziert exakt das Verhalten $T_H \propto 1/M$, jedoch ohne Rückgriff auf statistische Ensembles. Das Schwarze Loch ist ein Kinematik-Transformator, der Masse durch hohe Tick-Dichte in Strahlung konvertiert.

\section{Der Pfeil der Zeit: Entropie als Bremse}

Der fundamentale Pfeil der Zeit ergibt sich in diesem Framework nicht aus einer statistischen Unwahrscheinlichkeit, sondern aus einer gerichteten energetischen Transformation: Die Umwandlung von kohärenter Quantenwirkung ($hf$) in gravitative Trägheit ($mc^2$).

Wir definieren die Zeitrate $\dot{t}$ als invers proportional zur Zander-Entropie $S_Z$. Ein Anstieg der geometrischen Bremslast führt zu einer lokalen Zeitdilatation. Der „Fluss“ der Zeit ist somit die fortschreitende Sättigung des $\sigma_P$-Gitter mit Masse-Information. 

\section{Thermal Spacetime Feedback: Der universelle Atem}

Um den drohenden „Massentod“ (Stasis) zu vermeiden, postuliert das Framework ein Feedback-Gesetz: Bei Erreichen einer kritischen Sättigungsdichte $\rho_{max} \sim 1/\sigma_P^2$ tritt eine elastische Inversion der Wirkung ein. Die akkumulierte Trägheit wird schlagartig in hochenergetische Oszillationen ($hf$) zurückgeführt. Dieser „Bounce“ verhindert den nihilistischen Stillstand und überführt den Pfeil der Zeit in einen zyklischen Puls – einen kosmischen Atemzug zwischen reiner Aktion und notwendiger Trägheit.

\section{Fazit: Die Auflösung des Zeit-Paradoxons}

Wir haben gezeigt, dass Zeit kein leerer Parameter ist, in dem universelle Prozesse ablaufen, sondern das \textbf{Abfallprodukt der Wirkung}. Das Universum altert, weil es „schwerer“ wird – nicht an Masse im Sinne der Erhaltungssätze, sondern an gebundener geometrischer Information.

Die Zander-Gleichung der Zeit lautet:
\begin{equation}
    \text{Arrow of Time} \equiv \frac{d}{dt} \left( k_B \ln \frac{E_m}{E_{hf}} \right) \ge 0, \quad \text{bis zur Bounce-Sättigung}.
\end{equation}

Das Universum begann als reiner, masseloser Spin-Zustand und entwickelt sich hin zu einem Zustand maximaler geometrischer Trägheit, um schließlich durch Feedback wiedergeboren zu werden.

\end{document}
